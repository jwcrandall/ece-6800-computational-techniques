\documentclass[main.tex]{subfiles}
\begin{document}

\href{https://www2.seas.gwu.edu/~simhaweb/quantum/modules/module4/module4.html}{Module 4: Quantum Linear Algebra Part 2}\\
\href{https://www2.seas.gwu.edu/~simhaweb/quantum/modules/module4/problems3.html}{Module-4 solved Problems}

\begin{enumerate}

\item[] \textbf{In-Class Exercise 1:} \textbf{Q.} Compute the following tensor products in column format: \textbf{A.}
    \begin{enumerate}
        \item[1.] $|+\rangle \otimes |+\rangle$
        \begin{align*}
            |+\rangle \otimes |+\rangle     & = \left[\begin{array}{l}\frac{1}{\sqrt{2}} \\ \frac{1}{\sqrt{2}} \end{array}\right]   
                                            \otimes\left[\begin{array}{l} \frac{1}{\sqrt{2}} \\ \frac{1}{\sqrt{2}}\end{array}\right] \\
                                            & = \left[\begin{array}{l} \frac{1}{\sqrt{2}}
                                            \left[\begin{array}{l} \frac{1}{\sqrt{2}} \\ \frac{1}{\sqrt{2}} \end{array}\right]\\ 
                                            \frac{1}{\sqrt{2}}
                                            \left[\begin{array}{l} \frac{1}{\sqrt{2}} \\ \frac{1}{\sqrt{2}} \end{array}\right]
                                            \end{array}\right] \\
                                            & = \left[\begin{array}{l} \frac{1}{2} \\ \frac{1}{2} \\
                                            \frac{1}{2} \\ \frac{1}{2} \end{array}\right]
        \end{align*}
        \item[2.] $|+\rangle \otimes|0\rangle$
        \begin{align*}
            |+\rangle \otimes |0\rangle     & = \left[\begin{array}{l}\frac{1}{\sqrt{2}} \\ \frac{1}{\sqrt{2}} \end{array}\right]   
                                            \otimes\left[\begin{array}{l} 1 \\ 0 \end{array}\right] \\
                                            & = \left[\begin{array}{l} \frac{1}{\sqrt{2}}
                                            \left[\begin{array}{l} 1 \\ 0 \end{array}\right]\\ 
                                            \frac{1}{\sqrt{2}}
                                            \left[\begin{array}{l} 1 \\ 0 \end{array}\right]
                                            \end{array}\right] \\
                                            & = \left[\begin{array}{l} \frac{1}{\sqrt{2}} \\ 0 \\
                                            \frac{1}{\sqrt{2}} \\ 0 \end{array}\right]
        \end{align*}
    \end{enumerate}

\item[] \textbf{In-Class Exercise 2:} \textbf{Q.} Evaluate the inner product $\langle|0\rangle \otimes|+\rangle||+\rangle \otimes|0\rangle\rangle$. \textbf{A.}
    \begin{align*}
        \langle|0\rangle \otimes|+\rangle
        ||+\rangle \otimes|0\rangle\rangle      & = (\langle 0 \mid + \rangle)(\langle + \mid 0\rangle)\\
                                                & = \left(\left[\begin{array}{ll} 1 & 0 \end{array}\right]
                                                \left[\begin{array}{l} \frac{1}{\sqrt{2}} \\ \frac{1}{\sqrt{2}} \end{array}\right]\right)
                                                \left(\left[\begin{array}{ll} \frac{1}{\sqrt{2}} & \frac{1}{\sqrt{2}} \end{array}\right]
                                                \left[\begin{array}{l} 1 \\ 0 \end{array}\right]\right)\\
                                                & = \frac{1}{2}
    \end{align*}
    
\item[] \textbf{In-Class Exercise 3:} \textbf{Q.} Fill in the missing steps in the matrix tensor derivation above. \textbf{A.}
    \begin{align*}
    |v\rangle                       & = \left[\begin{array}{l} v_{1} \\ v_{2} \end{array}\right] \\
    |w\rangle                       & = \left[\begin{array}{l} w_{1} \\ w_{2} \end{array}\right] \\
    A                               & = \left[\begin{array}{ll} a_{11} & a_{12} \\ a_{21} & a_{22} \end{array}\right] \\
    B                               & = \left[\begin{array}{ll} b_{11} & b_{12} \\ b_{21} & b_{22} \end{array}\right] \\
    A|v\rangle \otimes B|w\rangle   & = \left[\begin{array}{l} a_{11} v_{1}+a_{12} v_{2} \\ a_{21} v_{1}+a_{22} v_{2} \end{array}\right] 
                                    \otimes\left[\begin{array}{l} b_{11} w_{1}+b_{12} w_{2} \\ b_{21} w_{1}+b_{22} w_{2} \end{array}\right] \\
                                    & = \left[\begin{array}{ll} \left(a_{11} v_{1}+a_{12} v_{2}\right) 
                                    & {\left[\begin{array}{l} b_{11} w_{1}+b_{12} w_{2} \\ b_{21} w_{1}+b_{22} w_{2} \end{array}\right]} \\
                                    \left(a_{21} v_{1}+a_{22} v_{2}\right) & {\left[\begin{array}{l} b_{11} w_{1}+b_{12} w_{2} \\
                                    b_{21} w_{1}+b_{22} w_{2} \end{array}\right]} \end{array}\right] \\
                                    & = \left[\begin{array}{l} 
                                    a_{11}v_{1}b_{11}w_{1} + a_{11}v_{1}b_{12}w_{2} + a_{12}v_{2}b_{11}w_{1} + a_{12}v_{2}b_{12}w_{2}\\ 
                                    a_{11}v_{1}b_{21}w_{1} + a_{11}v_{1}b_{22}w_{2} + a_{12}v_{2}b_{21}w_{1} + a_{12}v_{2}b_{22}w_{2}\\
                                    a_{21}v_{1}b_{11}w_{1} + a_{21}v_{1}b_{12}w_{2} + a_{22}v_{2}b_{11}w_{1} + a_{22}v_{2}b_{12}w_{2}\\
                                    a_{21}v_{1}b_{21}w_{1} + a_{21}v_{1}b_{22}w_{2} + a_{22}v_{2}b_{21}w_{1} + a_{22}v_{2}b_{22}w_{2}\\
                                    \end{array}\right]\\
                                    & = \left[\begin{array}{llll} 
                                    a_{11}b_{11} & a_{11}b_{12} & a_{12}b_{11} & a_{12}b_{12}\\ 
                                    a_{11}b_{21} & a_{11}b_{22} & a_{12}b_{21} & a_{12}b_{22}\\
                                    a_{21}b_{11} & a_{21}b_{12} & a_{22}b_{11} & a_{22}b_{12}\\
                                    a_{21}b_{21} & a_{21}b_{22} & a_{22}b_{21} & a_{22}b_{22}\\
                                    \end{array}\right]
                                    \left[\begin{array}{l} v_{1} w_{1} \\ v_{1} w_{2} \\ v_{2} w_{1} \\ v_{2} w_{2} \end{array}\right]\\
                                    & = \left[\begin{array}{ll} 
                                    a_{11} \left[\begin{array}{ll} b_{11} & b_{12} \\ b_{21} & b_{22} \end{array}\right] 
                                    & a_{12}\left[\begin{array}{ll} b_{11} & b_{12} \\ b_{21} & b_{22} \end{array}\right] \\
                                    a_{21}\left[\begin{array}{ll} b_{11} & b_{12} \\ b_{21} & b_{22} \end{array}\right] 
                                    & a_{22} \left[\begin{array}{ll} b_{11} & b_{12} \\ b_{21} & b_{22} \end{array}\right]\end{array}\right]
                                    \left[\begin{array}{l} v_{1} w_{1} \\ v_{1} w_{2} \\ v_{2} w_{1} \\ v_{2} w_{2} \end{array}\right]\\
                                    & = \left[\begin{array}{ll} 
                                    a_{11} \left[\begin{array}{ll} b_{11} & b_{12} \\ b_{21} & b_{22} \end{array}\right] 
                                    & a_{12}\left[\begin{array}{ll} b_{11} & b_{12} \\ b_{21} & b_{22} \end{array}\right] \\
                                    a_{21}\left[\begin{array}{ll} b_{11} & b_{12} \\ b_{21} & b_{22} \end{array}\right] 
                                    & a_{22} \left[\begin{array}{ll} b_{11} & b_{12} \\ b_{21} & b_{22} \end{array}\right]\end{array}\right]
                                    |v\rangle \otimes |w\rangle
    \end{align*}    

\item[] \textbf{In-Class Exercise 4:} \textbf{Q.} Write down the single qubit matrices $|0\rangle\langle 0|$, $|1\rangle\langle1|$, and then calculate
$$|0\rangle\langle 0|\otimes I+| 1\rangle\langle 1| \otimes X$$
where $X=\left[\begin{array}{ll}0 & 1 \\ 1 & 0\end{array}\right]$. The result is one of the most important operators in quantum computing called $C_{NOT}$. \textbf{A.}
    \begin{align*}
        |0\rangle\langle 0|             & = \left[\begin{array}{ll} 1 & 0 \\ 0 & 0 \end{array}\right] \\
        |1\rangle\langle 1|             & = \left[\begin{array}{ll} 0 & 0 \\ 0 & 1 \end{array}\right] \\
        |0\rangle\langle 0|\otimes I 
        + |1\rangle\langle1|\otimes X   & = \left[\begin{array}{ll} 1 & 0 \\ 0 & 0 \end{array}\right]
                                        \otimes \left[\begin{array}{ll}1 & 0 \\ 0 & 1\end{array}\right]
                                        + \left[\begin{array}{ll} 0 & 0 \\ 0 & 1 \end{array}\right]
                                        \otimes \left[\begin{array}{ll}0 & 1 \\ 1 & 0\end{array}\right]\\
                                        & = \left[\begin{array}{llll}1&0&0&0\\0&1&0&0\\0&0&0&1\\0&0&1&0\end{array}\right]
    \end{align*}

\item[] \textbf{In-Class Exercise 5:} Recall that $H$ is the Hadamard operator $H=\left[\begin{array}{cc}\frac{1}{\sqrt{2}} & \frac{1}{\sqrt{2}} \\ \frac{1}{\sqrt{2}} & -\frac{1}{\sqrt{2}}\end{array}\right]$
\begin{enumerate}
    \item[1.] \textbf{Q.} Compute and compare $H|0\rangle \otimes H|0\rangle$ 
    and $\left(H \otimes H\right)\left(|0\rangle \otimes|0\rangle\right)$ \textbf{A.}
    \begin{align*}
        H|0\rangle \otimes H|0\rangle   & = \left[\begin{array}{cc}\frac{1}{\sqrt{2}} & \frac{1}{\sqrt{2}} \\ 
                                        \frac{1}{\sqrt{2}} & -\frac{1}{\sqrt{2}}\end{array}\right] 
                                        \left[\begin{array}{c} 1 \\ 0 \end{array}\right]
                                        \otimes \left[\begin{array}{cc}\frac{1}{\sqrt{2}} & \frac{1}{\sqrt{2}} \\ 
                                        \frac{1}{\sqrt{2}} & -\frac{1}{\sqrt{2}}\end{array}\right] 
                                        \left[\begin{array}{c} 1 \\ 0 \end{array}\right] \\
                                        & = \left[\begin{array}{ll} \frac{1}{\sqrt{2}} 
                                        & {\left[\begin{array}{l} \frac{1}{\sqrt{2}} \\ \frac{1}{\sqrt{2}} \end{array}\right]} \\
                                        \frac{1}{\sqrt{2}} & {\left[\begin{array}{l} \frac{1}{\sqrt{2}} \\ \frac{1}{\sqrt{2}} \end{array}\right]} \end{array}\right]\\
                                        & = \left[\begin{array}{l}\frac{1}{2}\\\frac{1}{2}\\\frac{1}{2}\\\frac{1}{2}\end{array}\right]\\
        \left(H \otimes H\right)
        \left(|0\rangle
        \otimes|0\rangle\right)         & = \left[\begin{array}{llll} 
                                        \frac{1}{\sqrt{2}} 
                                        & {\left[\begin{array}{ll} \frac{1}{\sqrt{2}} & \frac{1}{\sqrt{2}} \\ 
                                        \frac{1}{\sqrt{2}} & - \frac{1}{\sqrt{2}} \end{array}\right]}
                                        & \frac{1}{\sqrt{2}} 
                                        & {\left[\begin{array}{ll} \frac{1}{\sqrt{2}} & \frac{1}{\sqrt{2}} \\ 
                                        \frac{1}{\sqrt{2}} & - \frac{1}{\sqrt{2}} \end{array}\right]}\\
                                        \frac{1}{\sqrt{2}} 
                                        & {\left[\begin{array}{ll} \frac{1}{\sqrt{2}} & \frac{1}{\sqrt{2}} \\ 
                                        \frac{1}{\sqrt{2}} & - \frac{1}{\sqrt{2}} \end{array}\right]}
                                        & - \frac{1}{\sqrt{2}} 
                                        & {\left[\begin{array}{ll} \frac{1}{\sqrt{2}} & \frac{1}{\sqrt{2}} \\ 
                                        \frac{1}{\sqrt{2}} & - \frac{1}{\sqrt{2}} \end{array}\right]}
                                        \end{array}\right]
                                        \left[\begin{array}{ll} 1 & {\left[\begin{array}{l} 1 \\ 0 \end{array}\right]} \\
                                        0 & {\left[\begin{array}{l} 1 \\ 0 \end{array}\right]} \end{array}\right]\\
                                        & = \left[\begin{array}{llll}
                                        \frac{1}{2}&\frac{1}{2}&\frac{1}{2}&\frac{1}{2}\\
                                        \frac{1}{2}&-\frac{1}{2}&\frac{1}{2}&-\frac{1}{2}\\
                                        \frac{1}{2}&\frac{1}{2}&-\frac{1}{2}&-\frac{1}{2}\\
                                        \frac{1}{2}&-\frac{1}{2}&-\frac{1}{2}&\frac{1}{2}
                                        \end{array}\right]
                                        \left[\begin{array}{l}1\\0\\0\\0\end{array}\right]\\
                                        & = \left[\begin{array}{l}\frac{1}{2}\\\frac{1}{2}\\\frac{1}{2}\\\frac{1}{2}\end{array}\right]
    \end{align*}
    \item[2.] \textbf{Q.} Show that $(H \otimes H)\left(|0\rangle \otimes|0\rangle\right)=\frac{1}{2}\left(|0\rangle \otimes|0\rangle+|0\rangle \otimes|1\rangle+|1\rangle \otimes|0\rangle+|1\rangle \otimes|1\rangle\right)$ \textbf{A.}
    \begin{align*}
        (H \otimes H)
        \left(|0\rangle\otimes|0\rangle\right)  & = \left[\begin{array}{l}\frac{1}{2}\\\frac{1}{2}\\\frac{1}{2}\\\frac{1}{2}\end{array}\right]\\
                                                & = \frac{1}{2}\left[\begin{array}{l}1\\1\\1\\1\end{array}\right]\\
                                                & = \frac{1}{2}\left(
                                                \left[\begin{array}{ll} 1 & {\left[\begin{array}{l} 1 \\ 0 \end{array}\right]} \\
                                                0 & {\left[\begin{array}{l} 1 \\ 0 \end{array}\right]} \end{array}\right]
                                                + \left[\begin{array}{ll} 1 & {\left[\begin{array}{l} 0 \\ 1 \end{array}\right]} \\
                                                0 & {\left[\begin{array}{l} 0 \\ 1 \end{array}\right]} \end{array}\right]
                                                + \left[\begin{array}{ll} 0 & {\left[\begin{array}{l} 1 \\ 0 \end{array}\right]} \\
                                                1 & {\left[\begin{array}{l} 1 \\ 0 \end{array}\right]} \end{array}\right]
                                                + \left[\begin{array}{ll} 0 & {\left[\begin{array}{l} 0 \\ 1 \end{array}\right]} \\
                                                1 & {\left[\begin{array}{l} 0 \\ 1 \end{array}\right]} \end{array}\right]
                                                \right)\\
                                                & = \frac{1}{2}\left(|0\rangle\otimes|0\rangle
                                                +|0\rangle\otimes|1\rangle
                                                +|1\rangle\otimes|0\rangle
                                                +|1\rangle \otimes|1\rangle\right)
    \end{align*}
\end{enumerate}

\item[] \textbf{In-Class Exercise 6:} \textbf{Q.} Let's explore this non-equivalence. Consider the first way (tensoring two vectors):
$$|\psi\rangle=(\alpha|0\rangle+\beta|1\rangle) \otimes(\gamma|0\rangle+\delta|1\rangle)$$
and the second way (using the larger basis vectors):
$$|\phi\rangle=a_{00}|00\rangle+a_{01}|01\rangle+a_{10}|10\rangle+a_{11}|11\rangle$$
For each $|\phi\rangle$ below
\begin{enumerate}
    \item[1.]$|\phi\rangle=\frac{1}{\sqrt{2}}(|10\rangle+|11\rangle)$
    \item[2.]$|\phi\rangle=\frac{1}{\sqrt{2}}(|01\rangle+|10\rangle)$
\end{enumerate}
 identify the coefficients $a_{00}, \ldots, a_{11}$, then reason about whether the equation $|\psi\rangle=|\phi\rangle$ can be solved for $\alpha, \beta, \gamma, \delta$. \textbf{A.}
 
    \begin{align*}
        |\phi\rangle    & = \frac{1}{\sqrt{2}}(|10\rangle+|11\rangle)
                        = \left[\begin{array}{l} 0 \\ 0 \\ \frac{1}{\sqrt{2}} \\ \frac{1}{\sqrt{2}} \end{array}\right]
                        \therefore \alpha_{00} = 0, \alpha_{01} = 0, \alpha_{10} = \frac{1}{\sqrt{2}}, \alpha_{11} = \frac{1}{\sqrt{2}}\\
        |\psi\rangle    & = |\phi\rangle \text{ when } \alpha = 0, \beta = 1, 
                        \gamma = \frac{1}{\sqrt{2}}, \text{ and } \delta = \frac{1}{\sqrt{2}}\\ 
        |\phi\rangle    & = \frac{1}{\sqrt{2}}(|01\rangle+|10\rangle)
                        = \left[\begin{array}{l} 0 \\ \frac{1}{\sqrt{2}} \\ \frac{1}{\sqrt{2}} \\ 0 \end{array}\right]
                        \therefore \alpha_{00} = 0, \alpha_{01} = \frac{1}{\sqrt{2}}, \alpha_{10} = \frac{1}{\sqrt{2}}, \alpha_{11} = 0\\
        |\psi\rangle    & = |\phi\rangle \text{ cannot be solved for } \alpha, \beta, \gamma, \text{ and } \delta
    \end{align*}
    
    For the second example, although we can use the basis of the larger space and build a linear combination in the larger space, we cannot construct the larger vector by tensoring two smaller vectors, one from each space.

\item[] \textbf{In-Class Exercise 7:} \textbf{Q.} Complete the missing proof steps above (propositions 4.3, 4.6(ii), 4.12).
    \textbf{A.} Proof of Proposition 4.3: The operator tensor product $A \otimes B$ satisfies
    \begin{align*}
    (A \otimes B)
    \left(\alpha_{1}\left|v_{1}\right\rangle 
    \otimes\left|w_{1}\right\rangle
    +\alpha_{2}\left|v_{2}\right\rangle 
    \otimes\left|w_{2}\right\rangle\right)          & = \alpha_{1}A\left|v_{1}\right\rangle
                                                    \otimes B\left|w_{1}\right\rangle
                                                    +\alpha_{2}A\left|v_{2}\right\rangle
                                                    \otimes B\left|w_{2}\right\rangle \tag{Bilinearity}\\
    (A \otimes B)
    \left(\left(\alpha_{1}\left|v_{1}\right\rangle
    +\alpha_{2}\left|v_{2}\right\rangle\right) 
    \otimes|w\rangle\right)                         & = A\left(\alpha_{1} \left|v_{1}\right\rangle
                                                    +\alpha_{2} \left|v_{2}\right\rangle\right)\otimes B|w\rangle \tag{Bilinearity}\\
                                                    & = \left(\alpha_{1} A\left|v_{1}\right\rangle
                                                    +\alpha_{2} A\left|v_{2}\right\rangle\right)\otimes B|w\rangle 
                                                    \tag{Distribute Operator}\\
                                                    & = \alpha_{1} A\left|v_{1}\right\rangle 
                                                    \otimes B|w\rangle+\alpha_{2}A\left|v_{2}\right\rangle
                                                    \otimes B|w\rangle \tag{Tensor Vectors}
    \end{align*}
    and symmetrically,
    \begin{align*}
        (A \otimes B)
        \left(\left|v_{1}\right\rangle 
        \otimes \beta_{1}\left|w_{1}\right\rangle
        +\left|v_{2}\right\rangle 
        \otimes \beta_{2}\left|w_{2}\right\rangle\right)    & = A\left|v_{1}\right\rangle 
                                                            \otimes \beta_{1} B\left|w_{1}\right\rangle
                                                            +A\left|v_{2}\right\rangle 
                                                            \otimes \beta_{2} B\left|w_{2}\right\rangle \tag{Bilineariy}\\
        (A \otimes B) \left(|v\rangle 
        \otimes\left(\beta_{1}\left|w_{1}\right\rangle
        +\beta_{2}\left|w_{2}\right\rangle\right)\right)    & = A|v\rangle \otimes\left(\beta_{1} B\left|w_{1}\right\rangle
                                                            +\beta_{2} B\left|w_{2}\right\rangle\right) \tag{Bilineariy}\\
                                                            & = A|v\rangle \otimes \beta_{1} B\left|w_{1}\right\rangle+A|v\rangle 
                                                            \otimes \beta_{2} B\left|w_{2}\right\rangle \tag{Tensor Vectors}
    \end{align*}
    Application: This says that the tensored operator $A \otimes B$ is linear, which we will take advantage of when simplifying quantum operations.
    
    \textbf{A.} Proof of Proposition 4.6(ii):
    \begin{align*}
        (A \otimes B)^{T}   & = \left[\begin{array}{ccc} a_{11} B & \ldots & a_{1 n} B \\ 
                            \vdots & \vdots & \vdots \\ 
                            a_{n 1} B & \ldots & a_{n n} B \end{array}\right]^{T} \\
                            & = \left[\begin{array}{ccc} a_{11} B^{T} & \ldots & a_{n 1} B^{T} \\
                            \vdots & \vdots & \vdots \\ 
                            a_{1 n} B^{T} & \ldots & a_{n n} B^{T} \end{array}\right] \\
                            & = A^{T} \otimes B^{T}
    \end{align*}
    
    \textbf{A.} Proof of Proposition 4.12: If the $\left|v_{i}\right\rangle\mathrm{'s}$ are orthonormal and the $\left|w_{j}\right\rangle\mathrm{'s}$ are orthonormal in proposition 4.11, then so are the tensored eigenvectors $\left|v_{i}\right\rangle \otimes\left|w_{j}\right\rangle$. Suppose $A$ has orthonormal eigenvectors $\left|v_{1}\right\rangle, \ldots,\left|v_{m}\right\rangle$ and eigenvalues $\lambda_{1}, \ldots, \lambda_{m}$, and $B$ has orthonormal eigenvectors $\left|w_{1}\right\rangle, \ldots,\left|w_{n}\right\rangle$ with eigenvalues $\gamma_{1}, \ldots, \gamma_{n}$. Then $A \otimes B$ has $m n$ orthonormal eigenvectors $\left|v_{i}\right\rangle \otimes\left|w_{j}\right\rangle$ with corresponding eigenvalues $\lambda_{i} \gamma_{j}.
    
    \begin{align*}
        (A \otimes B)\left(\left|v_{i}\right\rangle 
        \otimes\left|w_{j}\right\rangle\right)      & = A\left|v_{i}\right\rangle 
                                                    \otimes B\left|w_{j}\right\rangle \\
                                                    & = \lambda_{i}\left|v_{i}\right\rangle 
                                                    \otimes \gamma_{j}\left|w_{j}\right\rangle \\
                                                    & = \lambda_{i} \gamma_{j}\left(\left|v_{i}\right\rangle 
                                                    \otimes \left|w_{j}\right\rangle\right)
    \end{align*}
    

\item[] \textbf{In-Class Exercise 8:} \textbf{Q.} Calculate the matrix form of the projectors \textbf{A.}
    \begin{enumerate}
        \begin{align*}
            |0 \otimes 0\rangle\langle 0 \otimes 0| & = \left[\begin{array}{ll} 1 & {\left[\begin{array}{l} 1 \\ 0 \end{array}\right]} \\
                                                    0 & {\left[\begin{array}{l} 1 \\ 0 \end{array}\right]} \end{array}\right]
                                                    \left[\begin{array}{ll} 1 & 0 \\
                                                    {\left[\begin{array}{ll} 1 & 0 \end{array}\right]} & 
                                                    {\left[\begin{array}{ll} 1 & 0 \end{array}\right]} \end{array}\right]\\
                                                    & = \left[\begin{array}{l} 1 \\ 0 \\ 0 \\ 0 \end{array}\right]
                                                    \left[\begin{array}{llll} 1 & 0 & 0 & 0 \end{array}\right]\\
                                                    & = \left[\begin{array}{llll} 1&0&0&0\\0&0&0&0\\0&0&0&0\\0&0&0&0\end{array}\right]\\
            |0 \otimes 1\rangle\langle 0 \otimes 1| & = \left[\begin{array}{ll} 1 & {\left[\begin{array}{l} 0 \\ 1 \end{array}\right]} \\
                                                    0 & {\left[\begin{array}{l} 0 \\ 1 \end{array}\right]} \end{array}\right]
                                                    \left[\begin{array}{ll} 1 & 0 \\
                                                    {\left[\begin{array}{ll} 0 & 1 \end{array}\right]} & 
                                                    {\left[\begin{array}{ll} 0 & 1 \end{array}\right]} \end{array}\right]\\
                                                    & = \left[\begin{array}{l} 0 \\ 1 \\ 0 \\ 0 \end{array}\right]
                                                    \left[\begin{array}{llll} 0 & 1 & 0 & 0 \end{array}\right]\\
                                                    & = \left[\begin{array}{llll} 0&0&0&0\\0&1&0&0\\0&0&0&0\\0&0&0&0\end{array}\right]\\
            |1 \otimes 0\rangle\langle 1 \otimes 0| & = \left[\begin{array}{ll} 0 & {\left[\begin{array}{l} 1 \\ 0 \end{array}\right]} \\
                                                    1 & {\left[\begin{array}{l} 1 \\ 0 \end{array}\right]} \end{array}\right]
                                                    \left[\begin{array}{ll} 0 & 1 \\
                                                    {\left[\begin{array}{ll} 1 & 0 \end{array}\right]} & 
                                                    {\left[\begin{array}{ll} 1 & 0 \end{array}\right]} \end{array}\right]\\
                                                    & = \left[\begin{array}{l} 0 \\ 0 \\ 1 \\ 0 \end{array}\right]
                                                    \left[\begin{array}{llll} 0 & 0 & 1 & 0 \end{array}\right]\\
                                                    & = \left[\begin{array}{llll} 0&0&0&0\\0&0&0&0\\0&0&1&0\\0&0&0&0\end{array}\right]\\
            |1 \otimes 1\rangle\langle 1 \otimes 1| & = \left[\begin{array}{ll} 0 & {\left[\begin{array}{l} 0 \\ 1 \end{array}\right]} \\
                                                    1 & {\left[\begin{array}{l} 0 \\ 1 \end{array}\right]} \end{array}\right]
                                                    \left[\begin{array}{ll} 0 & 1 \\
                                                    {\left[\begin{array}{ll} 0 & 1 \end{array}\right]} & 
                                                    {\left[\begin{array}{ll} 0 & 1 \end{array}\right]} \end{array}\right]\\
                                                    & = \left[\begin{array}{l} 0 \\ 0 \\ 0 \\ 1 \end{array}\right]
                                                    \left[\begin{array}{llll} 0 & 0 & 0 & 1 \end{array}\right]\\
                                                    & = \left[\begin{array}{llll} 0&0&0&0\\0&0&0&0\\0&0&0&0\\0&0&0&1\end{array}\right]
        \end{align*}
    \end{enumerate}

\item[] \textbf{In-Class Exercise 9:} \textbf{Q.} Use the just-described approach above to write $|0\rangle\langle 0|\otimes| 1\rangle\langle 0|$ as a single outer-product using the 2-qubit basis vectors, and then confirm your results by working through the matrix versions. \textbf{A.}

    \begin{align*}
        |0\rangle\langle0| \otimes |1\rangle\langle0|   & = |01\rangle \langle00|\\
                                                        & = \left[\begin{array}{l} 0 \\ 1 \\ 0 \\ 0 \end{array}\right]
                                                        \left[\begin{array}{llll} 1 & 0 & 0 & 0 \end{array}\right]\\
                                                        & = \left[\begin{array}{llll} 0&0&0&0\\1&0&0&0\\0&0&0&0\\0&0&0&0\end{array}\right]
    \end{align*}
    
\item[] \textbf{In-Class Exercise 10:} Compute the projector $|0,+\rangle\langle 0,+|$ in two ways:
    \begin{enumerate}
        \item[1.] Write out the column and row forms of $|0,+\rangle$ and then tensor.
        \item[2.] Tensor the smaller projectors using Proposition $4.5$.
    \end{enumerate}
    
    \begin{align*}
        |0,+\rangle\langle 0,+| & = \left[\begin{array}{l} \frac{1}{\sqrt{2}} \\ \frac{1}{\sqrt{2}} \\ 0 \\ 0 \end{array}\right]
                                \left[\begin{array}{llll} \frac{1}{\sqrt{2}} & \frac{1}{\sqrt{2}} & 0 & 0 \end{array}\right]\\
                                & = \left[\begin{array}{llll} \frac{1}{2}&\frac{1}{2}&0&0\\\frac{1}{2}&\frac{1}{2}&0&0\\
                                0&0&0&0\\0&0&0&0\end{array}\right]\\
        P_{0} \otimes P_{+}     & = |0\rangle\langle 0|\otimes| +\rangle\langle +| \\
                                & = |0 \otimes +\rangle\langle 0 \otimes +| \\
                                & = P_{0 \otimes +}\\
                                & = \left[\begin{array}{llll} \frac{1}{2}&\frac{1}{2}&0&0\\\frac{1}{2}&\frac{1}{2}&0&0\\
                                0&0&0&0\\0&0&0&0\end{array}\right]
    \end{align*}

\item[] \textbf{In-Class Exercise 11:} Compute the matrix resulting from adding the operators $|0\rangle\langle 1|+| 1\rangle\langle 0|$ and apply that to $|1\rangle$.
    \begin{align*}
        \left(|0\rangle\langle 1|+| 1\rangle\langle 0|\right)|1\rangle  & = \left[\begin{array}{ll} 0 & 1 \\ 1 & 0 \end{array}\right]
                                                                        \left[\begin{array}{l}0\\1\end{array}\right]\\
                                                                        & = \left[\begin{array}{l}1\\0\end{array}\right] 
    \end{align*}

\item[] \textbf{In-Class Exercise 12:} Use both the outer-product and regular matrix approaches to show $Z=H X H$.
    \begin{align*}
        Z   & = H X H \\
            & = \left[\begin{array}{ll} \frac{1}{\sqrt{2}} & \frac{1}{\sqrt{2}} \\ 
            \frac{1}{\sqrt{2}} & -\frac{1}{\sqrt{2}} \end{array}\right]
            \left[\begin{array}{ll} 0 & 1 \\ 1 & 0 \end{array}\right]
            \left[\begin{array}{ll} \frac{1}{\sqrt{2}} & \frac{1}{\sqrt{2}} \\ 
            \frac{1}{\sqrt{2}} & -\frac{1}{\sqrt{2}} \end{array}\right] \\
            & = \left[\begin{array}{ll} \frac{1}{\sqrt{2}} & \frac{1}{\sqrt{2}} \\ 
            -\frac{1}{\sqrt{2}} & \frac{1}{\sqrt{2}} \end{array}\right]
            \left[\begin{array}{ll} \frac{1}{\sqrt{2}} & \frac{1}{\sqrt{2}} \\ 
            \frac{1}{\sqrt{2}} & -\frac{1}{\sqrt{2}} \end{array}\right] \\
            & = \left[\begin{array}{ll} 1 & 0 \\ 0 & -1 \end{array}\right] \\
            & = \frac{1}{\sqrt{2}}(|0\rangle\langle 0|+| 1\rangle\langle 0|+| 0\rangle\langle 1|-| 1\rangle\langle 1|)
            \left(|0\rangle\langle 1|+| 1\rangle\langle 0|\right)
            \frac{1}{\sqrt{2}}(|0\rangle\langle 0|+| 1\rangle\langle 0|+| 0\rangle\langle 1|-| 1\rangle\langle 1|)\\
            & = \frac{1}{\sqrt{2}} (|0\rangle\langle 0|+| 1\rangle\langle 0|+| 0\rangle\langle 1|-| 1\rangle\langle 1|)
            \frac{1}{\sqrt{2}} (|0\rangle\langle 0|-|0\rangle\langle 1|+| 1\rangle\langle 0|+| 1\rangle\langle 1|)\\
            & = \frac{1}{2} (2|0\rangle\langle 0|-2| 1\rangle\langle 1|)\\
            & = \left[\begin{array}{ll} 1 & 0 \\ 0 & -1 \end{array}\right]
    \end{align*}

\item[] \textbf{In-Class Exercise 13:}  Show that the four Bell vectors do in fact form a basis for the 2-qubit vector space.
    \begin{align*}
        \left|\Phi^{+}\right\rangle & = \frac{1}{\sqrt{2}}(|00\rangle+|11\rangle)
                                    = \left[\begin{array}{l} \frac{1}{\sqrt{2}} \\ 0 \\ 0 \\ \frac{1}{\sqrt{2}} \end{array}\right]\\ 
        \left|\Phi^{-}\right\rangle & = \frac{1}{\sqrt{2}}(|00\rangle-|11\rangle)
                                    = \left[\begin{array}{l} \frac{1}{\sqrt{2}} \\ 0 \\ 0 \\ -\frac{1}{\sqrt{2}} \end{array}\right]\\ 
        \left|\Psi^{+}\right\rangle & = \frac{1}{\sqrt{2}}(|01\rangle+|10\rangle)
                                    = \left[\begin{array}{l} 0 \\ \frac{1}{\sqrt{2}} \\ \frac{1}{\sqrt{2}} \\ 0 \end{array}\right]\\ 
        \left|\Psi^{-}\right\rangle & = \frac{1}{\sqrt{2}}(|01\rangle-|10\rangle)
                                    = \left[\begin{array}{l} 0 \\ \frac{1}{\sqrt{2}} \\ -\frac{1}{\sqrt{2}} \\ 0 \end{array}\right]\\ 
    \end{align*}
    $\left|\Phi^{+}\right\rangle, \left|\Phi^{-}\right\rangle, \left|\Psi^{+}\right\rangle, \left|\Psi^{-}\right\rangle$ are linearly independent, and for the 2-qubit vector space $V$:
    $$V = \operatorname{span}\left(\left|\Phi^{+}\right\rangle,\left|\Phi^{-}\right\rangle,\left|\Psi^{+}\right\rangle,\left|\Psi^{-}\right\rangle\right)$$

\end{enumerate}
\end{document}