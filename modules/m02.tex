\documentclass[main.tex]{subfiles}
\begin{document}

\href{https://www2.seas.gwu.edu/~simhaweb/quantum/modules/module2/module2.html}{Quantum Linear Algebra Part I}

\begin{enumerate}

\item[] \textbf{In-Class Exercise 1:} Review \href{https://www2.seas.gwu.edu/~simhaweb/quantum/modules/module2/problems2.html#complex}{these examples} and then solve:

    \begin{enumerate}
        \item[a.] \textbf{Q.} Compute $|z|$ when $z = 4-3i$ \textbf{A.} $|z| \triangleq \sqrt{a^{2}+b^{2}} = \sqrt{4^2 + (-3)^2} = \sqrt{25} = 5$ 
        
        \item[b.] \textbf{Q.} Express $3-3 i$ in polar form. \textbf{A.}
        \begin{align*}
            z       & = a+ib \\
                    & = r(\cos (\theta)+i \sin (\theta)) \\
            r       & = \sqrt{a^2 + b^2} \\ 
                    & = 3\sqrt{2} \\
            \theta  & = \tan^{-1}(\frac{b}{a}) \\ 
                    & = \tan^{-1}(-1) \\ 
                    & = \frac{\pi}{4}\\ 
            z       & =3\sqrt{2}(\cos (-\frac{\pi}{4})+i \sin (-\frac{\pi}{4})) 
        \end{align*}
        
        \item[c.] \textbf{Q.} Show that Im $(2 i(1+4 i)-3 i(2-i))=-4$ \textbf{A.} 
        \begin{align*}
            \text{Im}(2i(1+4i)-3i(2-i)) & = \text{Im}((-8 + 2i) - (3 - 6i))\\
                                        & = \text{Im}(-11 - 4i)\\
                                        & = -4
        \end{align*}
        
        \item[d.] \textbf{Q.} Write $\frac{1}{(1-i)(3+i)}$ in $a+ib$ form. \textbf{A.}
        \begin{align*}
            \frac{1}{(1-i)(3+i)} &= \frac{1}{4-2i}\\
                                    &= \frac{1}{2(2-i)}\\
                                    &= \frac{2+i}{2(2-i)(2+i)}\\
                                    &= \frac{2+i}{10}\\
                                    &= 0.2 + 0.1i
        \end{align*}
        
        \item[e.] \textbf{Q.} If $z=i^{\frac{1}{3}}$, what is the phase of $z^{*}$ in degrees? \textbf{A.}
        \begin{align*}
            z^{*} &= -i^{\frac{1}{3}}\\
                    &= \cos \left(-\frac{5 \pi}{6}\right)+i \sin \left(-\frac{5 \pi}{6}\right)\\
            \therefore \theta &= -\frac{5\pi}{6}
        \end{align*}
    \end{enumerate}

\item[] \textbf{In-Class Exercise 2:} \textbf{Q.} Suppose $z=z^{*}$ for a complex number $z$. What can you infer about the imaginary part of $z$? \textbf{A.} A complex number is equal to its complex conjugate if its imaginary part is zero, or equivalently, if the number is real. 

\item[] \textbf{In-Class Exercise 3:} Prove the two results stated above, and one more

    \begin{enumerate}
        \item[a.] \textbf{Q.} $\langle u \mid \alpha v+\beta w\rangle=\alpha\langle u \mid v\rangle+\beta\langle u \mid w\rangle$ \textbf{A.}
        \begin{align*}
            \langle u \mid \alpha v+\beta w\rangle &= \langle u \mid (\alpha \mid v \rangle + \beta \mid w\rangle)\\
            &=\alpha\langle u \mid v\rangle+\beta\langle u \mid w\rangle
        \end{align*}
        
        \item[b.] \textbf{Q.} $\langle\alpha u+\beta v \mid w\rangle=\alpha^{*}\langle u \mid w\rangle+\beta^{*}\langle v \mid w\rangle$ \textbf{A.}
        \begin{align*}
            \langle\alpha u+\beta v \mid w\rangle &= (\alpha^{*}\langle u \mid + \beta^{*} \langle v \mid) \mid w\rangle)\\
            &=\alpha^{*}\langle u \mid w\rangle+\beta^{*}\langle v \mid w\rangle
        \end{align*}
        
        \item[c.] \textbf{Q.} $\langle u \mid v\rangle=\langle v \mid u\rangle^{*}$ \textbf{A.}
        \begin{align*}
            \langle u \mid v \rangle & = \mid u \rangle^{\dagger}  \langle v \mid^{\dagger}\\
            &=\langle v \mid u\rangle^{\dagger}\\
            &=\langle v \mid u\rangle^{*}
        \end{align*}   
        Note that $\langle v \mid u \rangle$ is a scalar, so the Hermitian conjugate is just the complex conjugate, i.e., (\langle v \mid u \rangle)^{\dagger}=\langle v \mid u \rangle^{*}
        
    \end{enumerate}

\item[] \textbf{In-Class Exercise 4:} Review \href{https://www2.seas.gwu.edu/~simhaweb/quantum/modules/module2/problems2.html#vectors}{these examples} and then solve the following. Given $|u\rangle=\left(\frac{1}{\sqrt{2}}, \frac{i}{\sqrt{2}}\right)$, $|v\rangle=\left(\frac{1}{\sqrt{2}}, -\frac{i}{\sqrt{2}}\right)$, $w=(1,0)$, $\alpha=i$, $\beta=-\sqrt{2}i$, calculate (without converting $\sqrt{2}$ to decimal format):

    \begin{enumerate}
        \item[a.] \textbf{Q.} $\alpha|v\rangle$, $|\alpha v\rangle$, $\langle\alpha u|$, $\alpha^{*}\langle u |$; \textbf{A.}
        \begin{align*}
            \alpha|v\rangle         & = |\alpha v\rangle\\ 
                                    & = i \left[\begin{array}{l}\frac{1}{\sqrt{2}}\\-\frac{i}{\sqrt{2}}\end{array}\right]\\ 
                                    & = \left[\begin{array}{l}\frac{i}{\sqrt{2}}\\\frac{1}{\sqrt{2}}\end{array}\right]\\
            \langle\alpha u|        & = \left[\begin{array}{ll} -\frac{i}{\sqrt{2}} & -\frac{1}{\sqrt{2}}\end{array}\right] \\
            \alpha^{*}\langle u |   &= -i \left[\begin{array}{ll} \frac{1}{\sqrt{2}} & -\frac{i}{\sqrt{2}}\end{array}\right] \\
                                    &= \left[\begin{array}{ll} -\frac{i}{\sqrt{2}} & \frac{1}{\sqrt{2}}\end{array}\right]
        \end{align*}
        
        \item[b.] \textbf{Q.} $\langle u \mid v\rangle$; \textbf{A.}
        \begin{align*}
            \langle u \mid v\rangle & = \left[\begin{array}{ll} \frac{1}{\sqrt{2}} & -\frac{i}{\sqrt{2}} \end{array}\right] \left[\begin{array}{r} \frac{1}{\sqrt{2}} \\ -\frac{i}{\sqrt{2}} \end{array}\right]\\
                                    & = 0
        \end{align*} 
        
        \item[c.] \textbf{Q.} $|u\rangle\langle v|,| u\rangle\langle v|| w\rangle$, and $\langle v \mid w\rangle|u\rangle$, and compare the latter two results; \textbf{A.}
        \begin{align*}
            |u\rangle\langle v|             & = \left[\begin{array}{r} \frac{1}{\sqrt{2}} \\ \frac{i}{\sqrt{2}} \end{array}\right] \left[\begin{array}{ll} \frac{1}{\sqrt{2}} & \frac{i}{\sqrt{2}} \end{array}\right]\\ 
                                            & = \left[\begin{array}{cc} \frac{1}{2} & \frac{i}{2} \\ \frac{i}{2} & -\frac{1}{2} \end{array}\right]\\
            |u\rangle\langle v|| w\rangle   & = \left[\begin{array}{cc} \frac{1}{2} & \frac{i}{2} \\ \frac{i}{2} & -\frac{1}{2} \end{array}\right] \left[\begin{array}{r} 1 \\ 0 \end{array}\right]\\
                                            & = \left[\begin{array}{r} \frac{1}{2} \\ \frac{i}{2} \end{array}\right]\\
            \langle v\mid w\rangle|u\rangle & = \left[\begin{array}{ll} \frac{1}{\sqrt{2}} & \frac{i}{\sqrt{2}} \end{array}\right] 
                                                \left[\begin{array}{r} 1 \\ 0 \end{array}\right] 
                                                \left[\begin{array}{r} \frac{1}{\sqrt{2}} \\ \frac{i}{\sqrt{2}} \end{array}\right]\\
                                            & = \left[\begin{array}{r} \frac{1}{2} \\ \frac{i}{2} \end{array}\right]
        \end{align*}
        The second and third results are equivalent.
        
        \item[d.] \textbf{Q.} $\|u\rangle|^{2},\langle u \mid u\rangle$ \textbf{A.}
        \begin{align*}
            \| u\rangle|^{2}   & = |\frac{1}{\sqrt{2}}|^2 + |\frac{i}{\sqrt{2}}|^2\\
                                    & = \frac{1}{2} + \frac{1}{2} \\
                                    & = 1 \\
            \langle u \mid u\rangle & = \left[\begin{array}{ll} \frac{1}{\sqrt{2}} & -\frac{i}{\sqrt{2}} \end{array}\right] \left[\begin{array}{r} \frac{1}{\sqrt{2}} \\ \frac{i}{\sqrt{2}} \end{array}\right]\\
                                    & = 1
        \end{align*}
        
        \item[e.] \textbf{Q.} $\langle w|\alpha| u\rangle+\beta|v\rangle\rangle$ and $\alpha\langle w \mid u\rangle+\beta\langle w \mid v\rangle$; \textbf{A.}
        \begin{align*}
            \langle w|\alpha| u\rangle+\beta|v\rangle\rangle            & = \left[\begin{array}{ll} 1 & 0 \end{array}\right] \left( i \left[\begin{array}{r} \frac{1}{\sqrt{2}} \\ \frac{i}{\sqrt{2}} \end{array}\right] 
                                                                            + -\sqrt{2}i \left[\begin{array}{l}\frac{1}{\sqrt{2}}\\-\frac{i}{\sqrt{2}}\end{array}\right] \right) \\            
                                                                        & = \left[\begin{array}{ll} 1 & 0 \end{array}\right] \left[\begin{array}{r} \frac{i(1-\sqrt{2})}{\sqrt{2}} \\ -\frac{1+\sqrt{2}}{\sqrt{2}} \end{array}\right] \\
                                                                        & = \frac{i(1-\sqrt{2})}{\sqrt{2}}\\
            \alpha\langle w \mid u\rangle+\beta\langle w \mid v\rangle  & = i \left[\begin{array}{ll} 1 & 0 \end{array}\right] \left[\begin{array}{r} \frac{1}{\sqrt{2}} \\ \frac{i}{\sqrt{2}} \end{array}\right] 
                                                                            + -\sqrt{2}i \left[\begin{array}{ll} 1 & 0 \end{array}\right] \left[\begin{array}{l}\frac{1}{\sqrt{2}}\\-\frac{i}{\sqrt{2}}\end{array}\right] \\
                                                                        & = \frac{i}{\sqrt{2}} - i\\
                                                                        & = \frac{i(1-\sqrt{2})}{\sqrt{2}}
        \end{align*}
        
        \item[f.] \textbf{Q.} $\langle \alpha \mid u\rangle+\beta|v\rangle|w\rangle$ and $\alpha^{*} \langle u \mid w\rangle+\beta^{*}\langle v \mid w\rangle$. \textbf{A.}
        \begin{align*}
            \langle \alpha \mid u\rangle+\beta|v\rangle|w\rangle                & = \left( i \left[\begin{array}{r} \frac{1}{\sqrt{2}} \\ \frac{i}{\sqrt{2}} \end{array}\right] 
                                                                                    + -\sqrt{2}i \left[\begin{array}{l}\frac{1}{\sqrt{2}}\\-\frac{i}{\sqrt{2}}\end{array}\right] \right) \left[\begin{array}{r} 1 \\ 0 \end{array}\right]\\
                                                                                & = \left[\begin{array}{r} \frac{i(1-\sqrt{2})}{\sqrt{2}} \\ -\frac{1+\sqrt{2}}{\sqrt{2}} \end{array}\right] \left[\begin{array}{r} 1 \\ 0 \end{array}\right] \\
                                                                                & = ? \\
            \alpha^{*} \langle u \mid w\rangle+\beta^{*}\langle v \mid w\rangle & = -i \left[\begin{array}{ll} \frac{1}{\sqrt{2}} & -\frac{i}{\sqrt{2}} \end{array}\right] \left[\begin{array}{r} 1 \\ 0 \end{array}\right]
                                                                                    + \sqrt{2}i \left[\begin{array}{ll} \frac{1}{\sqrt{2}} & \frac{i}{\sqrt{2}} \end{array}\right] \left[\begin{array}{r} 1 \\ 0 \end{array}\right]  \\
                                                                                & = \frac{-i}{\sqrt{2}} + i\\
                                                                                & = \frac{i(\sqrt{2}-1)}{\sqrt{2}}
        \end{align*}
        
    \end{enumerate}

\item[] \textbf{In-Class Exercise 5:} Suppose $|u\rangle=(1,0,0,0)$, $|v\rangle=(0,0,1,0)$ and $W=\operatorname{span}(|u\rangle,|v\rangle)$.

    \begin{enumerate}
        \item[a.] \textbf{Q.} Show that $|u\rangle$, $|v\rangle$ are linearly independent. \textbf{A.} Vectors $\mathbf{u}, \mathbf{v}$ are linearly independent if the only solution to the equation $x_{1} \mathbf{v}_{1}+x_{2} \mathbf{v}_{2}=\mathbf{0}$ is $x_{1}=x_{2}=0$. No other values of $x_1$ or $x_2$ exists that satisfy this condition.
        \begin{align*}
            0 \left[\begin{array}{l} 1 \\ 0 \\ 0 \\ 0 \end{array}\right] + 0 \left[\begin{array}{l} 0 \\ 0 \\ 1 \\ 0 \end{array}\right] = \left[\begin{array}{l} 0 \\ 0 \\ 0 \\ 0 \end{array}\right]
        \end{align*}
        
        
        \item[b.] \textbf{Q.} What is the dimension of $W$? \textbf{A.} 
        
        \begin{itemize}
            \item \textbf{Span}: Given a collection of vectors $\left|v_{1}\right\rangle,\left|v_{2}\right\rangle, \ldots,\left|v_{k}\right\rangle$, the span of these vectors is the set of all possible linear combinations of these (with complex scalars): $\operatorname{span}\left(\left|v_{1}\right\rangle,\left|v_{2}\right\rangle, \ldots,\left|v_{k}\right\rangle\right) \triangleq\left\{\alpha_{1}\left|v_{1}\right\rangle+\alpha_{2}\left|v_{2}\right\rangle+\ldots+\alpha_{k}\left|v_{k}\right\rangle: \alpha_{i} \in \mathrm{C}\right\}$
            \item \textbf{Vector Space}: The vector space $V$ is a set of vectors such that, for any subset $\left|v_{1}\right\rangle,\left|v_{2}\right\rangle, \ldots,\left|v_{k}\right\rangle$ where $\left|v_{i}\right\rangle \in V$, $\operatorname{span}\left(\left|v_{1}\right\rangle,\left|v_{2}\right\rangle, \ldots,\left|v_{k}\right\rangle\right) \subseteq V$.
            \item \textbf{Basis}: There are multiple equivalent definitions for the Basis:
            \begin{enumerate}
                \item [1.] A basis for a given vector space $V$, is a set of vectors $\left|v_{1}\right\rangle,\left|v_{2}\right\rangle, \ldots,\left|v_{n}\right\rangle$ from $V$ such that:
                \begin{enumerate}
                    \item [i.] $\left|v_{1}\right\rangle,\left|v_{2}\right\rangle, \ldots,\left|v_{n}\right\rangle$ are linearly independent.
                    \item [ii.] $V=\operatorname{span}\left(\left|v_{1}\right\rangle,\left|v_{2}\right\rangle, \ldots,\left|v_{n}\right\rangle\right)$ 
                \end{enumerate}
                \item [2.] A basis for a given vector space $V$, is a set of vectors $\left|v_{1}\right\rangle,\left|v_{2}\right\rangle, \ldots,\left|v_{n}\right\rangle$ from $V$ such that any vector $|u\rangle \in V$ is uniquely expressible as a linear combination of the $\left|v_{i}\right\rangle$ 's. That is, there is only one linear combination $|u\rangle=\sum_{i} \alpha_{i}\left|v_{i}\right\rangle$.
            \end{enumerate}
            Note: all bases have the same number of vectors. That is, if one basis has $n$ vectors, so does any other basis.
            \item \textbf{Dimension}: The dimension of a vector space $V$ is the number of vectors in any basis, written as $\operatorname{dim}(V)$. As a consequence, if $\operatorname{dim}(V)=n$, any $n$ linearly independent vectors from $V$ forms a basis for $V$.
        \end{itemize}
        
        \begin{align*}
            W = \operatorname{span}\left(\left|u\right\rangle,\left|v\right\rangle\right) 
            & \triangleq\left\{\alpha_{1}\left|u\right\rangle+\alpha_{2}\left|v\right\rangle: \alpha_{i} \in C\right\} \\
            & = \alpha_1 \left[\begin{array}{l} 1 \\ 0 \\ 0 \\ 0 \end{array}\right] + \alpha_2 \left[\begin{array}{l} 0 \\ 0 \\ 1 \\ 0 \end{array}\right] \subseteq V \\
            \operatorname{dim}(V) & = 2 
        \end{align*}
        
        \item[c.] \textbf{Q.} Express $|x\rangle=(-i, 0, i+1,0)$ in terms of $|u\rangle$, $|v\rangle)$. \textbf{A.}
        \begin{align*}
            -i |u\rangle + i |v\rangle + |v\rangle = \left[\begin{array}{l} -i \\ 0 \\ i+1 \\ 0 \end{array}\right] 
        \end{align*}
        
        \item[d.] \textbf{Q.} Is $|u\rangle$, $|v\rangle)$ a basis for $W$? Explain. \textbf{A.} Yes, $|u\rangle$, $|v\rangle)$ are linearly independent and $W=\operatorname{span}(|u\rangle,|v\rangle)$, which combined constitute a basis as defined in b.
        \item[e.] \textbf{Q.} Is $W=\mathbb{C}^{k}$ for any value of $k$? \textbf{A.} No, only when $k=4$. The special vector space $\mathbb{C}^{n}=\left\{|v\rangle=\left(v_{1}, \ldots, v_{n}\right): v_{i} \in \mathbb{C}\right\}$ is the set of all vectors with $n$ elements, where each element is a complex number. In 5b we can see that $W = \operatorname{span}\left(\left|u\right\rangle,\left|v\right\rangle\right)$ is composed of two vectors with four elements each of complex value. 

    \end{enumerate}
    
\item[] \textbf{In-Class Exercise 6:} For the vectors $|00\rangle$, $|01\rangle$, $|10\rangle$, $|11\rangle$ defined above, compute the outer products

    \begin{enumerate}
        \item[a.] \textbf{Q.} $|00\rangle\langle 00|$ \textbf{A.}
        \begin{align*}
            & = \left[\begin{array}{l} 1 \\ 0 \\ 0 \\ 0 \end{array}\right] \left[\begin{array}{llll} 1 & 0 & 0 & 0 \end{array}\right] \\
            & = \left[\begin{array}{llll} 1 & 0 & 0 & 0 \\ 0 & 0 & 0 & 0 \\ 0 & 0 & 0 & 0 \\ 0 & 0 & 0 & 0 \end{array}\right] 
        \end{align*}
        \item[b.] \textbf{Q.} $|01\rangle\langle 01|$ \textbf{A.}
        \begin{align*}
            & = \left[\begin{array}{l} 0 \\ 1 \\ 0 \\ 0 \end{array}\right] \left[\begin{array}{llll} 0 & 1 & 0 & 0 \end{array}\right] \\
            & = \left[\begin{array}{llll} 0 & 0 & 0 & 0 \\ 0 & 1 & 0 & 0 \\ 0 & 0 & 0 & 0 \\ 0 & 0 & 0 & 0 \end{array}\right] 
        \end{align*}
        \item[c.] \textbf{Q.} $|10\rangle\langle 10|$ \textbf{A.}
        \begin{align*}
            & = \left[\begin{array}{l} 0 \\ 0 \\ 1 \\ 0 \end{array}\right] \left[\begin{array}{llll} 0 & 0 & 1 & 0 \end{array}\right] \\
            & = \left[\begin{array}{llll} 0 & 0 & 0 & 0 \\ 0 & 0 & 0 & 0 \\ 0 & 0 & 1 & 0 \\ 0 & 0 & 0 & 0 \end{array}\right] 
        \end{align*}
        \item[d.] \textbf{Q.} $|11\rangle\langle 11|$ \textbf{A.}
        \begin{align*}
            & = \left[\begin{array}{l} 0 \\ 0 \\ 0 \\ 1 \end{array}\right] \left[\begin{array}{llll} 0 & 0 & 0 & 1 \end{array}\right] \\
            & = \left[\begin{array}{llll} 0 & 0 & 0 & 0 \\ 0 & 0 & 0 & 0 \\ 0 & 0 & 0 & 0 \\ 0 & 0 & 0 & 1 \end{array}\right] 
        \end{align*}
    \end{enumerate}

\item[] \textbf{In-Class Exercise 7:} Using the vectors $|0\rangle$, $|1\rangle$, compute the vectors

    \begin{enumerate}
        \item[a.] \textbf{Q.} $\left|h_{1}\right\rangle=\frac{1}{\sqrt{2}}(|0\rangle+|1\rangle)$ \textbf{A.}
        \begin{align*}
            & = \frac{1}{\sqrt{2}} \left( \left[\begin{array}{l} 1 \\ 0 \end{array}\right] + \left[\begin{array}{l} 0 \\ 1 \end{array}\right] \right)\\
            & = \left[\begin{array}{l} \frac{1}{\sqrt{2}} \\ \frac{1}{\sqrt{2}} \end{array}\right] 
        \end{align*}
        
        \item[b.] \textbf{Q.} $\left|h_{2}\right\rangle=\frac{1}{\sqrt{2}}(|0\rangle-|1\rangle)$ \textbf{A.}
        \begin{align*}
            & = \frac{1}{\sqrt{2}} \left( \left[\begin{array}{l} 1 \\ 0 \end{array}\right] - \left[\begin{array}{l} 0 \\ 1 \end{array}\right] \right)\\
            & = \left[\begin{array}{l} \frac{1}{\sqrt{2}} \\ -\frac{1}{\sqrt{2}} \end{array}\right]             
        \end{align*}
        
        \item[c.] \textbf{Q.} Show that $\left|h_{1}\right\rangle$, $\left|h_{2}\right\rangle$ are orthonormal (unit length and orthogonal). \textbf{A.} Two vectors $|u\rangle$ and $|v\rangle$ from a vector space $V$ are orthonormal if $\langle u \mid v\rangle=0$ and $|u|=|v|=1$ (that is, each is of unit length).
        \begin{align*}
            \left\langle h_{1} \mid h_{2}\right\rangle  & =  \left[\begin{array}{ll} \frac{1}{\sqrt{2}} & \frac{1}{\sqrt{2}} \end{array}\right] 
                                                            \left[\begin{array}{l} \frac{1}{\sqrt{2}} \\ -\frac{1}{\sqrt{2}} \end{array}\right] \\   
                                                        & = 0 \\  
                \left|\left|h_{1}\right\rangle\right|   & = \sqrt{\left(\frac{1}{\sqrt{2}}\right)^2 + \left(\frac{1}{\sqrt{2}}\right)^2} \\
                                                        & = 1 \\
                \left|\left|h_{2}\right\rangle\right|   & = \sqrt{\left(\frac{1}{\sqrt{2}}\right)^2 + \left(-\frac{1}{\sqrt{2}}\right)^2} \\
                                                        & = 1
        \end{align*}
        
        \item[d.] \textbf{Q.} Show that $\left|h_{1}\right\rangle$, $\left|h_{2}\right\rangle$ are a basis for $\mathbb{C}^2$. [Hint: start by expressing $|0\rangle$, $|1\rangle$ in terms of $\left|h_{1}\right\rangle,\left|h_{2}\right\rangle$.] \textbf{A.} Using definitions in 5a, 5b, and 5e, $\left|h_{1}\right\rangle$, $\left|h_{2}\right\rangle$ are linearly independent 
       \begin{align*}
            0 \left[\begin{array}{l} \frac{1}{\sqrt{2}} \\ \frac{1}{\sqrt{2}} \end{array}\right] 
            + 0 \left[\begin{array}{l} \frac{1}{\sqrt{2}} \\ \frac{1}{-\sqrt{2}} \end{array}\right] & = \left[\begin{array}{l} 0 \\ 0 \end{array}\right]
        \end{align*}
        and
        \begin{align*}
            \mathbb{C}^2 = \operatorname{span} \left( \left|h_{1} \right\rangle, \left|h_{2}\right\rangle \right)  
        \end{align*}
    \end{enumerate}

\item[] \textbf{In-Class Exercise 8:} \textbf{Q.} For any projector $|v\rangle\langle v|$ show that $(|v\rangle\langle v|)^{\dagger}=|v\rangle\langle v|$. [Hint: you can transpose and then conjugate.] Note: a matrix $A$ like $P_{v}=|v\rangle\langle v|$ that satisfies $A^{\dagger}=A$ is called \emph{Hermitian}, as we'll see below. \textbf{A.}

    \begin{align*}
        (|v\rangle\langle v|)^{\dagger} & = (P_{v})^{\dagger} \tag {outerproduct $\left|v_{1}\right\rangle\left\langle v_{1}\right|$ equals projector matrix}\\    
        (P_{v})^{\dagger} & = P_{v} \tag{Hermitian projector}\\
        P_{v} & = |v\rangle\langle v|
    \end{align*}

\item[] \textbf{In-Class Exercise 9:} Using the vectors $|0\rangle$, $|1\rangle$, compute the vectors

    \begin{enumerate}
        \item[a.] \textbf{Q.} $\left|y_{1}\right\rangle=\frac{1}{\sqrt{2}}|0\rangle+\frac{i}{\sqrt{2}}|1\rangle$ \textbf{A.}
        \begin{align*}
            \frac{1}{\sqrt{2}}|0\rangle+\frac{i}{\sqrt{2}}|1\rangle & = \frac{1}{\sqrt{2}} \left[\begin{array}{l} 1 \\ 0 \end{array}\right] 
                                                                        + \frac{i}{\sqrt{2}} \left[\begin{array}{l} 0 \\ 1 \end{array}\right] \\
                                                                    & = \left[\begin{array}{l} \frac{1}{\sqrt{2}} \\ \frac{i}{\sqrt{2}} \end{array}\right] 
        \end{align*}
        
        \item[b.] \textbf{Q.} $\left.\left|y_{2}\right\rangle=\frac{1}{\sqrt{2}}|0\rangle-\frac{i}{\sqrt{2}}|1\rangle\right$ \textbf{A.}
        \begin{align*}
            \frac{1}{\sqrt{2}}|0\rangle-\frac{i}{\sqrt{2}}|1\rangle & = \frac{1}{\sqrt{2}} \left[\begin{array}{l} 1 \\ 0 \end{array}\right] 
                                                                        - \frac{i}{\sqrt{2}} \left[\begin{array}{l} 0 \\ 1 \end{array}\right] \\
                                                                    & = \left[\begin{array}{l} \frac{1}{\sqrt{2}} \\ -\frac{i}{\sqrt{2}} \end{array}\right] 
        \end{align*}
        
        \item[c.] \textbf{Q.} Show that $\left|y_{1}\right\rangle$, $\left|y_{2}\right\rangle$ are a basis for $\mathbb{C}^{2}$. \textbf{A.} $\alpha_1=\alpha_2=0$ are the only values that satisfy
        \begin{align*}
            0 \left[\begin{array}{l} \frac{1}{\sqrt{2}} \\ \frac{i}{\sqrt{2}} \end{array}\right]  
            + 0 \left[\begin{array}{l} \frac{1}{\sqrt{2}} \\ -\frac{i}{\sqrt{2}} \end{array}\right]  & = \left[\begin{array}{l} 0 \\ 0 \end{array}\right]
        \end{align*}
        and
        \begin{align*}
            \mathbb{C}^2 = \operatorname{span} \left( \left|y_{1} \right\rangle, \left|y_{2}\right\rangle \right)  
        \end{align*}
        
        \item[d.] \textbf{Q.} Show the completeness relation for this basis. \textbf{A.} We define the completeness relation as the sum of all outer products (projectors) of the vectors that comprise the basis as the identity matrix  
        \begin{align*}
            I   & = \left|v_{1}\right\rangle\left\langle v_{1}|+\ldots+| v_{n}\right\rangle\left\langle v_{n}\right|= P_{1}+\ldots+P_{n}\\
                & = \left[\begin{array}{l} \frac{1}{\sqrt{2}} \\ \frac{i}{\sqrt{2}} \end{array}\right] 
                    \left[\begin{array}{ll} \frac{1}{\sqrt{2}} & \frac{-i}{\sqrt{2}} \end{array}\right]
                    + \left[\begin{array}{l} \frac{1}{\sqrt{2}} \\ -\frac{i}{\sqrt{2}} \end{array}\right]
                    \left[\begin{array}{ll} \frac{1}{\sqrt{2}} & \frac{i}{\sqrt{2}} \end{array}\right] \\
                & = \left[\begin{array}{ll} 1 & 0 \\ 0 & 1 \end{array}\right]
        \end{align*}
        
    \end{enumerate}

\item[] \textbf{In-Class Exercise 10:} Consider these matrices. (Yes, the latter two have special names.)

    $$
    A=\left[\begin{array}{cc}
    1 & -i \\
    i & 1
    \end{array}\right] \quad B=\left[\begin{array}{cc}
    i & 1 \\
    1 & -i
    \end{array}\right] \quad Y=\left[\begin{array}{cc}
    0 & -i \\
    i & 0
    \end{array}\right] \quad H=\left[\begin{array}{cc}
    \frac{1}{\sqrt{2}} & \frac{1}{\sqrt{2}} \\
    \frac{1}{\sqrt{2}} & -\frac{1}{\sqrt{2}}
    \end{array}\right]
    $$
    
    Compute
    
    \begin{enumerate}
        \item[1.] \textbf{Q.} The adjoints $A^{\dagger}$, $B^{\dagger}$, $Y^{\dagger}$, $H^{\dagger}$ \textbf{A.}
        \begin{align*}
            A^{\dagger} & = \left[\begin{array}{cc} 1 & -i \\ i & 1 \end{array}\right]\\
            B^{\dagger} & = \left[\begin{array}{cc} -i & 1 \\ 1 & i \end{array}\right]\\
            Y^{\dagger} & = \left[\begin{array}{cc} 0 & -i \\ i & 0 \end{array}\right]\\
            H^{\dagger} & = \left[\begin{array}{cc} \frac{1}{\sqrt{2}} & \frac{1}{\sqrt{2}} \\ \frac{1}{\sqrt{2}} & -\frac{1}{\sqrt{2}} \end{array} \right]
        \end{align*}
        
        \item[2.] \textbf{Q.} $A^{\dagger} A, Y^{\dagger} Y \text{ and } H^{\dagger} H$ \textbf{A.}
        \begin{align*}
            A^{\dagger} A   & = \left[\begin{array}{cc} 1 & -i \\ i & 1 \end{array}\right] \left[\begin{array}{cc} 1 & -i \\ i & 1 \end{array}\right]\\
                            & = \left[\begin{array}{cc} 2 & -2i \\ 2i & 2 \end{array}\right] \\
            Y^{\dagger} Y   & = \left[\begin{array}{cc} 0 & -i \\ i & 0 \end{array}\right] \left[\begin{array}{cc} 0 & -i \\ i & 0 \end{array}\right]\\
                            & = \left[\begin{array}{cc} 1 & 0 \\ 0 & 1 \end{array}\right]\\
            H^{\dagger} H   & = \left[\begin{array}{cc} \frac{1}{\sqrt{2}} & \frac{1}{\sqrt{2}} \\ \frac{1}{\sqrt{2}} & -\frac{1}{\sqrt{2}} \end{array} \right]
                                \left[\begin{array}{cc} \frac{1}{\sqrt{2}} & \frac{1}{\sqrt{2}} \\ \frac{1}{\sqrt{2}} & -\frac{1}{\sqrt{2}} \end{array} \right]\\
                            & = \left[\begin{array}{cc} 1 & 0 \\ 0 & 1 \end{array}\right]\\
        \end{align*}
        
        \item[3.] \textbf{Q.} $A A^{\dagger}, Y Y^{\dagger}$ and $H H^{\dagger}$ \textbf{A.}
        \begin{align*}
            A A^{\dagger}   & = \left[\begin{array}{cc} 1 & -i \\ i & 1 \end{array}\right] \left[\begin{array}{cc} 1 & -i \\ i & 1 \end{array}\right]\\
                            & = \left[\begin{array}{cc} 2 & -2i \\ 2i & 2 \end{array}\right] \\
            Y Y^{\dagger}   & = \left[\begin{array}{cc} 0 & -i \\ i & 0 \end{array}\right] \left[\begin{array}{cc} 0 & -i \\ i & 0 \end{array}\right]\\
                            & = \left[\begin{array}{cc} 1 & 0 \\ 0 & 1 \end{array}\right]\\
            H H^{\dagger}   & = \left[\begin{array}{cc} \frac{1}{\sqrt{2}} & \frac{1}{\sqrt{2}} \\ \frac{1}{\sqrt{2}} & -\frac{1}{\sqrt{2}} \end{array} \right]
                                \left[\begin{array}{cc} \frac{1}{\sqrt{2}} & \frac{1}{\sqrt{2}} \\ \frac{1}{\sqrt{2}} & -\frac{1}{\sqrt{2}} \end{array} \right]\\
                            & = \left[\begin{array}{cc} 1 & 0 \\ 0 & 1 \end{array}\right]\\
        \end{align*}
        
    \end{enumerate}
    
    \textbf{Q.} In computing the adjoint, do we get the same result if we first apply conjugation and then transpose? \textbf{A.} Yes
    
\item[] \textbf{In-Class Exercise 11:} \textbf{Q.} Use

    $$A=\left[\begin{array}{cc}1 & -i \\ i & 1\end{array}\right] \quad B=\left[\begin{array}{cc}i & 1 \\ 1 & -i\end{array}\right] \quad \alpha=1+i$$
    
    as examples in confirming (i)-(iv) above. \textbf{A.}
    
    \begin{align*}
        (A^{\dagger})^{\dagger}             & = \left[\begin{array}{cc}1 & -i \\ i & 1\end{array}\right]^{\dagger}\\
                                            & = \left[\begin{array}{cc}1 & -i \\ i & 1\end{array}\right]\\
                                            & = A\\
        (\alpha A)^{\dagger}                & = \left[1+i \left[\begin{array}{cc}1 & -i \\ i & 1\end{array}\right] \right]^{\dagger}\\
                                            & = \left[\begin{array}{cc} 1+i & 1-i \\ -1+i & 1+i \end{array}\right]^{\dagger}\\
                                            & = \left[\begin{array}{cc} 1-i & -1-i \\ 1+i & 1-i \end{array}\right]\\
                                            & = 1-i \left[\begin{array}{cc}1 & -i \\ i & 1\end{array}\right]\\
                                            & = \alpha^{*} A^{\dagger} \\
        (A+B)^{\dagger}                     & = \left( \left[\begin{array}{cc}1 & -i \\ i & 1\end{array}\right] + \left[\begin{array}{cc}i & 1 \\ 1 & -i\end{array}\right] \right)^{\dagger}\\
                                            & = \left( \left[\begin{array}{cc} 1+i & 1-i \\ 1+i & 1-i\end{array}\right] \right)^{\dagger}\\
                                            & = \left[\begin{array}{cc} 1-i & 1-i \\ 1+i & 1+i\end{array}\right]\\
                                            & = \left[\begin{array}{cc} 1 & -i \\ i & 1\end{array}\right] + \left[\begin{array}{cc}-i & 1 \\ 1 & i\end{array}\right]\\
                                            & = A^{\dagger}+B^{\dagger}\\
        (A B)^{\dagger}                     & = \left( \left[\begin{array}{cc}1 & -i \\ i & 1\end{array}\right] \left[\begin{array}{cc}i & 1 \\ 1 & -i\end{array}\right] \right)^{\dagger}\\
                                            & = \left[\begin{array}{cc} 0 & 0 \\ 0 & 0\end{array}\right]^{\dagger}\\
                                            & = \left[\begin{array}{cc} 0 & 0 \\ 0 & 0\end{array}\right]\\
                                            & = \left[\begin{array}{cc} -i & 1 \\ 1 & i\end{array}\right] \left[\begin{array}{cc} 1 & -i \\ i & 1\end{array}\right]\\
                                            & = B^{\dagger} A^{\dagger}
    \end{align*}

\item[] \textbf{In-Class Exercise 12:} \textbf{Q.} If $\alpha$ were not real above, show with an $2 \times 2$ example that $\alpha A$ is not Hermitian when $A$ is Hermitian. \textbf{A.}
    \begin{align*}
        \alpha                  & = i\\
        A                       & = \left[\begin{array}{cc}1 & -i \\ i & 1\end{array}\right]\\
                                & = A^{\dagger}\\ 
        \alpha A                & = \left[\begin{array}{cc}i & 1 \\ -1 & i\end{array}\right]\\
        (\alpha A)^{\dagger}    & = \left[\begin{array}{cc}-i & -1 \\ 1 & -i\end{array}\right]\\
        \alpha A                & \neq (\alpha A)^{\dagger}
    \end{align*}

\item[] \textbf{In-Class Exercise 13:} \textbf{Q.} Use a $2 \times 2$ example to show that the sum of two unitary matrices is not necessarily unitary. \textbf{A.}
    \begin{align*}
        Y                                           & = \left[\begin{array}{cc} 0 & -i \\ i & 0 \end{array}\right]\\
        Y^{\dagger} Y = Y Y^{\dagger}               & = \left[\begin{array}{cc} 0 & -i \\ i & 0 \end{array}\right] \left[\begin{array}{cc} 0 & -i \\ i & 0 \end{array}\right]
                                                    = \left[\begin{array}{cc} 1 & 0 \\ 0 & 1 \end{array}\right]
                                                    = I\\
        H                                           & = \left[\begin{array}{cc} \frac{1}{\sqrt{2}} & \frac{1}{\sqrt{2}} \\ \frac{1}{\sqrt{2}} & -\frac{1}{\sqrt{2}} \end{array} \right]\\
        H^{\dagger} H = H H^{\dagger}               & = \left[\begin{array}{cc} \frac{1}{\sqrt{2}} & \frac{1}{\sqrt{2}} \\ \frac{1}{\sqrt{2}} & -\frac{1}{\sqrt{2}} \end{array} \right]
                                                    \left[\begin{array}{cc} \frac{1}{\sqrt{2}} & \frac{1}{\sqrt{2}} \\ \frac{1}{\sqrt{2}} & -\frac{1}{\sqrt{2}} \end{array} \right]
                                                    = \left[\begin{array}{cc} 1 & 0 \\ 0 & 1 \end{array}\right]
                                                    = I\\
        Y + H                                       & = \left[\begin{array}{cc} 0 & -i \\ i & 0 \end{array}\right] 
                                                    + \left[\begin{array}{cc} \frac{1}{\sqrt{2}} & \frac{1}{\sqrt{2}} \\ \frac{1}{\sqrt{2}} & -\frac{1}{\sqrt{2}} \end{array} \right]\\
                                                    & = \left[\begin{array}{cc} \frac{1}{\sqrt{2}} & \frac{1}{\sqrt{2}} -i \\ \frac{1}{\sqrt{2}} + i & -\frac{1}{\sqrt{2}} \end{array}\right]\\
        (Y+H)^{\dagger}(Y+H) = (Y+H)(Y+H)^{\dagger} & = \left[\begin{array}{cc} \frac{1}{\sqrt{2}} & \frac{1}{\sqrt{2}} -i \\ \frac{1}{\sqrt{2}} + i & -\frac{1}{\sqrt{2}} \end{array}\right]
                                                    \left[\begin{array}{cc} \frac{1}{\sqrt{2}} & \frac{1}{\sqrt{2}} -i \\ \frac{1}{\sqrt{2}} + i & -\frac{1}{\sqrt{2}} \end{array}\right]\\
                                                    & = \left[\begin{array}{cc} 2 & 0 \\ 0 & 2 \end{array}\right]\\
                                                    & \neq I
    \end{align*}

\item[] \textbf{In-Class Exercise 14:} Suppose $A=\left[\begin{array}{ll}2 & 0 \\ 0 & 3\end{array}\right]$ and $|u\rangle=\alpha\left|e_{1}\right\rangle+\beta\left|e_{2}\right\rangle$. Show that $\langle u|A| u\rangle=2|\alpha|^{2}+3|\beta|^{2}$.

        \begin{align*}
            \langle u|A| u\rangle & = \left[\begin{array}{ll} \alpha^* & \beta^* \end{array}\right]  \left[\begin{array}{ll}2 & 0 \\ 0 & 3\end{array}\right] \left[\begin{array}{l} \alpha \\ \beta \end{array}\right]\\
            & = \left[\begin{array}{ll} \alpha^* & \beta^* \end{array}\right] \left[\begin{array}{l} 2 \alpha \\ 3 \beta \end{array}\right]\\
            &=2|\alpha|^{2}+3|\beta|^{2}
        \end{align*}

\item[] \textbf{In-Class Exercise 15:} Using $\left|v_{1}\right\rangle,\left|v_{2}\right\rangle$ as the basis, 

    \begin{enumerate}
        \item[a.] \textbf{Q.} Find the coordinates of $\left|e_{1}\right\rangle,\left|e_{2}\right\rangle$; \textbf{A.}
        \begin{align*}
            \alpha_{1}\left|v_{1}\right\rangle+\alpha_{2}\left|v_{2}\right\rangle                               & = \left|e_{1}\right\rangle\\
            \alpha_{1} \left[\begin{array}{l} \frac{1}{\sqrt{2}} \\ \frac{1}{\sqrt{2}} \end{array}\right]
            + \alpha_{2} \left[\begin{array}{l} \frac{1}{\sqrt{2}} \\ -\frac{1}{\sqrt{2}} \end{array}\right]    & = \left[\begin{array}{l} 1 \\ 0 \end{array}\right]\\
            \left[\begin{array}{ll} \frac{1}{\sqrt{2}} & \frac{1}{\sqrt{2}} 
            \\ \frac{1}{\sqrt{2}} & -\frac{1}{\sqrt{2}} \end{array}\right]
            \left[\begin{array}{c} \alpha_{1} \\ \alpha_{2} \end{array}\right]                                  & = \left[\begin{array}{c} 1 \\ 0 \end{array}\right] \tag{matrix is unitary, inverse is the adjoint} \\
            \left|e_{1}\right\rangle = 
            \left[\begin{array}{ll} \frac{1}{\sqrt{2}} & \frac{1}{\sqrt{2}} 
            \\ \frac{1}{\sqrt{2}} & -\frac{1}{\sqrt{2}} \end{array}\right]
            \left[\begin{array}{c} 1 \\ 0 \end{array}\right]                                                    & = \left[\begin{array}{c} \frac{1}{\sqrt{2}} \\ \frac{1}{\sqrt{2}} \end{array}\right] \tag{new basis}\\
            \alpha_{1}\left|v_{1}\right\rangle+\alpha_{2}\left|v_{2}\right\rangle                               & = \left|e_{2}\right\rangle\\
            \alpha_{1} \left[\begin{array}{l} \frac{1}{\sqrt{2}} \\ \frac{1}{\sqrt{2}} \end{array}\right]
            + \alpha_{2} \left[\begin{array}{l} \frac{1}{\sqrt{2}} \\ -\frac{1}{\sqrt{2}} \end{array}\right]    & = \left[\begin{array}{l} 0 \\ 1 \end{array}\right]\\
            \left[\begin{array}{ll} \frac{1}{\sqrt{2}} & \frac{1}{\sqrt{2}} 
            \\ \frac{1}{\sqrt{2}} & -\frac{1}{\sqrt{2}} \end{array}\right]
            \left[\begin{array}{c} \alpha_{1} \\ \alpha_{2} \end{array}\right]                                  & = \left[\begin{array}{c} 0 \\ 1 \end{array}\right] \tag{matrix is unitary, inverse is the adjoint} \\
            \left|e_{2}\right\rangle = 
            \left[\begin{array}{ll} \frac{1}{\sqrt{2}} & \frac{1}{\sqrt{2}} 
            \\ \frac{1}{\sqrt{2}} & -\frac{1}{\sqrt{2}} \end{array}\right]
            \left[\begin{array}{c} 0 \\ 1 \end{array}\right]                                                    & = \left[\begin{array}{c} \frac{1}{\sqrt{2}} \\ -\frac{1}{\sqrt{2}} \end{array}\right] \tag{new basis}\\
        \end{align*}

        \item[b.] \textbf{Q.} Then, compute $\left|y_{2}\right\rangle=P_{e_{2}}|u\rangle$ in this basis. \textbf{A.}
        \begin{align*}
            P_{e_2}                                                                     & = \left|e_{2}\right\rangle \left\langle e_{2} \right| 
                                                                                        = \left[\begin{array}{l} \frac{1}{\sqrt{2}} \\ -\frac{1}{\sqrt{2}} \end{array}\right] \left[\begin{array}{ll} \frac{1}{\sqrt{2}} & -\frac{1}{\sqrt{2}} \end{array}\right]\\
                                                                                        & = \left[\begin{array}{ll} \frac{1}{2} & -\frac{1}{2} \\ -\frac{1}{2} & \frac{1}{2} \end{array}\right] \\
            \alpha_{1}\left|v_{1}\right\rangle+\alpha_{2}\left|v_{2}\right\rangle       & = \left|u\right\rangle\\
            \alpha_{1} \left[\begin{array}{l} \frac{1}{\sqrt{2}} 
            \\ \frac{1}{\sqrt{2}} \end{array}\right]
            + \alpha_{2} \left[\begin{array}{l} \frac{1}{\sqrt{2}} 
            \\ -\frac{1}{\sqrt{2}} \end{array}\right]                                   & = \left[\begin{array}{c} \frac{\sqrt{3}}{2} \\ \frac{1}{2} \end{array}\right] \\
            \left[\begin{array}{ll} \frac{1}{\sqrt{2}} & \frac{1}{\sqrt{2}} 
            \\ \frac{1}{\sqrt{2}} & -\frac{1}{\sqrt{2}} \end{array}\right]
            \left[\begin{array}{c} \alpha_{1} \\ \alpha_{2} \end{array}\right]          & = \left[\begin{array}{c} \frac{\sqrt{3}}{2} \\ \frac{1}{2} \end{array}\right] \tag{matrix is unitary, inverse is the adjoint} \\
            |u\rangle =
            \left[\begin{array}{cc} \frac{1}{\sqrt{2}} & \frac{1}{\sqrt{2}} 
            \\ \frac{1}{\sqrt{2}} & -\frac{1}{\sqrt{2}} \end{array}\right]
            \left[\begin{array}{c} \frac{\sqrt{3}}{2} 
            \\ \frac{1}{2} \end{array}\right]                                           & = \left[\begin{array}{c} \frac{\sqrt{3}+1}{2 \sqrt{2}} \\ \frac{\sqrt{3}-1}{2 \sqrt{2}} \end{array}\right] \tag{new basis}\\
            P_{e_2} \left|u\right\rangle                                                & = \left[\begin{array}{ll} \frac{1}{2} & -\frac{1}{2} \\ -\frac{1}{2} & \frac{1}{2} \end{array}\right] 
                                                                                        \left[\begin{array}{c} \frac{\sqrt{3}+1}{2 \sqrt{2}} \\ \frac{\sqrt{3}-1}{2 \sqrt{2}} \end{array}\right] \\
            \left|y_{2}\right\rangle                                                    & = \left[\begin{array}{c} \frac{1}{2\sqrt{2}} \\ -\frac{1}{2\sqrt{2}} \end{array}\right]
        \end{align*}
        
        \item[c.] \textbf{Q.} Check your calculations by obtaining $\left|y_{2}\right\rangle$ in the standard basis and converting that to the $\left|v_{1}\right\rangle,\left|v_{2}\right\rangle$ basis. \textbf{A.}
        \begin{align*}
            P_{e_2}                                                                                             & = \left|e_{2}\right\rangle \left\langle e_{2} \right| 
                                                                                                                = \left[\begin{array}{l} 0 \\ 1 \end{array}\right] \left[\begin{array}{ll} 0 & 1 \end{array}\right] 
                                                                                                                = \left[\begin{array}{ll} 0 & 0 \\ 0 & 1 \end{array}\right] \\
            P_{e_2} \left|u\right\rangle                                                                        & = \left[\begin{array}{ll} 0 & 0 \\ 0 & 1 \end{array}\right] 
                                                                                                                \left[\begin{array}{c} \frac{\sqrt{3}}{2} \\ \frac{1}{2} \end{array}\right] \\
            \left|y_{2}\right\rangle                                                                            & = \left[\begin{array}{c} 0 \\ \frac{1}{2} \end{array}\right] \\
            \gamma_{1}\left|v_{1}\right\rangle+\gamma_{2}\left|v_{2}\right\rangle                               & = \left|y_{2}\right\rangle\\
            \gamma_{1} \left[\begin{array}{l} \frac{1}{\sqrt{2}} \\ \frac{1}{\sqrt{2}} \end{array}\right]
            + \gamma_{2} \left[\begin{array}{l} \frac{1}{\sqrt{2}} \\ -\frac{1}{\sqrt{2}} \end{array}\right]    & = \left[\begin{array}{c} 0 \\ \frac{1}{2} \end{array}\right]\\
            \left[\begin{array}{ll} \frac{1}{\sqrt{2}} & \frac{1}{\sqrt{2}} 
            \\ \frac{1}{\sqrt{2}} & -\frac{1}{\sqrt{2}} \end{array}\right]
            \left[\begin{array}{l} \gamma_{1} \\ \gamma_{2} \end{array}\right]                                  & = \left[\begin{array}{c} 0 \\ \frac{1}{2} \end{array}\right] \tag{matrix is unitary, inverse is the adjoint}\\
            \left|y_{2}\right\rangle =
            \left[\begin{array}{ll} \frac{1}{\sqrt{2}} & \frac{1}{\sqrt{2}} 
            \\ \frac{1}{\sqrt{2}} & -\frac{1}{\sqrt{2}} \end{array}\right]
            \left[\begin{array}{c} 0 \\ \frac{1}{2} \end{array}\right]                                          & = \left[\begin{array}{c} \frac{1}{2\sqrt{2}} \\ -\frac{1}{2\sqrt{2}} \end{array}\right] \tag{new basis}\\
        \end{align*}
    \end{enumerate}

\item[] \textbf{In-Class Exercise 16:} This is just a writing exercise to get used to the new symbols. Rewrite the previous exercise with the new symbols. That is, using $|+\rangle,|-\rangle$ as the basis, 
    
    \begin{enumerate}
        \item[a.] \textbf{Q.} Find the coordinates of $|0\rangle,|1\rangle$; \textbf{A.}
        \begin{align*}
            \alpha_{1}\left|+\right\rangle+\alpha_{2}\left|-\right\rangle                                       & = \left|0\right\rangle\\
            \alpha_{1} \left[\begin{array}{l} \frac{1}{\sqrt{2}} \\ \frac{1}{\sqrt{2}} \end{array}\right]
            + \alpha_{2} \left[\begin{array}{l} \frac{1}{\sqrt{2}} \\ -\frac{1}{\sqrt{2}} \end{array}\right]    & = \left[\begin{array}{l} 1 \\ 0 \end{array}\right]\\
            \left[\begin{array}{ll} \frac{1}{\sqrt{2}} & \frac{1}{\sqrt{2}} 
            \\ \frac{1}{\sqrt{2}} & -\frac{1}{\sqrt{2}} \end{array}\right]
            \left[\begin{array}{c} \alpha_{1} \\ \alpha_{2} \end{array}\right]                                  & = \left[\begin{array}{c} 1 \\ 0 \end{array}\right] \tag{matrix is unitary, inverse is the adjoint} \\
            \left|0\right\rangle = 
            \left[\begin{array}{ll} \frac{1}{\sqrt{2}} & \frac{1}{\sqrt{2}} 
            \\ \frac{1}{\sqrt{2}} & -\frac{1}{\sqrt{2}} \end{array}\right]
            \left[\begin{array}{c} 1 \\ 0 \end{array}\right]                                                    & = \left[\begin{array}{c} \frac{1}{\sqrt{2}} \\ \frac{1}{\sqrt{2}} \end{array}\right] \tag{new basis}\\
            \alpha_{1}\left|+\right\rangle+\alpha_{2}\left|-\right\rangle                                       & = \left|1\right\rangle\\
            \alpha_{1} \left[\begin{array}{l} \frac{1}{\sqrt{2}} \\ \frac{1}{\sqrt{2}} \end{array}\right]
            + \alpha_{2} \left[\begin{array}{l} \frac{1}{\sqrt{2}} \\ -\frac{1}{\sqrt{2}} \end{array}\right]    & = \left[\begin{array}{l} 0 \\ 1 \end{array}\right]\\
            \left[\begin{array}{ll} \frac{1}{\sqrt{2}} & \frac{1}{\sqrt{2}} 
            \\ \frac{1}{\sqrt{2}} & -\frac{1}{\sqrt{2}} \end{array}\right]
            \left[\begin{array}{c} \alpha_{1} \\ \alpha_{2} \end{array}\right]                                  & = \left[\begin{array}{c} 0 \\ 1 \end{array}\right] \tag{matrix is unitary, inverse is the adjoint} \\
            \left|1\right\rangle = 
            \left[\begin{array}{ll} \frac{1}{\sqrt{2}} & \frac{1}{\sqrt{2}} 
            \\ \frac{1}{\sqrt{2}} & -\frac{1}{\sqrt{2}} \end{array}\right]
            \left[\begin{array}{c} 0 \\ 1 \end{array}\right]                                                    & = \left[\begin{array}{c} \frac{1}{\sqrt{2}} \\ -\frac{1}{\sqrt{2}} \end{array}\right] \tag{new basis}\\
        \end{align*}

        \item[b.] \textbf{Q.} Then, compute $\left|y_{2}\right\rangle=P_{0}|u\rangle$ in this basis. \textbf{A.}
        \begin{align*}
            P_{0}                                                               & = \left|0\right\rangle \left\langle 0\right| 
                                                                                        = \left[\begin{array}{l} \frac{1}{\sqrt{2}} \\ \frac{1}{\sqrt{2}} \end{array}\right] 
                                                                                        \left[\begin{array}{ll} \frac{1}{\sqrt{2}} & \frac{1}{\sqrt{2}} \end{array}\right]\\
                                                                                & = \left[\begin{array}{ll} \frac{1}{2} & \frac{1}{2} \\ \frac{1}{2} & \frac{1}{2} \end{array}\right] \\
            \alpha_{1}\left|+\right\rangle+\alpha_{2}\left|-\right\rangle       & = \left|u\right\rangle\\
            \alpha_{1} \left[\begin{array}{l} \frac{1}{\sqrt{2}} 
            \\ \frac{1}{\sqrt{2}} \end{array}\right]
            + \alpha_{2} \left[\begin{array}{l} \frac{1}{\sqrt{2}} 
            \\ -\frac{1}{\sqrt{2}} \end{array}\right]                                   & = \left[\begin{array}{c} \frac{\sqrt{3}}{2} \\ \frac{1}{2} \end{array}\right] \\
            \left[\begin{array}{ll} \frac{1}{\sqrt{2}} & \frac{1}{\sqrt{2}} 
            \\ \frac{1}{\sqrt{2}} & -\frac{1}{\sqrt{2}} \end{array}\right]
            \left[\begin{array}{c} \alpha_{1} \\ \alpha_{2} \end{array}\right]          & = \left[\begin{array}{c} \frac{\sqrt{3}}{2} \\ \frac{1}{2} \end{array}\right] \tag{matrix is unitary, inverse is the adjoint} \\
            |u\rangle =
            \left[\begin{array}{cc} \frac{1}{\sqrt{2}} & \frac{1}{\sqrt{2}} 
            \\ \frac{1}{\sqrt{2}} & -\frac{1}{\sqrt{2}} \end{array}\right]
            \left[\begin{array}{c} \frac{\sqrt{3}}{2} 
            \\ \frac{1}{2} \end{array}\right]                                           & = \left[\begin{array}{c} \frac{\sqrt{3}+1}{2 \sqrt{2}} \\ \frac{\sqrt{3}-1}{2 \sqrt{2}} \end{array}\right] \tag{new basis}\\
            P_{0} \left|u\right\rangle                                                & = \left[\begin{array}{ll} \frac{1}{2} & \frac{1}{2} \\ \frac{1}{2} & \frac{1}{2} \end{array}\right] 
                                                                                        \left[\begin{array}{c} \frac{\sqrt{3}+1}{2 \sqrt{2}} \\ \frac{\sqrt{3}-1}{2 \sqrt{2}} \end{array}\right] \\
            \left|y_{2}\right\rangle                                                    & = \left[\begin{array}{c} \frac{\sqrt{3}}{2\sqrt{2}} \\ \frac{\sqrt{3}}{2\sqrt{2}} \end{array}\right]
        \end{align*}
        
        \item[c.] \textbf{Q.} Check your calculations by obtaining $\left|y_{2}\right\rangle$ in the standard basis and converting that to the $|+\rangle,|-\rangle$ basis. \textbf{A.}
        \begin{align*}
            P_{0}                                                                                               & = \left|0\right\rangle \left\langle 0\right| 
                                                                                                                = \left[\begin{array}{l} 1 \\ 0 \end{array}\right] \left[\begin{array}{ll} 1 & 0 \end{array}\right] 
                                                                                                                = \left[\begin{array}{ll} 1 & 0 \\ 0 & 0 \end{array}\right] \\
            P_{0} \left|u\right\rangle                                                                          & = \left[\begin{array}{ll} 0 & 0 \\ 0 & 1 \end{array}\right] 
                                                                                                                \left[\begin{array}{c} \frac{\sqrt{3}}{2} \\ \frac{1}{2} \end{array}\right] \\
            \left|y_{2}\right\rangle                                                                            & = \left[\begin{array}{c} \frac{\sqrt{3}}{2} \\ 0 \end{array}\right] \\
            \gamma_{1}\left|+\right\rangle+\gamma_{2}\left|-\right\rangle                                       & = \left|y_{2}\right\rangle\\
            \gamma_{1} \left[\begin{array}{l} \frac{1}{\sqrt{2}} \\ \frac{1}{\sqrt{2}} \end{array}\right]
            + \gamma_{2} \left[\begin{array}{l} \frac{1}{\sqrt{2}} \\ -\frac{1}{\sqrt{2}} \end{array}\right]    & = \left[\begin{array}{c} \frac{\sqrt{3}}{2} \\ 0 \end{array}\right]\\
            \left[\begin{array}{ll} \frac{1}{\sqrt{2}} & \frac{1}{\sqrt{2}} 
            \\ \frac{1}{\sqrt{2}} & -\frac{1}{\sqrt{2}} \end{array}\right]
            \left[\begin{array}{l} \gamma_{1} \\ \gamma_{2} \end{array}\right]                                  & = \left[\begin{array}{c} \frac{\sqrt{3}}{2} \\ 0 \end{array}\right] \tag{matrix is unitary, inverse is the adjoint}\\
            \left|y_{2}\right\rangle =
            \left[\begin{array}{ll} \frac{1}{\sqrt{2}} & \frac{1}{\sqrt{2}} 
            \\ \frac{1}{\sqrt{2}} & -\frac{1}{\sqrt{2}} \end{array}\right]
            \left[\begin{array}{c} \frac{\sqrt{3}}{2} \\ 0 \end{array}\right]                                   & = \left[\begin{array}{c} \frac{\sqrt{3}}{2\sqrt{2}} \\ \frac{\sqrt{3}}{2\sqrt{2}} \end{array}\right] \tag{new basis}\\
        \end{align*}
    \end{enumerate}

\item[] \textbf{In-Class Exercise 17:} \textbf{Q.} Suppose A is an operator that "flips" the Hadamard basis vectors. That is, $A|+\rangle=|-\rangle$ and $A|-\rangle=|+\rangle$. Use the approach above to derive the matrix for $A$. \textbf{A.}
    \begin{align*}
        A|+\rangle          & = |-\rangle \\
        A|-\rangle          & = |+\rangle \\
        \langle+|+\rangle   & = \left[\begin{array}{ll} \frac{1}{\sqrt{2}} & \frac{1}{\sqrt{2}} \end{array}\right]   
                            \left[\begin{array}{c} \frac{1}{\sqrt{2}} \\ \frac{1}{\sqrt{2}} \end{array}\right] = 1\\
        \langle+|-\rangle   & = \left[\begin{array}{ll} \frac{1}{\sqrt{2}} & \frac{1}{\sqrt{2}} \end{array}\right] 
                            \left[\begin{array}{c} \frac{1}{\sqrt{2}} \\ -\frac{1}{\sqrt{2}} \end{array}\right] = 0 \\
        \langle-|+\rangle   & = \left[\begin{array}{ll} \frac{1}{\sqrt{2}} & -\frac{1}{\sqrt{2}} \end{array}\right] 
                            \left[\begin{array}{c} \frac{1}{\sqrt{2}} \\ \frac{1}{\sqrt{2}} \end{array}\right] = 0 \\
        \langle-|-\rangle   & = \left[\begin{array}{ll} \frac{1}{\sqrt{2}} & -\frac{1}{\sqrt{2}} \end{array}\right] 
                            \left[\begin{array}{c} \frac{1}{\sqrt{2}} \\ -\frac{1}{\sqrt{2}} \end{array}\right] = 1 \\
        A_{i j}             & = \left\langle - |A| + \right\rangle\\
                            & = \left[\begin{array}{ll} 1 \langle+\mid+\rangle & \langle+\mid-\rangle \\ 
                            -1 \langle-\mid+\rangle & -1\langle-\mid-\rangle \end{array}\right] \\
                            & = \left[\begin{array}{ll} 1 & 0 \\ 0 & -1 \end{array}\right]
    \end{align*}
    
    \textbf{Q.} Then, apply the matrix to $|\psi\rangle=\alpha|0\rangle+\beta|1\rangle$. Is the resulting vector orthogonal to $|\psi\rangle$? \textbf{A.}
    \begin{align*}
        A|\psi\rangle                                                       & = \left[\begin{array}{ll} 1 & 0 \\ 0 & -1 \end{array}\right] \left[\begin{array}{c} \alpha \\ \beta \end{array}\right]\\
                                                                            & = \left[\begin{array}{c} \alpha \\ -\beta \end{array}\right]\\
        \left[\begin{array}{c} \alpha \\ \beta \end{array}\right] 
        \cdot \left[\begin{array}{c} \alpha \\ -\beta \end{array}\right]    & \neq 0 \tag{not orthogonal, missing something}\\
    \end{align*}

\item[] \textbf{In-Class Exercise 18:} \textbf{Q.} Show that $|z\rangle \in W$. \textbf{A.} In vector space $V$
    \begin{align*}
        V   & = \operatorname{span}\left\{\left|v_{1}\right\rangle,\left|v_{2}\right\rangle, \ldots,\left|v_{n}\right\rangle\right\} \\
            & = \left\{\alpha_{1}\left|v_{1}\right\rangle+\alpha_{2}\left|v_{2}\right\rangle+\ldots+\alpha_{k}\left|v_{n}\right\rangle: \alpha_{i} \in \mathbb{C}\right\}
    \end{align*}
    consider just the span of the first two of these vectors $W$
    \begin{align*}
        W   & = \operatorname{span}\left\{\left|v_{1}\right\rangle,\left|v_{2}\right\rangle\right\} \\
            & = \left\{\alpha_{1}\left|v_{1}\right\rangle+\alpha_{2}\left|v_{2}\right\rangle: \alpha_{i} \in \mathbb{C}\right\}
    \end{align*}
    and pick three vectors in $W$ and ask if the linear combination is in $W$:
    \begin{align*}
        \left|x_{1}\right\rangle    & = a_{1}\left|v_{1}\right\rangle+a_{2}\left|v_{2}\right\rangle \\
        \left|x_{2}\right\rangle    & = b_{1}\left|v_{1}\right\rangle+b_{2}\left|v_{2}\right\rangle \\
        \left|x_{3}\right\rangle    & = c_{1}\left|v_{1}\right\rangle+c_{2}\left|v_{2}\right\rangle \\
        |z\rangle                   & = \beta_{1}\left|x_{1}\right\rangle+\beta_{2}\left|x_{2}\right\rangle+\beta_{3}\left|x_{3}\right\rangle
    \end{align*}
    $W$ is a vector space, meaning that all linear combinations of any subset of vectors in $W$ is in $W$. Vectors $\left|v_{1}\right\rangle$, and $\left|v_{2}\right\rangle$ are a subset of $W$, vectors $\left|x_{1}\right\rangle$, $\left|x_{2}\right\rangle$, and $\left|x_{3}\right\rangle$, are linear combinations of vectors $\left|v_{1}\right\rangle$ and $\left|v_{2}\right\rangle$, and vector $|z\rangle$ is a linear combination of vectors $\left|x_{1}\right\rangle$, $\left|x_{2}\right\rangle$, and $\left|x_{3}\right\rangle$. Vector $|z\rangle$ is a linear combination of a subset of vectors in $W$ and $\therefore |z\rangle \in W$.
    
\item[] \textbf{In-Class Exercise 19:} \textbf{Q.} Write down the projector $P _$ for the subspace $\operatorname{span}(|-\rangle)$. \textbf{A.}

    \begin{align*}
        P_{-}   & = \left[\begin{array}{c} \frac{1}{\sqrt{2}} \\ -\frac{1}{\sqrt{2}} \end{array}\right] \left[\begin{array}{ll} \frac{1}{\sqrt{2}} & -\frac{1}{\sqrt{2}} \end{array}\right]\\
                & = \left[\begin{array}{cc} \frac{1}{2} & -\frac{1}{2} \\ -\frac{1}{2} & \frac{1}{2} \end{array}\right]
    \end{align*}

    \begin{enumerate}
    \item[1.] \textbf{Q.} Show that $P_{-}|0\rangle=\frac{1}{\sqrt{2}}|-\rangle$. \textbf{A.}
        \begin{align*}
            P_{-}|0\rangle  & = \left[\begin{array}{cc} \frac{1}{2} & -\frac{1}{2} \\ -\frac{1}{2} & \frac{1}{2} \end{array}\right] \left[\begin{array}{c} 1 \\ 0 \end{array}\right]\\
                            & = \left[\begin{array}{c} \frac{1}{2} \\ -\frac{1}{2} \end{array}\right]\\
                            & = \frac{1}{\sqrt{2}}|-\rangle
        \end{align*}
    
    \item[2.] \textbf{Q.} What is the result of applying $P_{-}$to $|+\rangle ?$. \textbf{A.}
        \begin{align*}
            P_{-}|+\rangle  & = \left[\begin{array}{cc} \frac{1}{2} & -\frac{1}{2} \\ -\frac{1}{2} & \frac{1}{2} \end{array}\right] \left[\begin{array}{c} \frac{1}{\sqrt{2}} \\ \frac{1}{\sqrt{2}} \end{array}\right]\\
                            & = \left[\begin{array}{l} 0 \\ 0 \end{array}\right]
        \end{align*}

    \end{enumerate}

\item[] \textbf{In-Class Exercise 20:} Suppose $|w\rangle=\frac{\sqrt{3}}{2}|0\rangle-\frac{1}{2}|1\rangle$.

    \begin{enumerate}
    \item[1.] \textbf{Q.} Show that $\left|w^{\perp}\right\rangle=\frac{1}{2}|0\rangle+\frac{\sqrt{3}}{2}|1\rangle$ is orthogonal and of unit-length. \textbf{A.}
        \begin{align*}
            \left\langle w \mid w^{\perp}\right\rangle          & = \left\langle \frac{\sqrt{3}}{2}\langle0|-\frac{1}{2}\langle1|  \mid \frac{1}{2}|0\rangle+\frac{\sqrt{3}}{2}|1\rangle \right\rangle \\
                                                                & = \frac{\sqrt{3}}{4}\langle 0 \mid 0\rangle - \frac{\sqrt{3}}{4} \left\langle 1 \mid 1 \right\rangle \\
                                                                & = 0 \tag{orthogonal} \\
             \left| |w^{\perp}\rangle \right|             & = \sqrt{\left(\frac{1}{2}\right)^2+\left(\frac{\sqrt{3}}{2}\right)^2}\\
                                                                & = 1 \tag{unit length}
        \end{align*}
    \item[2.] \textbf{Q.} Express $|+\rangle$ in this basis by computing the coefficients $\langle w \mid+\rangle$ and $\left\langle w^{\perp} \mid+\right\rangle$. \textbf{A.}
        \begin{align*}
            \langle w \mid+\rangle          & = \left[\begin{array}{ll} \frac{\sqrt{3}}{2} & -\frac{1}{2} \end{array}\right] \left[\begin{array}{c} \frac{1}{\sqrt{2}} \\ \frac{1}{\sqrt{2}} \end{array}\right]\\
                                            & = \frac{\sqrt{3}-1}{2\sqrt{2}} \\
            \langle w^{\perp} \mid+\rangle  & = \left[\begin{array}{ll} \frac{1}{2} & \frac{\sqrt{3}}{2} \end{array}\right] \left[\begin{array}{c} \frac{1}{\sqrt{2}} \\ \frac{1}{\sqrt{2}} \end{array}\right]\\
                                            & = \frac{\sqrt{3}+1}{2\sqrt{2}} \\
        \end{align*}
        
    \item[3.] \textbf{Q.} Check your calculations by showing $|\langle w \mid+\rangle|^{2}+\left|\left\langle w^{\perp} \mid+\right\rangle\right|^{2}=1$. \textbf{A.}
        \begin{align*}
            \left(\frac{\sqrt{3}-1}{2\sqrt{2}}\right)^2 + \left( \frac{\sqrt{3}+1}{2\sqrt{2}} \right)^2 & = \frac{3 -2\sqrt{3} + 1 + 3 +2\sqrt{3} + 1}{8}\\
                                                                                                        & = 1
        \end{align*}
    \end{enumerate}

\item[] \textbf{In-Class Exercise 21:}

    \begin{enumerate}
    \item[1.] \textbf{Q.} Write down the two projectors $P_{0}, P_{1}$ for $|0\rangle,|1\rangle$ respectively. Then compute the matrix product $P_{0} P_{1}$. \textbf{A.}
        \begin{align*}
            P_{0} = |0 \rangle \langle 0|   & = \left[\begin{array}{c} 1 \\ 0 \end{array}\right] \left[\begin{array}{ll} 1 & 0 \end{array}\right] = \left[\begin{array}{ll} 1 & 0 \\ 0 & 0 \end{array}\right]\\
            P_{1} = |1 \rangle \langle 1|   & = \left[\begin{array}{c} 0 \\ 1 \end{array}\right] \left[\begin{array}{ll} 0 & 1 \end{array}\right] = \left[\begin{array}{ll} 0 & 0 \\ 0 & 1 \end{array}\right]\\
            P_{0}P_{1}                      & = \left[\begin{array}{ll} 1 & 0 \\ 0 & 0 \end{array}\right] \left[\begin{array}{ll} 0 & 0 \\ 0 & 1 \end{array}\right]\\
                                            & = \left[\begin{array}{ll} 1 & 0 \\ 0 & 1 \end{array}\right]
        \end{align*}
    \item[2.] \textbf{Q.} Write down the two projectors $P_{+}, P_{-}$ for $|+\rangle,|-\rangle$, respectively. Then compute the matrix product $P_{+} P_{-}$. \textbf{A.}
        \begin{align*}
            P_{+} = |+ \rangle \langle +|   & = \left[\begin{array}{c} \frac{1}{\sqrt{2}} \\ \frac{1}{\sqrt{2}} \end{array}\right] 
                                                \left[\begin{array}{ll} \frac{1}{\sqrt{2}} & \frac{1}{\sqrt{2}} \end{array}\right] 
                                                = \left[\begin{array}{ll} \frac{1}{2} & \frac{1}{2} \\ \frac{1}{2} & \frac{1}{2} \end{array}\right]\\
            P_{-} = |- \rangle \langle -|   & = \left[\begin{array}{c} \frac{1}{\sqrt{2}} \\ -\frac{1}{\sqrt{2}} \end{array}\right] 
                                                \left[\begin{array}{ll} \frac{1}{\sqrt{2}} & -\frac{1}{\sqrt{2}} \end{array}\right] 
                                                = \left[\begin{array}{ll} \frac{1}{2} & -\frac{1}{2} \\ -\frac{1}{2} & \frac{1}{2} \end{array}\right]\\
            P_{0}P_{1}                      & = \left[\begin{array}{ll} \frac{1}{2} & \frac{1}{2} \\ \frac{1}{2} & \frac{1}{2} \end{array}\right]
                                                \left[\begin{array}{ll} \frac{1}{2} & -\frac{1}{2} \\ -\frac{1}{2} & \frac{1}{2} \end{array}\right]\\
                                            & = \left[\begin{array}{ll} 0 & 0 \\ 0 & 0 \end{array}\right]
        \end{align*}
    \end{enumerate}
    
    \textbf{Q.} Explain the resulting product matrices. \textbf{A.} The product of the projectors of the standard basis is the identity matrix and the product of the projectors of the Hadamard basis is the zero matrix.

\end{enumerate}
\end{document}