\documentclass[main.tex]{subfiles}
\begin{document}

\subsection{Recall from prior lecture}

    $$
    \begin{aligned}
    A & =\left[\begin{array}{ll}
    a & b \\
    c & d
    \end{array}\right] \\
    A^{-1} & =\frac{1}{a d-c b}\left[\begin{array}{cc}
    d & -b \\
    -c & a
    \end{array}\right]
    \end{aligned}
    $$

    For singular matrices $A \Rightarrow A^{-1}$ exits. For matrix equation $A \bm{x}=\bm{b}$, if $A^{-1}$ exist, then $\bm{x}=A^{-} \bm{b}$.

\subsection{Properties of Inverse}

    $$
    (A B)^{-1}=B^{-1} \times A^{-1}
    $$

    Row operations for finding $A^{-1}$
    \begin{enumerate}
        \item multiply
        \item add
        \item interchange
    \end{enumerate}

    Can we prove that row operation for augmented matrix give the inverse $A$. Suppose matrix $E[A \mid I]$ performs a row operation $E=\left[\begin{array}{ll}0 & 1 \\ 1 & 0\end{array}\right]$ interchanges two rows.

    $$
    \left[\begin{array}{ll}
    0 & 1 \\
    1 & 0
    \end{array}\right]\left[\begin{array}{ll|ll}
    2 & 5 & 1 & 0 \\
    1 & 3 & 0 & 1
    \end{array}\right]=\left[\begin{array}{ll|ll}
    1 & 3 & 0 & 1 \\
    2 & 5 & 1 & 0
    \end{array}\right]
    $$

    Note, we show for every elementary row operation, there in an elementary matrix which performs some row operation by multiplying that matrix.

    $$
    E_{3}\left(E_{3}\left(E_{1}[A \mid I]\right)\right) = {\left[E_{3} E_{2} E_{1} A \mid E_{3} E_{2} E_{1} I\right]}
    $$

    How do we get the elementary matrix Which does contain element row operations on an identity matrix.

    $$
    E_{1}=\left[\begin{array}{ll}
    1 & 0 \\
    0 & 1
    \end{array}\right] \stackrel{r_{1} \longleftrightarrow r_{2}}{\longrightarrow}\left[\begin{array}{ll}
    0 & 1 \\
    1 & 0
    \end{array}\right]
    $$

    Using (Gauss - Jordan) elementary row operations.

    $$
    \begin{aligned}
    &\left[\begin{array}{ccc|ccc}
    2 & -1 & 0 & 1 & 0 & 0 \\
    -1 & 2 & -1 & 0 & 1 & 0 \\
    0 & -1 & 2 & 0 & 0 & 1
    \end{array}\right]\\
    &R_{2}+R_{1} \rightarrow R_{1}\\
    &\left[\begin{array}{ccc|ccc}
    1 & 1 & -1 & 1 & 1 & 0 \\
    -1 & 2 & -1 & 0 & 1 & 0 \\
    0 & -1 & 2 & 0 & 0 & 1
    \end{array}\right]\\
    &R_{1}+R_{2} \rightarrow R_{2}\\
    &\left[\begin{array}{ccc|ccc}
    1 & 1 & -1 & 1 & 1 & 0 \\
    0 & 3 & -2 & 1 & 2 & 0 \\
    0 & -1 & 2 & 0 & 0 & 1
    \end{array}\right]\\
    &2 R_{3}+R_{2} \rightarrow R_{2}\\
    &\left[\begin{array}{ccc|ccc}
    1 & 1 & -1 & 1 & 1 & 0 \\
    0 & 1 & 2 & 1 & 2 & 2 \\
    0 & -1 & 2 & 0 & 0 & 1
    \end{array}\right]\\
    &-R_{2}+R_{1}\\
    &R_{2}+R_{3}\\
    &\left[\begin{array}{lll|lll}
    1 & 0 & -3 & 0 & -1 & -2 \\
    0 & 1 & 2 & 1 & 2 & 2 \\
    0 & 0 & 4 & 1 & 2 & 3
    \end{array}\right]\\
    &\frac{1}{4} R_{3} \rightarrow R_{3}\\
    &\left[\begin{array}{ccc|ccc}
    1 & 0 & -3 & 0 & -1 & -2 \\
    0 & 1 & 2 & 1 & 2 & 2 \\
    0 & 0 & 1 & 1 / 4 & 3 / 4 & 3 / 4
    \end{array}\right]\\
    &-2R_{3}+R_{2}\\
    &3R_{3}+R_{1}\\
    &\left[\begin{array}{lll|lll}
    1 & 0 & 0 & 3 / 4 & 2 / 4 & 1 / 4 \\
    0 & 1 & 0 & 2 / 4 & 1 & 2 / 4 \\
    0 & 0 & 1 & 1 / 4 & 2 / 4 & 3 / 4
    \end{array}\right]\\
    A^{-1}&=\left[\begin{array}{ccc}
    .75 & .5 & .25 \\
    .5 & 1 & .5 \\
    .25 & .5 & .75
    \end{array}\right]
    \end{aligned}
    $$

\subsection{Factorization}

    Upper triangular matrix has all zero elements bellow the diagonal matrix. The diagonal matrix has all zeros above and below the diagonal. The lower triangular matrix has all zeros above the diagonal. To factoring matrix $A$, write $A$ as products of two or more matrices. $A=L \cdot L$ is a product of tho matrices, an upper triangular matrix $L$ and a lower triangular matrix $U$.

    $$
    A=\left[\begin{array}{ll}
    2 & 1 \\
    6 & 8
    \end{array}\right]
    

    Convert $A$ into an upper triangular matrix, using element row operations.

    $$
    \left[\begin{array}{ll}
    2 & 1 \\
    6 & 8
    \end{array}\right]-3R_{1}+R_{2} \rightarrow\left[\begin{array}{ll}
    2 & 1 \\
    0 & 5
    \end{array}\right]
    $$

    Find the element matrix that performs the above row operation.

    $$
    \left[\begin{array}{cc}
    1 & 0 \\
    -3 & 1
    \end{array}\right]\left[\begin{array}{ll}
    2 & 1 \\
    6 & 8
    \end{array}\right]=\left[\begin{array}{ll}
    2 & 1 \\
    0 & 5
    \end{array}\right]
    $$

    Note the element matrix is a lower triangular matrix. Also note that the inverse of a lower triangular matrix is also a lower triangular matrix.

    $$
    \begin{aligned}
    &E_{1}^{-1}=\frac{1}{a d-b c}\\
    &=\frac{1}{(1)(1)-(0)(3)}=\frac{1}{1}\left[\begin{array}{cc}
    d & -b \\
    -c & a
    \end{array}\right]\\
    &=\left[\begin{array}{ll}
    1 & 0 \\
    3 & 1
    \end{array}\right]
    \end{aligned}
    $$

    $$
    \begin{aligned}
    E_{1} \cdot A & = u\\
    A & = E^{-1} \cdot U\\
    A & = L \cdot U
    \end{aligned}
    $$

    What if we need to do many row operations

    $$
    \begin{aligned}
    \left(\varepsilon_{n} \ldots \varepsilon_{2} \varepsilon_{1}\right) A &= U \\
    A &=\left(\varepsilon_{n} \ldots \varepsilon_{2} \varepsilon_{1}\right)^{-1} \cdot U \\
    & = =E_{1}^{-1} E_{2}^{-1} \ldots E_{n} \cdot U 
    \end{aligned}
    $$

    Note: $E_{1}^{-1}, E_{2}^{-1}, \ldots E_{n}^{-1}$ are all linear triangular matrices. If you need to interchange rows to set $u$ then we do a permutation

    $$
    PA =L \times U\\
    $$

    Factorization

    $$
    \begin{aligned}
    A &= P^{-1} \times L \times U\\
    A & =\left[\begin{array}{ll}
    1 & 0 \\
    3 & 1
    \end{array}\right]\left[\begin{array}{ll}
    2 & 1 \\
    0 & 5
    \end{array}\right]\\
    A &= \left[\begin{array}{ll}
    1 & 0 \\
    3 & 1
    \end{array}\right]\left[\begin{array}{ll}
    2 & 0 \\
    0 & 5
    \end{array}\right]\left[\begin{array}{ll}
    1 & 1/2 \\
    0 & 1
    \end{array}\right]\\
    & =L \times D \times u
    \end{aligned}
   
    $$
    
\subsection{Solution of $Ax=0$}

    $$
    \begin{aligned}
        A^{-1}(A \bm{x}) & =A^{-1}(0)\\
        \bm{x} & = A^{-1} \cdot \bm{0}\\
    \end{aligned}
    $$

    Suppose $A \bm{x}=\bm{0}$ has non-zero solution, then $A^{-1}$ does not exist.
    
    
\subsection{Transpose of a Matrix}

    Transpose of a matrix means interchange of rows and columns.

    $$
    \begin{aligned}
    A & =\left[a_{1}\left|a_{3}\right| \ldots \mid a_{n}\right]\\
    A^{T} & = \left[\frac{\frac{a_{1}}{a_{2}}}{\vdots}\right]\\
    A &= \left[\begin{array}{lll}
    1 & 2 & 3 \\
    4 & 5 & 6 \\
    7 & 8 & 9
    \end{array}\right]\\
    A^{T} &= \left[\begin{array}{l|l|l}
    1 & 4 & 7 \\
    2 & 5 & 8 \\
    3 & 6 & 9
    \end{array}\right]
    \end{aligned}
    $$

    Note if $A^{T} = A \Rightarrow A$ is symmetric. Matrix $A$ is symmetric when $a_{ij}=a_{ji} \forall i, j$. Suppose

    $$
    \begin{aligned}
    A & = R^{T} \cdot R\\
    A^{T} & = \left(R^{T} \cdot R\right)^{T}\\
    & = R^{T} \cdot \left(R^{T}\right)^{T} \\
    & =R^{T} \cdot R
    \end{aligned}
    $$

\subsection{Vector Space}

    Vector space: $\mathbb{R}^{1}, \mathbb{R}^{2}, \mathbb{R}^{3} \mathbb{R}^{4}, \dots, \mathbb{R}^{n}$. Subspoce: any collection of a space With infinitely many vectors and three properties.

    \begin{enumerate}
        \item [1.] $\bm{0}$ should be in the collection such that $\bm{0}+\bm{v}=\bm{v}$ Note: vector addition, subtraction, scalar multiplication are defined within the space.
        \item [2.] If you take any two vectors $\bm{v}_1, \bm{v}_2$, then $\bm{v}_1 + \bm{v}_2$ should be in the collection $c \cdot $\bm{v}$.
        \item [3.] For any vector $\bm{v}$ in the collection, $c_{0}\bm{v}$ there should be in the collection. Note $c$ is a constant where $c \in \mathbb{R}$.
    \end{enumerate}

    Note properties 2 & 3 can be combined. For any two vectors $\bm{v}_1$, and $\bm{v}_2$ in the collection, the $c_{1} \bm{v}_1 + c_{2} \bm{v}_2$ should be in the collection. Where $c_{1}, c_{2} \in \mathbb{R}$ are constants. Collection subspace properties

\subsection{Function Space}

    All polynomials of  function:
    
    $$
    p(x) = a x^{3}+ b x^{2}+ d x \forall a, b, d \in \mathbb{R}
    $$
        
    in this collection have the following subspace properties 
    \begin{enumerate}
        \item [1.]
        $$
        \begin{aligned}
        $\bm{0} & = 0 x^{3} +0 x^{2} +0 x\\
        $\bm{0} + \left(a x^{3}+b x^{2} + d x\right) & = a x^{3}+b x^{3}+d x$
        $\bm{0} \bm{v} = \bm{v}$
        \end{aligned}
        $$
        \item [2.]        
        $$
        \begin{aligned}
        \left.\left(a x^{3}+b x^{2}+ d x\right)+ c x^{3}+ f x^{2}+ g\right) 
        & =(a+e) x^{3}+(b+f) x^{2}+(d+g)x \\
        c \cdot\left(a x^{3}+ b x^{2}+ d\right) & = c a x^{3} + c b x^{2} +c d\\
        c \cdot \bm{v} & = cv
        \end{aligned}
        $$
    \end{enumerate}

    This collection is a subspace in $\mathbb{R}$

\subsection{Column space of a matrix} 

    $$
    \left[\begin{array}{ll}
    2 & 4 \\
    0 & 6
    \end{array}\right]\left[\begin{array}{l}
    x_{1} \\
    x_{2}
    \end{array}\right]=x_{1}\left[\begin{array}{l}
    2 \\
    0
    \end{array}\right]+x_{2}\left[\begin{array}{l}
    4 \\
    6
    \end{array}\right]
    $$
    
    $$
    \left[\bm{a}_{1} \mid \bm{a}_{2} \mid \ldots \mid \bm{a}_{n}\right]
    \left[\begin{array}{c}
    c_{1} \\
    c_{2} \\
    \vdots \\
    c_{n}
    \end{array}\right] = c_{1} \bm{a}_{1} + c_{2}\bm{a}_2 + \ldots + c_{n} \bm{a}_n
    $$
    
    all linear combination of columns of matrix $A$ are called the column space of $A$. Example shown bellow
    
    $$
    c(A)=\left[\begin{array}{ll}
    1 & 0 \\
    0 & 4
    \end{array}\right]\left[\begin{array}{l}
    c_{1} \\
    c_{2}
    \end{array}\right]=c_{1}\left(\begin{array}{l}
    1 \\
    0
    \end{array}\right)+c_{2}\left(\begin{array}{l}
    0 \\
    4
    \end{array}\right)
    $$
    
    Another example
    
    $$
    \begin{aligned}
    A & = \left[\begin{array}{lll}
    1 & 2 & 4 \\
    0 & 0 & 5
    \end{array}\right]\\
    C(A) & = x_{1}\left[\begin{array}{l}
    1 \\
    0
    \end{array}\right]+x_{2}\left[\begin{array}{l}
    2 \\
    0
    \end{array}\right]+x_{3}\left[\begin{array}{l}
    1 \\
    5
    \end{array}\right]\\
    & =\left(x_{1}+2 y_{2}\right)\left[\begin{array}{l}
    1 \\
    0
    \end{array}\right]+x_{3}\left[\begin{array}{l}
    4 \\
    5
    \end{array}\right]
    \end{aligned}
    $$
    
    $C(A)=$ all linear combinations of
    
    $$
    \left[\begin{array}{l}
    1 \\
    0
    \end{array}\right] \text { and }\left[\begin{array}{l}
    4 \\
    5
    \end{array}\right]
    $$
    
    Column space is in the output space so it is in $\mathbb{R}^{n}$

\subsection{Null space of a matrix} 

    Null space of $A$ is a collection of all vectors $\bm{x}$ such that $A \bm{x}=0$
    
    $$
    \begin{aligned}
    \left[\begin{array}{ll}
    1 & 2 \\
    3 & 6
    \end{array}\right]\left[\begin{array}{l}
    x_{1} \\
    x_{2}
    \end{array}\right] & =\left[\begin{array}{l}
    c \\
    c
    \end{array}\right]\\
    \left[\begin{array}{ll|l}
    1 & 2 & 0 \\
    3 & 6 & 0
    \end{array}\right] & \\
    -3R_{1}+R_{2} & \rightarrow R_2\\
    \left[\begin{array}{ll|l}
    1 & 2 & 0 \\
    0 & 0 & 0
    \end{array}\right] & \\
    x_1 + 2 x_2 & =0 \\
    x_1 & = -2 x_2 \\
    x_{2}\left[\begin{array}{c}
    -2 \\
    1
    \end{array}\right], \forall X_{3} \in \mathbb{R} &
    \end{aligned}
    $$

\end{document}