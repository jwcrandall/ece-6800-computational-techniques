\documentclass[main.tex]{subfiles}
\begin{document}

\subsection{Column Space}

Column space of $A$ or $\operatorname{col}(A)$ is the collection of all vectors that are linear combinations of columns of matrix $A$.

$$
\begin{aligned}
A_{m \times n} &= \left[\bm{a_{1}} \mid \bm{a}_{2} \mid \ldots \mid \bm{a}_{n}\right]\\
A_{m \times n} \bm{x}_{n \times 1} &= \bm{b}_{m \times 1}\\
\left[\bm{a}_1 \mid \bm{a}_2 \mid \ldots \mid \bm{a}_n \right]
\left[\begin{array}{c}
x_{1} \\
x_{2} \\
\vdots \\
x_{n}
\end{array}\right]&=\left[\begin{array}{c}
b_{1} \\
b_{2} \\
\vdots \\
b_{m}
\end{array}\right]
\end{aligned}
$$

Multiply each column by element resulting in a linear combination of columns or vectors. $b$ is in column space of $A$. $\operatorname{col}(A)$ includes all such vectors included in a linear combination of columns of matrix $A$.

$$
\begin{aligned}
A &= \left[\begin{array}{lll}
3 & 6 \\
3 & 4
\end{array}\right]\\
c(A) &= x_{1}\left[\begin{array}{l}
3 \\
3
\end{array}\right]+x_{2}\left[\begin{array}{l}
6 \\
4
\end{array}\right]\\
& = x_{1}\left[\begin{array}{l}
3 \\
3
\end{array}\right]+2 x_{2}\left[\begin{array}{l}
3 \\
2
\end{array}\right]\\
&=c\left[\begin{array}{l}
3 \\
2
\end{array}\right] \forall c \in \mathbb{R}
\end{aligned}
$$

$c(A)$ is space of all possible vectors out of system $A$.

$$
\begin{aligned}
A &= \left[\begin{array}{lll}
1 & 2 & 1 \\
0 & 0 & 5
\end{array}\right]_{2 \times 3} \\
{\left[\begin{array}{lll}
1 & 2 & 4 \\
0 & 0 & 5
\end{array}\right]\left[\begin{array}{l}
x_{1} \\
x_{3} \\
x_{3}
\end{array}\right] &= \left[\begin{array}{l}
b_{1} \\
b_{3}
\end{array}\right]_{3 \times 1}}\\
\bm{b} &= x_{1}\left[\begin{array}{l}
1 \\
0
\end{array}\right]+x_{2}\left[\begin{array}{l}
2 \\
0
\end{array}\right]+x_{3}\left[\begin{array}{l}
4 \\
5
\end{array}\right] \\
&= \left(x_{1}+2 x_{2}\right)\left[\begin{array}{l}
1 \\
0
\end{array}\right]+x_{3}\left[\begin{array}{l}
4 \\
5
\end{array}\right]
\end{aligned}
$$

$c(A)$ is in the plane $\mathbb{R}^{2}$

\subsection{Linearize Nonlinear Systems}

We can also linearize systems about an operating point. This is accomplished when nonlinear $\mathbb{R}^{\prime}$ space is linearized around a specific point.

$$
\begin{aligned}
y-y_{1} &= m\left(x-x_{1}\right) \\
y-2.71 &= 2 \cdot \pi(x-1) \\
y &= 2.71 x-2.71+2.71 \\
y &= 2.71 x
\end{aligned}
$$

linear system operating at $x=1$. Most real world Systems are non-linear, but most get linearized when you Work with them.

\subsection{Null space}

Null space of $A$ is collection of all vectors (in the input space) $\bm{x}$ such that $A \bm{x}=\bm{0}$. $N(A)$ is collection of all input vectors that make the system $\bm{0}$ in the output space.

$$
\begin{aligned}
A&=\left[\begin{array}{ll}
1 & 2 \\
3 & 6
\end{array}\right]\\
\left[\begin{array}{ll}
1 & 2 \\
3 & 6
\end{array}\right]\left[\begin{array}{l}
x_{1} \\
x_{2}
\end{array}\right]&=\left[\begin{array}{l}
0 \\
0
\end{array}\right]\\
&\left[\begin{array}{ll|l}1 & 2 & 0 \\ 3 & 6 & 0\end{array}\right]\\
-3 R_{1}+R_{2} &\rightarrow R_{2} \\
&\left[\begin{array}{ll|l}
1 & 2 & 0 \\
0 & 0 & 0
\end{array}\right]\\
x_{1}+2 x_{2} &=0 \\
x_{1} &= -2 x_{2} \\
\bm{x}&=\left[\begin{array}{c}
-2 \\
1
\end{array}\right] x_{2} \quad \forall x_{2} \in \mathbb{R}
\end{aligned}
$$

where $x_2$ is a free variable. All non-pivot column elements are free variables.

$$
\begin{aligned}
A &= \left[\begin{array}{llll}
1 & 2 & 2 & 4 \\
3 & 8 & 6 & 16
\end{array}\right]\\
{\left[\begin{array}{llll}
1 & 2 & 2 & 4 \\
3 & 8 & 6 & 16
\end{array}\right]\left[\begin{array}{l}
x_{1} \\
x_{2} \\
x_{3} \\
x_{4}
\end{array}\right] &= \left[\begin{array}{l}
0 \\
0
\end{array}\right]}\\
&{\left[\begin{array}{llllll}
1 & 2 & 2 & 1 \\
3 & 8 & 6 & 1 & 1 & 8
\end{array}\right]} \\
-3 R_{1}+R_{2} &\rightarrow R_{2} \\
&{\left[\begin{array}{llll|l}
1 & 2 & 2 & 4 & 0 \\
0 & 2 & 0 & 4 & 0
\end{array}\right]}\\
\frac{1}{2} R_{2} & \rightarrow R_{2} \\
&\left[\begin{array}{llll|l}
1 & 2 & 2 & 4 & 0 \\
0 & 1 & 0 & 2 & 0
\end{array}\right]\\
x_1 + 2x_2 + 2x_3 + 4x_4 &= 0\\
x_2 + 2x_4 &= 0 \\
x_2 &= -2 x_4 \\
x_{1}-4 x_{4}+2 x_{3}+4 x_{4} &= 0 \\
x_{1} &= -2 x_{3}\\
N(A) &= \left[\begin{array}{l}
x_{1} \\
x_{2} \\
x_{3} \\
x_{4}
\end{array}\right] = \left[\begin{array}{c}
-2x_3 \\
-2x_4 \\
x_3 \\
x_4
\end{array}\right]\\
&= x_{3}\left[\begin{array}{c}
-2 \\
0 \\
1 \\
0
\end{array}\right]+x_{4}\left[\begin{array}{c}
0 \\
-2 \\
0 \\
1
\end{array}\right] \quad \forall x_3, x_4 \in \mathbb{R}.
\end{aligned}
$$

For a matrix to be in echelon form every column can have only one pivot element at the most. All elements below pivot elements must be zero below non-zero values. In reduced echelon form all elements above and below pivot elements must be zero.

$$
\begin{aligned}
\left[\begin{array}{llll|l}
1 & 2 & 2 & 4 & 0 \\
0 & 1 & 0 & 2 & 0
\end{array}\right]&\\
-2R_{2}+R_{1} & \rightarrow R_{1}\\
\left[\begin{array}{llll|l}
1 & 0 & 2 & 0 & 0 \\
0 & 1 & 0 & 2 & 0
\end{array}\right]&\\
x_{1} &=-2 x_{3} \\
x_{2} &=-2 x_{4}
\end{aligned}
$$

$x_{3}$, $x_4$ are free variables

\subsection{Rank of Matrix A}

$rank(A)=r$ is the number of pivots. Where $A_{m\times n}$ has full row rank with $m$ pivots, and full column rank with $n$ pivots.  

$$
A=\left[\begin{array}{llll|l}
1 & 2 & 7 & 1 & 0 \\
0 & 1 & 0 & 2 & 0
\end{array}\right]
$$

Above example has rank 2, with full row rank where every row has a pivot column.

$A x=b$ solution is called a particular solution $x_p$. $A \bm{x}=\bm{0}$ solution is called a special solution $x_h$.

$$
\begin{aligned}
A \bm{x}_p &=\bm{b} \\
A \bm{x}_h &=\bm{0}
\end{aligned}
$$

Solve for particular solution

$$
\left[\begin{array}{llll}
1 & 3 & 0 & 2 \\
0 & 0 & 1 & 4 \\
1 & 3 & 1 & 6
\end{array}\right]\left[\begin{array}{l}
x_{1} \\
x_{2} \\
x_{3} \\
x_{4}
\end{array}\right]=\left[\begin{array}{l}
1 \\
6 \\
7
\end{array}\right]
$$

Complete solution is $\left[x_{p}+x_{h}\right]$

$$
\begin{aligned}
&{\left[\begin{array}{llll|l}
1 & 3 & 0 & 0 & 0 \\
0 & 0 & 1 & 4 & 0 \\
1 & 3 & 1 & 6 & 0
\end{array}\right]} \\
&\cdots\\
&{\left[\begin{array}{llll|l}
1 & 3 & 0 & 2 & 0 \\
0 & 0 & 1 & 4 & 0 \\
0 & 0 & 0 & 0 & 0
\end{array}\right]}\\
x_{1} &= 3x_{2} - 2x_{4} \\
x_{3} &= -4x_{4} \\
\bm{x_h} & = x_{2}\left[\begin{array}{c}
-3 \\
1 \\
0 \\
0
\end{array}\right] + x_{4}\left[\begin{array}{c}
-2 \\
0 \\
-4 \\
1
\end{array}\right]
\end{aligned}
$$

Solution for $x_p$ using row operation.

$$
\begin{aligned}
\left[\begin{array}{llll}
1 & 3 & 0 & 2 \\
0 & 0 & 1 & 4 \\
1 & 3 & 1 & 6
\end{array}\right]\left[\begin{array}{l}
x_{1} \\
x_{2} \\
x_{3} \\
x_{4}
\end{array}\right]&=\left[\begin{array}{l}
1 \\
6 \\
7
\end{array}\right]\\
x_{1}\left[\begin{array}{l}
1 \\
0 \\
1
\end{array}\right]+x_{2}\left[\begin{array}{l}
3 \\
0 \\
3
\end{array}\right]+x_{3}\left[\begin{array}{l}
0 \\
1 \\
1
\end{array}\right]+x_{4}\left[\begin{array}{l}
2 \\
4 \\
6
\end{array}\right] & = \left[\begin{array}{l}
1 \\
6 \\
7
\end{array}\right]\\
1\left[\begin{array}{l}
1 \\
0 \\
1
\end{array}\right]+0\left[\begin{array}{l}
3 \\
6 \\
3
\end{array}\right]+6\left[\begin{array}{l}
0 \\
1 \\
1
\end{array}\right]+0\left[\begin{array}{l}
2 \\
4 \\
6
\end{array}\right]&=\left[\begin{array}{l}
1 \\
6 \\
7
\end{array}\right] \\
\left[\begin{array}{l}
1 \\
0 \\
1
\end{array}\right]+\left[\begin{array}{l}
0 \\
0 \\
0
\end{array}\right]+\left[\begin{array}{l}
0 \\
6 \\
6
\end{array}\right]+\left[\begin{array}{l}
0 \\
0 \\
0
\end{array}\right]&=\left[\begin{array}{l}
1 \\
6 \\
7
\end{array}\right]\\
x_{p}\left[\begin{array}{l}
x_{1} \\
x_{2} \\
x_{3} \\
x_{4}
\end{array}\right] &= \left[\begin{array}{l}
1 \\
0 \\
6 \\
0
\end{array}\right]\\
x_{p}+x_{h} & =\left[\begin{array}{l}
1 \\
0 \\
6 \\
0
\end{array}\right] + x_{2}\left[\begin{array}{c}
-3 \\
1 \\
0
\end{array}\right] + x_{4}\left[\begin{array}{c}
-2 \\
0 \\
-4 \\
1
\end{array}\right]
\end{aligned}
$$

In the complete general solution $x_h + x_p$, $x_p$ shifts the hyperplane, and $x_h$ is the hyperplane.

$$
\begin{aligned}
A &= \left[\begin{array}{cc}
1 & 1 \\
1 & 3 \\
-2 & -3
\end{array}\right]\\
[A]\left[\begin{array}{l}
x_{1} \\
x_{3}
\end{array}\right]&=\left[\begin{array}{l}
b_{1} \\
b_{2} \\
b_{3}
\end{array}\right]\\
-R_1 + R_2 &\rightarrow R_2 \\
2R_1 + R_3 &\rightarrow R_3 \\
&\left[\begin{array}{cc|c}
1 & 1 & b_{1} \\
0 & 1 & b_{2}-b_{1} \\
0 & -1 & b_{3}+2 b_{1}
\end{array}\right]\\
-R_2 + R_1 &\rightarrow R_1\\
R_2+R_3 \rightarrow R_3\\
&\left[\begin{array}{ll|l}
1 & 0 & 2 b_{1}-b_{2} \\
0 & 1 & b_{2}-b_{1} \\
0 & 0 & b_{3}+b_{2}+b_{1}
\end{array}\right]
\end{aligned}
$$

For this system to have any solution $b_{3}+b_{2}+b_{1}=0$. For the particular solution 

$$
\begin{aligned}
x_{1} &= 2 b_{1}-b_{3} \\
x_{2} &= b_{2}-b_{1} \\
X_{p}&=\left[\begin{array}{c}
x_{1} \\
x_{2}
\end{array}\right]=\left[\begin{array}{c}
2 b_{1}-b_{2} \\
b_{2}-b_{1}
\end{array}\right]
\end{aligned}
$$

Particular solution

$$
\begin{aligned}
A \bm{x}_{p}&=\bm{b}\\
\left[\begin{array}{cc}
1 & 1 \\
1 & 2 \\
-2 & -3
\end{array}\right]\left[\begin{array}{l}
x_{1} \\
x_{1}
\end{array}\right]&=\left[\begin{array}{l}
b_{1} \\
b_{3} \\
b_{0}
\end{array}\right]\\
\left[\begin{array}{cc}
1 & 1 \\
1 & 2 \\
-2 & -3
\end{array}\right]\left[\begin{array}{c}
2 b_{1}-b_{2} \\
b_{2}-b_{1}
\end{array}\right]&=\left[\begin{array}{l}
b_{1} \\
b_{2} \\
b_{3}
\end{array}\right]\\
\left[\begin{array}{c}
\left(2 b_{1}-b_{2}\right)+\left(b_{2}-b_{1}\right) \\
\left(2 b_{1}-b_{2}\right)+2\left(b_{2}-b_{1}\right) \\
-2\left(b_{2}-b_{1}\right)-3\left(b_{2}-b_{1}\right)
\end{array}\right]&=\left[\begin{array}{l}
b_{1} \\
b_{2} \\
b_{3}
\end{array}\right]
\end{aligned}
$$

\subsection{Basis linearly independent}

A set of vectors $\bm{v}_{1}, \bm{v}_{2}, \cdots, \bm{v}_n$ are linearly independent if $c_{1} \bm{v}_{1} + c_{2} \bm{v}_{2} + \ldots + c_{n} \bm{v}_{n}=0$ and if $c_{1}=c_{2}=\ldots=c_{n}$.

$$
\left(\begin{array}{l}
0 \\
0 \\
1
\end{array}\right),\left(\begin{array}{l}
0 \\
1 \\
0
\end{array}\right),\left(\begin{array}{l}
1 \\
0 \\
0 
\end{array}\right)
$$

True set of vectors is linearly independent.

$$
c_{1}\left[\begin{array}{l}
1 \\
0 \\
0
\end{array}\right]+c_{2}\left[\begin{array}{l}
0 \\
1 \\
0
\end{array}\right]+c_{3}\left[\begin{array}{l}
0 \\
0 \\
1
\end{array}\right]=\left[\begin{array}{l}
0 \\
0 \\
0
\end{array}\right]
$$

Suppose 

$$
\begin{aligned}
c_{1} \bm{v}_{1} + c_{2} \bm{v}_2 + c_{3} \bm{v}_{3} &= 0\\
\bm{v}_1 &= -\frac{c_2}{c_1} \bm{v}_2 + \frac{c_{3}}{c_{1}} \bm{v}_{2}
\end{aligned}
$$

Linearly independent vectors can not be written in terms of each other. Suppose we hove three vectors in $v_1, v_2, v_3$ in $\mathbb{R}^{3}$. Note $\bar{v}_{3}$ is not in the plane going through $\bm{v}_1 \& \bm{v}_2$.

$$
\bm{v}_{3}= c_{1} \bm{v}_{1} + c_{2} \bm{v}_{2}, \quad c_{1}, c_{2} \neq 0
$$

\end{document}
