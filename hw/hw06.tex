\documentclass[main.tex]{subfiles}
\begin{document}

\subsection{Section 6.4}
\begin{enumerate}
    \item [3.] \textbf{Q.} Find the eigenvalues and the unit eigenvectors of
    $$
    A=\left[\begin{array}{lll}
    2 & 2 & 2 \\
    2 & 0 & 0 \\
    2 & 0 & 0
    \end{array}\right]
    $$
    
    \textbf{A.}

    $$
    \begin{aligned}
    \operatorname{det}(A-\lambda I) &=0 \\
    \left|\begin{array}{ccc}
    2-\lambda & 2 & 2 \\
    2 & -\lambda & 0 \\
    2 & 0 & -\lambda
    \end{array}\right| &=0 \\
    (2-\lambda)(-\lambda)(-\lambda)-2(-2 \lambda)+2(-2(-\lambda)) &=0 \\
    -\lambda^{3}+2 \lambda^{2}+8 \lambda &=0 \\
    \lambda(\lambda-4)(\lambda+2)&=0 \\
    \lambda &= 0\\
    (A-\lambda I) x &= 0\\
    \left(\begin{array}{lll}
    2 & 2 & 2 \\
    2 & 0 & 0 \\
    2 & 0 & 0
    \end{array}\right)\left(\begin{array}{l}
    x \\
    y \\
    z
    \end{array}\right) &=\left(\begin{array}{l}
    0 \\
    0 \\
    0
    \end{array}\right) \\
    \left(\begin{array}{l}
    x \\
    y \\
    z
    \end{array}\right) &=\left(\begin{array}{l}
    0 \\
    1 \\
    -1
    \end{array}\right)\\
    & \pm \frac{1}{\sqrt{2}}\left(\begin{array}{l}
    0 \\
    1 \\
    -1
    \end{array}\right)\\
    \lambda &= 4\\
    (A-\lambda I) x &= 0\\
    \left(\begin{array}{ccc}
    -2 & 2 & 2 \\
    2 & -4 & 0 \\
    2 & 0 & -4
    \end{array}\right)\left(\begin{array}{l}
    x \\
    y \\
    z
    \end{array}\right) &=\left(\begin{array}{l}
    0 \\
    0 \\
    0
    \end{array}\right) \\
    \left(\begin{array}{l}
    x \\
    y \\
    z
    \end{array}\right) &=\left(\begin{array}{l}
    2 \\
    1 \\
    1
    \end{array}\right)\\
    & \pm \frac{1}{\sqrt{6}}\left(\begin{array}{l}
    2 \\
    1 \\
    1
    \end{array}\right)\\
    \end{aligned}
    $$

    Final eigen value and unit eigen vector

    $$
    \begin{aligned}        
    \lambda &= -2\\
    (A-\lambda I) x &= 0\\
    \left(\begin{array}{lll}
    4 & 2 & 2 \\
    2 & 2 & 0 \\
    2 & 0 & 2
    \end{array}\right)\left(\begin{array}{l}
    x \\
    y \\
    z
    \end{array}\right)&=\left(\begin{array}{l}
    0 \\
    0 \\
    0
    \end{array}\right) \\
    \left(\begin{array}{l}
    x \\
    y \\
    z
    \end{array}\right)&=\left(\begin{array}{l}
    1 \\
    -1 \\
    -1
    \end{array}\right)\\
    & \pm \frac{1}{\sqrt{3}}\left(\begin{array}{l}
    1 \\
    -1 \\
    -1
    \end{array}\right)
    \end{aligned}
    $$
        
    \item [5.] \textbf{Q.} Find an orthogonal matrix $Q$ that diagonalizes this symmetric matrix:
    $$
    A=\left[\begin{array}{rrr}
    1 & 0 & 2 \\
    0 & -1 & -2 \\
    2 & -2 & 0
    \end{array}\right]
    $$
    
    \textbf{A.}
    
    $$
    \begin{aligned}
    \operatorname{det}(A-\lambda I) &=0 \\
    \left|\begin{array}{ccc}
    1-\lambda & 0 & 2 \\
    0 & -1-\lambda & -2 \\
    2 & -2 & -\lambda
    \end{array}\right| & = 0 \\
    (1-\lambda)((-1-\lambda)(-\lambda)-4)+2(-2(-1-\lambda)) &=0 \\
    -\lambda^{3}+9 \lambda &=0\\
    \lambda(\lambda-3)(\lambda+3) &=0\\
    \lambda &= 0\\
    (A-\lambda I) x&=0\\
    \left(\begin{array}{ccc}
    1 & 0 & 2 \\
    0 & -1 & -2 \\
    2 & -2 & 0
    \end{array}\right)\left(\begin{array}{l}
    x \\
    y \\
    z
    \end{array}\right) &=\left(\begin{array}{l}
    0 \\
    0 \\
    0
    \end{array}\right) \\
    \left(\begin{array}{l}
    x \\
    y \\
    z
    \end{array}\right) &=\left(\begin{array}{l}
    2 \\
    2 \\
    -1
    \end{array}\right)\\
    &\pm \frac{1}{3}\left(\begin{array}{l}
    2 \\
    2 \\
    -1
    \end{array}\right)\\
    \lambda &= 3\\
    (A-\lambda I) x&=0\\
    \left(\begin{array}{ccc}
    -2 & 0 & 2 \\
    0 & -4 & -2 \\
    2 & -2 & -3
    \end{array}\right)\left(\begin{array}{l}
    x \\
    y \\
    z
    \end{array}\right) &=\left(\begin{array}{l}
    0 \\
    0 \\
    0
    \end{array}\right) \\
    \left(\begin{array}{l}
    x \\
    y \\
    z
    \end{array}\right) &=\left(\begin{array}{l}
    2 \\
    -1 \\
    2
    \end{array}\right)\\
    &\pm \frac{1}{3}\left(\begin{array}{l}
    2 \\
    -1 \\
    2
    \end{array}\right)\\
    \lambda &= -3\\
    (A-\lambda I) x&=0\\
    \left(\begin{array}{ccc}
    4 & 0 & 2 \\
    0 & 2 & -2 \\
    2 & -2 & 3
    \end{array}\right)\left(\begin{array}{l}
    x \\
    y \\
    z
    \end{array}\right) &=\left(\begin{array}{l}
    0 \\
    0 \\
    0
    \end{array}\right) \\
    \left(\begin{array}{l}
    x \\
    y \\
    z
    \end{array}\right) &=\left(\begin{array}{l}
    1 \\
    -2 \\
    -2
    \end{array}\right)\\
    &\pm \frac{1}{3}\left(\begin{array}{l}
    1 \\
    -2 \\
    -2
    \end{array}\right)\\
    Q&=\frac{1}{3}\left(\begin{array}{ccc}
    2 & 2 & 1 \\
    2 & -1 & -2 \\
    -1 & 2 & -2
    \end{array}\right)
    \end{aligned}
    $$
    
    \item [11.] \textbf{Q.} Write $A$ and $B$ in the form $\lambda_{1} x_{1} x_{1}^{\mathrm{T}}+\lambda_{2} x_{2} x_{2}^{\mathrm{T}}$ of the spectral theorem $Q \Lambda Q^{T}$ :
    $$
    A=\left[\begin{array}{ll}
    3 & 1 \\
    1 & 3
    \end{array}\right] \quad B=\left[\begin{array}{rr}
    9 & 12 \\
    12 & 16
    \end{array}\right] \quad\left(\text { keep }\left\|x_{1}\right\|=\left\|x_{2}\right\|=1\right)
    $$
    
    \textbf{A.}

    Matrix $A$
    
    $$
    \begin{aligned}
    \operatorname{det}(A-\lambda I) &=0 \\
    \left|\begin{array}{cc}
    3-\lambda & 1 \\
    1 & 3-\lambda
    \end{array}\right| &=0 \\
    (3-\lambda)(3-\lambda)-(1)(1) &=0 \\
    \lambda^{2}-6 \lambda+8 &=0\\
    (\lambda-2)(\lambda-4)&=0 \\
    \lambda &= 2\\
    (A-\lambda I) x &=0\\
    \left(\begin{array}{ll}
    1 & 1 \\
    1 & 1
    \end{array}\right)\left(\begin{array}{l}
    x_{1} \\
    y_{1}
    \end{array}\right) &=\left(\begin{array}{l}
    0 \\
    0
    \end{array}\right) \\
    \left(\begin{array}{l}
    x_{1} \\
    y_{1}
    \end{array}\right) &=\left(\begin{array}{l}
    1 \\
    -1
    \end{array}\right)\\
    & \pm \frac{1}{\sqrt{2}}\left(\begin{array}{l}
    1 \\
    -1
    \end{array}\right)\\
    \lambda &= 4\\
    (A-\lambda I) x &=0\\
    \left(\begin{array}{cc}
    -1 & 1 \\
    1 & -1
    \end{array}\right)\left(\begin{array}{l}
    x_{2} \\
    y_{2}
    \end{array}\right) &=\left(\begin{array}{l}
    0 \\
    0
    \end{array}\right) \\
    \left(\begin{array}{l}
    x_{2} \\
    y_{2}
    \end{array}\right) &=\left(\begin{array}{l}
    1 \\
    1
    \end{array}\right)\\
    & \pm \frac{1}{\sqrt{2}}\left(\begin{array}{l}
    1 \\
    1
    \end{array}\right)\\
    \end{aligned}
    $$

    Using spectral theorem

    $$
    \begin{aligned}
    A &=\lambda_{1} x_{1} x_{1}^{T}+\lambda_{2} x_{2} x_{2}^{T} \\
    &=2 \times \frac{1}{\sqrt{2}} \times \frac{1}{\sqrt{2}}\left(\begin{array}{l}
    1 \\
    -1
    \end{array}\right)\left(\begin{array}{ll}
    1 & -1
    \end{array}\right)+4 \times \frac{1}{\sqrt{2}} \times \frac{1}{\sqrt{2}}\left(\begin{array}{l}
    1 \\
    1
    \end{array}\right)\left(\begin{array}{ll}
    1 & 1
    \end{array}\right) \\
    &=2\left(\begin{array}{cc}
    \frac{1}{2} & -\frac{1}{2} \\
    -\frac{1}{2} & \frac{1}{2}
    \end{array}\right)+4\left(\begin{array}{ll}
    \frac{1}{2} & \frac{1}{2} \\
    \frac{1}{2} & \frac{1}{2}
    \end{array}\right)
    \end{aligned}
    $$

    Matrix $B$

    $$
    \begin{aligned}
    \operatorname{det}(B-\lambda I) &=0 \\
    \left|\begin{array}{cc}
    9-\lambda & 12 \\
    12 & 16-\lambda
    \end{array}\right| &=0 \\
    (9-\lambda)(16-\lambda)-(12)(12) &=0 \\
    \lambda^{2}-25 \lambda &=0 \\
    \lambda(\lambda-25)&=0\\
    \lambda &= 0\\
    (B-\lambda I) x &=0\\
    \left(\begin{array}{ll}
    9 & 12 \\
    12 & 16
    \end{array}\right)\left(\begin{array}{l}
    x \\
    y
    \end{array}\right) &=\left(\begin{array}{l}
    0 \\
    0
    \end{array}\right) \\
    \left(\begin{array}{l}
    x \\
    y
    \end{array}\right) &=\left(\begin{array}{l}
    4 \\
    -3
    \end{array}\right)\\
    & \frac{1}{5}\left(\begin{array}{c}
    4 \\
    -3
    \end{array}\right)\\
    \lambda &= 25\\
    (B-\lambda I) x&=0\\
    \left(\begin{array}{cc}
    -16 & 12 \\
    12 & -9
    \end{array}\right)\left(\begin{array}{l}
    x \\
    y
    \end{array}\right) &=\left(\begin{array}{l}
    0 \\
    0
    \end{array}\right) \\
    \left(\begin{array}{l}
    x \\
    y
    \end{array}\right) &=\left(\begin{array}{l}
    3 \\
    4
    \end{array}\right)\\
    &\frac{1}{5}\left(\begin{array}{l}
    3 \\
    4
    \end{array}\right)\\
    \end{aligned}
    $$

    Using spectral theorem 

    $$
    \begin{aligned}
    B &=\lambda_{1} x_{1} x_{1}^{T}+\lambda_{2} x_{2} x_{2}^{T} \\
    &=0 \times \frac{1}{5} \times 5\left(\begin{array}{c}
    4 \\
    -3
    \end{array}\right)\left(\begin{array}{ll}
    4 & -3
    \end{array}\right)+25 \times \frac{1}{5} \times \frac{1}{5}\left(\begin{array}{l}
    3 \\
    4
    \end{array}\right)\left(\begin{array}{ll}
    3 & 4
    \end{array}\right) \\
    &=0\left(\begin{array}{cc}
    0.64 & -0.48 \\
    -0.48 & 0.36
    \end{array}\right)+25\left(\begin{array}{cc}
    0.36 & 0.48 \\
    0.48 & 0.64
    \end{array}\right)
    \end{aligned}
    $$
    
\end{enumerate}

\subsection{Section 6.6}
\begin{enumerate}
    \item [1.] \textbf{Q.} If $C=F^{-1} A F$ and also $C=G^{-1} B G$, what matrix $M$ gives $B=M^{-1} A M$ ? Conclusion: If $C$ is similar to $A$ and also to $B$ then \rule{1cm}{0.15mm}.

    \textbf{A.} 
    $$
    \begin{aligned}
    B &=G C G^{-1}\\
    B &=G\left(F^{-1} A F\right) G^{-1} \\
    &=\left(F G^{-1}\right)^{-1} A F G^{-1}\\
    M&=F G^{-1}\\
    \end{aligned}
    $$

    If $C$ is similar to $A$ and also to $B$, then $A$ is similar to $B$.
    
    \item [7.]
    \begin{enumerate}
        \item [a.] \textbf{Q.} If $x$ is in the nullspace of $A$ show that $M^{-1} x$ is in the nullspace of $M^{-1} A M$. \textbf{A.}

        $$
        \begin{aligned}
        A x &=0\\
        \left(M^{-1} A M\right)\left(M^{-1} x\right) &=M^{-1} A\left(M M^{-1}\right) x \\
        &=M^{-1}(A x) \\
        &=M^{-1}(0) \\
        &=0
        \end{aligned}
        $$

        Vector $M^{-1} x$ lays in the null space of $M^{-1} A M$

        \item [b.] \textbf{Q.} The nullspaces of $A$ and $M^{-1} A M$ have the same (vectors)(basis)(dimension). 
        
        \textbf{A.} The null spaces of matrices $A$ and $M^{-1} A M$ have different vectors. If $\mathrm{x}$ lies in the null space of A, then $M^{-1} x$ lie in the null space of matrix $M^{-1} A M$. Therefore they will have different basis but have the same dimension.
        
        
    \end{enumerate}
    
\end{enumerate}

\subsection{Section 6.7}
\begin{enumerate}
    \item [1.] \textbf{Q.} Find the eigenvalues and unit eigenvectors $v_{1}, v_{2}$ of $A^{\mathrm{T}} A$. Then find $u_{1}=A v_{1} / \sigma_{1}$ : 
    
    $$
    A=\left[\begin{array}{ll}1 & 2 \\ 3 & 6\end{array}\right]  \text{ and } 
    A^{\mathrm{T}} A=\left[\begin{array}{ll}10 & 20 \\ 20 & 40\end{array}\right] \text{ and }
    AA^{\mathrm{T}}=\left[\begin{array}{rr}5 & 15 \\ 15 & 45\end{array}\right]
    $$ 
    
    Verify that $\boldsymbol{u}_{1}$ is a unit eigenvector of $A A^{\mathrm{T}}$. Complete the matrices $U, \Sigma, V$.
    
    $$
    \text{SVD } \quad
    \left[\begin{array}{ll}
    1 & 2 \\
    3 & 6
    \end{array}\right]=\left[\begin{array}{ll}
    u_{1} & u_{2}
    \end{array}\right]\left[\begin{array}{ll}
    \sigma_{1} & \\
    & 0
    \end{array}\right]\left[\begin{array}{ll}
    v_{1} & v_{2}
    \end{array}\right]^{\mathrm{T}}
    $$
    
    \textbf{A.} $A, A^{T} A^{\text {and }} A A^{T}$ have only one independent row, therefore all are rank $1$. $A^{T} A$ is singular, one of its eigenvalue is zero and another is equal to its trace 50. 

    $$
    \begin{aligned}
    \lambda &= 0\\
    \left(A^{T} A-\lambda I\right) x&=0\\
    \left[\begin{array}{ll}
    10 & 20 \\
    20 & 40
    \end{array}\right]\left[\begin{array}{l}
    x \\
    y
    \end{array}\right] &=\left[\begin{array}{l}
    0 \\
    0
    \end{array}\right] \\
    {\left[\begin{array}{l}
    x \\
    y
    \end{array}\right] } &=\left[\begin{array}{l}
    2 \\
    -1
    \end{array}\right]\\
    &\frac{1}{\sqrt{5}}\left[\begin{array}{l}
    2 \\
    -1
    \end{array}\right]\\
    \lambda &= 50\\
    \left(A^{T} A-\lambda I\right) x&=0\\
    {\left[\begin{array}{cc}
    -40 & 20 \\
    20 & -10
    \end{array}\right]\left[\begin{array}{l}
    x \\
    y
    \end{array}\right] } &=\left[\begin{array}{l}
    0 \\
    0
    \end{array}\right] \\
    {\left[\begin{array}{l}
    x \\
    y
    \end{array}\right] } &=\left[\begin{array}{l}
    1 \\
    2
    \end{array}\right]\\
    &\frac{1}{\sqrt{5}}\left[\begin{array}{l}
    1 \\
    2
    \end{array}\right]\\
    \sigma_{1}^{2}&=50 \\
    \sigma_{1}&=\sqrt{50} \\
    u_{I} &=\frac{A v_{1}}{\sigma_{1}} \\
    &=\frac{1}{\sqrt{50}} \frac{1}{\sqrt{5}}\left[\begin{array}{ll}
    1 & 2 \\
    3 & 6
    \end{array}\right]\left[\begin{array}{l}
    1 \\
    2
    \end{array}\right] \\
    &=\frac{1}{\sqrt{10}}\left[\begin{array}{l}
    1 \\
    3
    \end{array}\right] \\
    u_{2}&=\frac{1}{\sqrt{10}}\left[\begin{array}{l}
    3 \\
    -1
    \end{array}\right] \quad \text{vector $u_{2}$ is orthogonal to vector $u_{1}$}\\
    \left[\begin{array}{ll}
    1 & 2 \\
    3 & 6
    \end{array}\right]&=\frac{1}{\sqrt{10}}\left[\begin{array}{cc}
    1 & 3 \\
    3 & -1
    \end{array}\right]\left[\begin{array}{cc}
    \sqrt{50} & 0 \\
    0 & 0
    \end{array}\right] \frac{1}{\sqrt{5}}\left[\begin{array}{cc}
    1 & 2 \\
    2 & -1
    \end{array}\right]
    \end{aligned}
    $$
    
    \item [2.] \textbf{Q.} Write down orthonormal bases for the four fundamental subspaces of this $A$. 
    
    \textbf{A.} The rank of the matrix A is one, therefore matrix A is singular.
    $$
    \begin{aligned}
    \left[\begin{array}{ll:l}
    1 & 2 & 0 \\
    3 & 6 & 0
    \end{array}\right]&\\
    R_{2}-3 R_{1} &\rightarrow R_{2}\\
    \left[\begin{array}{ll:l}
    1 & 2 & 0 \\
    0 & 0 & 0
    \end{array}\right]&\\
    x+2 y &= 0\\
    y \text{ free variable }\\
    y &= -1\\
    x &= 2\\
    \frac{1}{\sqrt{5}}\left[\begin{array}{l}
    2 \\
    -1
    \end{array}\right]& \text{ basis for null space}\\
    A^{T}&=\left[\begin{array}{ll}
    1 & 3 \\
    2 & 6
    \end{array}\right]\\
    \frac{1}{\sqrt{10}}\left[\begin{array}{l}
    3 \\
    -1
    \end{array}\right] & \text{ basis for the null space of $A^T$}
    \end{aligned}
    $$

    There is one independent row therefore the basis for the row space of $A$ is: $\frac{1}{\sqrt{5}}\left[\begin{array}{l}1 \\ 2\end{array}\right]$. There is one independent column therefore the basis for the column space of $A$ is:
    $\frac{1}{\sqrt{10}}\left[\begin{array}{l}1 \\ 3\end{array}\right]$
        
\end{enumerate}

\subsection{Section 7.1}
\begin{enumerate}
    \item [3.] (only acef) Which of these transformations are not linear? The input is $v=\left(v_{1}, v_{2}\right)$ :
    \begin{enumerate}
        \item [a.] \textbf{Q.} $T(\boldsymbol{v})=\left(v_{2}, v_{1}\right)$ 
        
        \textbf{A.} Transformation $T: \mathbf{V} \rightarrow \mathbf{V}$ is linear if the following conditions are satisfied:
        $$
        \begin{aligned}
        T(v+w) & =T(v)+T(w) & \\
        T(c v) & =c T(v) \quad \text { for all } v, w \in \mathbf{V}, c \in \mathbb{R}
        \end{aligned}
        $$

        For first condition let $\boldsymbol{w}=\left(w_{1}, w_{2}\right)$ be another vector.

        $$
        \begin{aligned}
        T(v+\boldsymbol{w}) &=T\left(\left(v_{1}, v_{2}\right)+\left(w_{1}, w_{2}\right)\right) \\
        &=T\left(v_{1}+w_{1}, v_{2}+w_{2}\right) \\
        &=\left(v_{2}+w_{2}, v_{1}+w_{1}\right) \\
        &=\left(v_{2}, v_{1}\right)+\left(w_{2}, w_{1}\right) \\
        &=T\left(v_{1}, v_{2}\right)+T\left(w_{1}, w_{2}\right) \\
        &=T(\boldsymbol{v})+T(\boldsymbol{w})
        \end{aligned}
        $$

        Second condition 

        $$
        \begin{aligned}
        T(c v) &=T\left(c\left(v_{1}, v_{2}\right)\right) \\
        &=T\left(c v_{1}, c v_{2}\right) \\
        &=\left(c v_{2}, c v_{1}\right) \\
        &=c\left(v_{2}, v_{1}\right) \\
        &=c T\left(v_{1}, v_{2}\right) \\
        &=c T(v)
        \end{aligned}
        $$

        $T(v+w)=T(v)+T(w), T(c v)=c T(v)$, the transformation is linear
        
        \item [c.] \textbf{Q.}$T(v)=\left(0, v_{1}\right)$ 
        
        \textbf{A.} For first condition let $\boldsymbol{w}=\left(w_{1}, w_{2}\right)$ be another vector.

        $$
        \begin{aligned}
        T(v+w) &=T\left(\left(v_{1}, v_{2}\right)+\left(w_{1}, w_{2}\right)\right) \\
        &=T\left(v_{1}+w_{1}, v_{2}+w_{2}\right) \\
        &=\left(0, v_{1}+w_{1}\right) \\
        &=\left(0, v_{1}\right)+\left(0, w_{1}\right) \\
        &=T\left(v_{1}, v_{2}\right) &+T\left(w_{1}, w_{2}\right) \\
        &=T(v)+T(w)
        \end{aligned}
        $$
        
        Second condition
        
        $$
        \begin{aligned}
        T(c v) &=T\left(c\left(v_{1}, v_{2}\right)\right) \\
        &=T\left(c v_{1}, c v_{2}\right) \\
        &=\left(0, c v_{1}\right) \\
        &=c\left(0, v_{1}\right) \\
        &=c T\left(v_{1}, v_{2}\right) \\
        &=c T(v)
        \end{aligned}
        $$

        $T(v+w)=T(v)+T(w), T(c v)=c T(v)$, the given transformation is linear.
        
        \item [e.] \textbf{Q.} $T(\boldsymbol{v})=v_{1}-v_{2}$ 
        
        \textbf{A.}  For first condition let $\boldsymbol{w}=\left(w_{1}, w_{2}\right)$ be another vector.

        $$
        \begin{aligned}
        T(\boldsymbol{v}+\boldsymbol{w}) &=T\left(\left(v_{1}, v_{2}\right)+\left(w_{1}, w_{2}\right)\right) \\
        &=T\left(v_{1}+w_{1}, v_{2}+w_{2}\right) \\
        &=v_{1}+w_{1}-v_{2}-w_{2} \\
        &=v_{1}-v_{2}+w_{1}-w_{2} \\
        &=T\left(v_{1}, v_{2}\right)+T\left(w_{1}, w_{2}\right) \\
        &=T(\boldsymbol{v})+T(\boldsymbol{w})
        \end{aligned}
        $$

        Second condition

        $$
        \begin{aligned}
        T(c v) &=T\left(c\left(v_{1}, v_{2}\right)\right) \\
        &=T\left(c v_{1}, c v_{2}\right) \\
        &=c v_{1}-c v_{2} \\
        &=c\left(v_{1}-v_{2}\right) \\
        &=c T\left(v_{1}, v_{2}\right) \\
        &=c T(v)
        \end{aligned}
        $$

        $T(v+w)=T(v)+T(\boldsymbol{w}), T(c v)=c T(v)$, the given transformation is linear
        
        \item [f.] \textbf{Q.} $T(\boldsymbol{v})=v_{1} v_{2}$. 
        
        \textbf{A.}  For first condition let $\boldsymbol{w}=\left(w_{1}, w_{2}\right)$ be another vector.

        \begin{aligned}
        T(v+w)&=T\left(\left(v_{1}, v_{2}\right)+\left(w_{1}, w_{2}\right)\right) \\
        &=T\left(v_{1}+w_{1}, v_{2}+w_{2}\right) \\
        &=\left(v_{1}+w_{1}\right)\left(v_{2}-w_{2}\right) \\
        &=v_{1} v_{2}-v_{1} w_{2}+v_{2} w_{1}-w_{1} w_{2} \\
        &\neq v_{1} v_{2}+w_{1} w_{2} \\
        &=T\left(v_{1}, v_{2}\right)+T\left(w_{1}, w_{2}\right) \\
        &=T(v)+T(\boldsymbol{w})
        \end{aligned}

        $T(v+w) \neq T(v)+T(w)$, the given transformation is not linear.
        
    \end{enumerate}
    
    \item [5.] \textbf{Q.} Suppose $T(v)=v$ except that $T\left(0, v_{2}\right)=(0,0)$. Show that this transformation satisfies $T(c v)=c T(v)$ but not $T(v+w)=T(v)+T(w)$ 
    
    \textbf{A.} 

    $$
    \begin{aligned}
    T(c v) &=T\left(\left(c v_{1}, c v_{2}\right)\right) \\
    &=\left(c v_{1}, c v_{2}\right)\\
    c T(v) &=c\left(v_{1}, v_{2}\right) \\
    &=\left(c v_{1}, c v_{2}\right)\\
    T(c v) &= c T(v)
    \end{aligned}
    $$

    Assume $\boldsymbol{v}=(1,1)$ and $\boldsymbol{w}=(-1,0)$ for simplicity.

    $$
    \begin{aligned}
    v+w&=(0,1) \\
    T(v+w)&=(0,0)\\
    T(v)+T(w) &=(1,1)+(-1,0) \\
    &=(0,1)
    \end{aligned}
    $$
    
    The given transform indeed does not satisfy $T(v+w)=T(v)+T(w)$
    
    \item [12.] Suppose a linear $T$ transforms $(1,1)$ to $(2,2)$ and $(2,0)$ to $(0,0)$. Find $T(v)$ :
    \begin{enumerate}
        \item [a.] \textbf{Q.} $v=(2,2)$ \textbf{A.}

        $$
        \begin{aligned}
        T(c v) &=c T(v)\\
        T(2,2) &=2(T(1,1)) \\
        &=2(2,2) \\
        &=(4,4)
        \end{aligned}
        $$
        
        \item [b.] \textbf{Q.} $v=(3,1)$ \textbf{A.}

        $$
        \begin{aligned}
        T(v+w)&=T(v)+T(w)\\
        T(3,1) &=T(1,1)+T(2,0) \\
        &=(2,2)+(0,0) \\
        &=(2,2)
        \end{aligned}
        $$
        
        \item [c.] \textbf{Q.} $v=(-1,1)$ \textbf{A.}

        $$
        \begin{aligned}
        T(\boldsymbol{v}+\boldsymbol{w})&=T(\boldsymbol{v})+T(\boldsymbol{w})\\
        T(-1,1) &=T(1,1)+(-T(2,0)) \\
        &=(2,2)-(0,0) \\
        &=(2,2)
        \end{aligned}
        $$
        
        \item [d.] \textbf{Q.} $v=(a, b)$ \textbf{A.}

        $$
        \begin{aligned}
        T(\boldsymbol{v}+\boldsymbol{w})&=T(\boldsymbol{v})+T(\boldsymbol{w})\\
        T(a, b) &=b(T(1,1))+\frac{a-b}{2}(T(2,0)) \\
        &=b(2,2)+(0,0)
        \end{aligned}
        $$
        
    \end{enumerate}
    
\end{enumerate}

\subsection{Section 7.2}
\begin{enumerate}
    \item [14.] 
    \begin{enumerate}
        \item [a.] \textbf{Q.} What matrix transforms $(1,0)$ into $(2,5)$ and transforms $(0,1)$ to $(1,3)$ ? 
        
        \textbf{A.} The constants from the transform go into the columns of the matrix. The number of basis vectors is 2, will have a 2 by 2 matrix.
    
        $$
        \left[\begin{array}{ll}
        2 & 1 \\
        5 & 3
        \end{array}\right]
        $$
        
        \item [b.] \textbf{Q.} What matrix transforms $(2,5)$ to $(1,0)$ and $(1,3)$ to $(0,1)$ ? 
        
        \textbf{A.} The number of basis vectors is 2, will have a 2 by 2 matrix. Operations here are the inverse of above. We are changing the output space back to the input space. Thus, the required matrix will be the inverse of $\left[\begin{array}{ll}2 & 1 \\ 5 & 3\end{array}\right]$ which is
        $$
        \left[\begin{array}{cc}
        3 & -1 \\
        -5 & 2
        \end{array}\right]
        $$
        
        \item [c.] \textbf{Q.} Why does no matrix transform $(2,6)$ to $(1,0)$ and $(1,3)$ to $(0,1)$ ? 
        
        \textbf{A.}  $\left[\begin{array}{l}2 \\ 6\end{array}\right]=2\left[\begin{array}{l}1 \\ 3\end{array}\right]$. The product of the matrix A with the vector $(2,6)$ should be double of its product with $(1,3)$ This contradicts our requirements. Thus, there is not matrix that transforms $(2,6)$ into $(1,0)$ and $(1,3)$ to $(0,1)$.
        
    \end{enumerate}
\end{enumerate}

\end{document}