\documentclass[main.tex]{subfiles}
\begin{document}

\href{https://www2.seas.gwu.edu/~simhaweb/quantum/modules/module1/module1.html}{Module 1 Quantum Strangeness}

\begin{enumerate}

\item[] \textbf{In-Class Exercise 1:} \textbf{Q.} What is the shape of the histogram? \textbf{A.} It should be a Gaussian on the $Z$ axis. \textbf{Q} Draw out a potential histogram by completing the histogram above. \textbf{A.} Figure \ref{fig:e01Gaussian}. \textbf{Q.} What would we expect to see with thinner and thinner bands (many of them), and lots of pellets? \textbf{A.} A continuous distribution. \textbf{Q.} If we replaced horizontal bands with vertical bands near the x-axis to get an x-histogram, what would it look like? \textbf{A.} It would also be Gaussian.

\begin{figure}
  \centering
  \includegraphics[width=3in]{modules/figs/m01/e01gaussian.jpeg}
  \caption{Gaussian distribution}
  \label{fig:e01Gaussian}
\end{figure}

\item[] \textbf{In-Class Exercise 2:} \textbf{Q.} How would your responses change if we placed a hole-in-a-wall between the gun and the screen? \textbf{A.} A more concentrated distribution.

\item[] \textbf{In-Class Exercise 3:}  Now consider moving the pellet gun further back so that the spread of trajectories is wide enough to encompass both holes shown here: \textbf{Q.} What would the x and y histograms look like? \textbf{A.} Figure \ref{fig:e03xaxis} and Figure \ref{fig:e03yaxis}. \textbf{Q.} And how would those change if we changed holes to slits? \textbf{A.} The x axis histogram would not change and the y axis histogram would become evenly distributed amongst the bins.

\begin{figure}
  \centering
  \includegraphics[width=3in]{modules/figs/m01/e03xaxis.png}
  \caption{Two hole x axis histogram}
  \label{fig:e03xaxis}
\end{figure}

\begin{figure}
  \centering
  \includegraphics[width=3in]{modules/figs/m01/e03yaxis.png}
  \caption{Two hole y axis histogram}
  \label{fig:e03yaxis}
\end{figure}

\item[] \textbf{In-Class Exercise 4:}

    \begin{enumerate}
        \item[1.] \textbf{Q.} What is the source of randomness in each case? \textbf{A.} The randomness from the coin flip originates from the unknown amount of force exerted on the coin when flipped, and randomness from the card shuffle is the unknown permutation that the cards are currently in as the shuffle process moves them around.
        \item[2.] \textbf{Q.} the coin-flip, could one in principle, knowing the launch velocity of the coin (and other variables) predict the outcome? What variables would be needed? Could a mechanical contraption be designed to flip a coin predictably? \textbf{A.} You could design a device that could remove most of the randomness if you can account for aerodynamic physics of the flip.
        \item[3.] \textbf{Q.} Does this predictability extend to the card example? \textbf{A.} Yes, as long as you can maintain knowledge of the current permutation as the shuffle is progressed.
        \item[4.] What is an example of a physical measurement that is likely to have random errors, and for which an average over several measurements is needed? \textbf{Q.} Rolling dice is similar to coin flip example where calculating the forces on the die to consistently roll the same number would be challenging. Higher number of rolls will result in an even distribution of 1/6 for each side. For a six sided die.
    \end{enumerate}

\item[] \textbf{In-Class Exercise 5:} \textbf{Q.} How would you describe the motion of the bar magnet? \textbf{A.} Alignment with the homogeneous magnetic field via rotation, bar magnet north will align with south pole of large magnet. \textbf{Q.} What is the net force experienced by the center of the bar magnet? \textbf{A.} The net force at the center of the bar magnet is zero.

\item[] \textbf{In-Class Exercise 6:} \textbf{Q.} Now how would you describe the motion of the bar magnet? \textbf{A.} The bar magnet will rotate in aliment with the field and move north. \textbf{Q.} What is the net force experienced by the center of the bar magnet, and why? \textbf{A.} The net force at the center of the bar magnet is the one pulling it up.

\item[] \textbf{In-Class Exercise 7:} \textbf{Q.} What does one expect for the z-histogram (vertical), and why? That is, what is the histogram going to look like if we count the "hits" in each little blue box? \textbf{A.} The z axis will exhibit a Gaussian distribution with its center above the x axis, this is due to the stronger field from the magnet on top. \textbf{Q.} If one knows the landing spot for a particular bar magnet, how could one compute the vertical force experienced by the magnet? \textbf{A.} Use the shift of the Gaussian up to be able to calculate the strength of the field.

\item[] \textbf{In-Class Exercise 8:} \textbf{Q.} What other questions might one ask? What are you curious to know about this type of experiment? \textbf{A.} What are the velocity of the silver atom, does relativity play a role. Why does nitrogen have 4 clusters, and so on. Do fixed orbitals of the electrons play a role.

\item[] \textbf{In-Class Exercise 9:} \textbf{Q.} Sketch, with modest neatness, the Type-1 and Type-2 diagrams for a two-stage SG experiment when the orientation of the second apparatus is

    \begin{enumerate}
        \item[1.] $\theta=0$
        \item[2.] $\theta=\pi$
        \item[3.] $\theta=\frac{\pi}{2}$
        \item[4.] $\theta=\frac{\pi}{4}$
    \end{enumerate}
    
    \textbf{A.} Figure \ref{fig:e09_Type_I}, and Figure \ref{fig:e09_Type_II}. 
    
    \begin{figure}
        \centering
        \includegraphics[width=3in]{modules/figs/m01/e09_Type_I_Zero.png}
        \includegraphics[width=3in]{modules/figs/m01/e09_Type_I_Pi.png}
        \includegraphics[width=3in]{modules/figs/m01/e09_Type_I_Pi_Over_2.png}
        \includegraphics[width=3in]{modules/figs/m01/e09_Type_I_Pi_Over_4.png}
        \caption{Type I}
        \label{fig:e09_Type_I}
    \end{figure}
    
    \begin{figure}
      \centering
      \includegraphics[width=3in]{modules/figs/m01/e09_Type_II_Zero.png}
      \includegraphics[width=3in]{modules/figs/m01/e09_Type_II_Pi.png}
      \includegraphics[width=3in]{modules/figs/m01/e09_Type_II_Pi_Over_2.png}
      \includegraphics[width=3in]{modules/figs/m01/e09_Type_II_Pi_Over_4.png}
      \caption{Type II}
      \label{fig:e09_Type_II}
    \end{figure}

\item[] \textbf{In-Class Exercise 10:} \textbf{Q.} Sketch Type-1 and Type-2 diagrams for a three-stage SG experiment where:

    \begin{enumerate}
        \item[1.] The + output of the first stage is fed into the second.
        \item[2.] The + output of the second stage is fed into a third apparatus.
        \item[3.] The orientations of the three stages are: $\theta_{1}=0$, $\theta_{2}=\frac{\pi}{2}$, $\theta_{3}=0$.
    \end{enumerate}
    
    \textbf{A.} Figure \ref{fig:e10_Type_I}, and Figure \ref{fig:e10_Type_II}.

    \begin{figure}
        \centering
        \includegraphics[width=4in]{modules/figs/m01/e10_Type_I.png}
        \caption{Type I}
        \label{fig:e10_Type_I}
    \end{figure}
    
    \begin{figure}
      \centering
      \includegraphics[width=4in]{modules/figs/m01/e10_Type_II.png}
      \caption{Type II}
      \label{fig:e10_Type_II}
    \end{figure}

\item[] \textbf{In-Class Exercise 11:} \textbf{Q.} How does one test that the splitting between two streams is random? Is it enough to merely establish that 50\% exit through the + stream? \textbf{A.} Strict alternation is not random, if there is a sequence, than that is not random. However this source of randomness is one of the best sources.

\item[] \textbf{In-Class Exercise 12:} \textbf{Q.} Explore the above notion of independence. Why does it matter? \textbf{A.} We need to understand the impact of the orientation of the magnetic field on the trajectory of the silver atoms. When are the trajectories independent of the field orientation.

\item[] \textbf{In-Class Exercise 13:} \textbf{Q.} What is a reasonable guess for the probabilities at two rightmost outputs? Start by asking: what is the probability that an atom arrives at the output of the merge apparatus? Then, further, what is the probability of such an atom to appear at the + output of the third stage? What is, therefore, the overall probability that an atom entering the first appears at the + of the third? [Answer provided in class] \textbf{A.} + = 0.25 and - = 0.25 is incorrect. In fact + = 0.5 and - = 0.

\item[] \textbf{In-Class Exercise 14:} \textbf{Q.} To clarify the orientation of the second SG apparatus, draw the Type-1 diagram for the above case. \textbf{A.} Figure \ref{fig:e14_Type_I}.

    \begin{figure}
        \centering
        \includegraphics[width=4in]{modules/figs/m01/e14_Type_I.png}
        \caption{Type I}
        \label{fig:e14_Type_I}
    \end{figure}

\item[] \textbf{In-Class Exercise 15:} \textbf{Q.} What are your own additional questions? \textbf{A.} When we have a merge apparatus that is not controlled why are all the + z orientation atoms preserved with a + = 0.5 and - = 0.0

\item[] \textbf{In-Class Exercise 16:} \textbf{Q.} What is observed when the first stage is an X-aligned SG? \textbf{A.} For X-aligned to Z-aligned + = 0.25 and - = 0.25. For X aligned to X aligned + = 0.5 and - = 0.0.

\item[] \textbf{In-Class Exercise 17:} \textbf{Q.} Consider the following two polarizers in sequence (Figure #). What fraction of the input light reaches the other (right) side of the second polarizer? \textbf{A.} $0\%$ of the light will be visible, the vertical polarizer will remove the horizontal component, and the horizontal polarizer will remove the vertical component.

\item[] \textbf{In-Class Exercise 18:} \textbf{Q.} Consider the inserting a 45∘ polarizer in the middle (Figure #). What fraction of the input light reaches the other (right) side of the third polarizer? \textbf{A.} The middle filter that adds a $45^{\circ}$ polarizer will allow some light to be visible on the other size of the third polarizer.

\item[] \textbf{In-Class Exercise 19:} \textbf{Q.} What is the distribution (histogram) expected if both slits are open and if light is made of particles? What is the height of the distribution at the center? \textbf{A.} If light is made of particles you expect a bi-modal Gaussian distribution of equal height on the other side of each hole. In reality there is a an interference pattern.

\item[] \textbf{In-Class Exercise 20:} \textbf{Q.} For a given photon, what is the probability that detector RT goes off? What are the probabilities for the other detectors? \textbf{A.} $\frac{1}{4}$, each beam splitter splits equally so you expect $\frac{1}{4}$ at each detector.

\item[] \textbf{In-Class Exercise 21:} \textbf{Q.} For a given photon, what are the probabilities for detecting at D1 vs D2? Does the adjustable distance matter? \textbf{A.} $\frac{1}{2}$ at each, the adjustable distance does not matter.

\item[] \textbf{In-Class Exercise 22:} \textbf{Q.} Based only on the reasoning in the previous two exercises, what is the probability of detection at D1? At D2? \textbf{A.} All photons are detected at $D1$, the adjustable distance can be changed to make all of the photons detected at $D2$. The interference pattern matters.

\end{enumerate}
\end{document}