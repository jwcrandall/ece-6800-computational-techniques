\documentclass[main.tex]{subfiles}
\begin{document}

\subsection{Section 4.1}
\begin{enumerate}
    \item [3.] Construct a matrix with the required property or say why that is impossible:
    \begin{enumerate}
        \item [a.] \textbf{Q.} Column space contains $\left[\begin{array}{c}1 \\ 2 \\ -3\end{array}\right]$ and $\left[\begin{array}{c}2 \\ -3 \\ 5\end{array}\right]$, nullspace contains $\left[\begin{array}{l}1 \\ 1 \\ 1\end{array}\right]$ 
        
        \textbf{A.} Column space contains two independent vectors. Therefore the rank of matrix $r = 2$. The dimension of the null space is 1. $n-r=1$, $n-2=1$, therefore $n=3$.
        
        $$
        \begin{aligned}
        A&=\left[\begin{array}{ccc}
        1 & 2 & a \\
        2 & -3 & b \\
        -3 & 5 & c
        \end{array}\right]\\
        \left[\begin{array}{ccc}
        1 & 2 & a \\
        2 & -3 & b \\
        -3 & 5 & c
        \end{array}\right]\left[\begin{array}{l}
        1 \\
        1 \\
        1
        \end{array}\right]&=\left[\begin{array}{l}
        0 \\
        0 \\
        0
        \end{array}\right]\\
        \left[\begin{array}{c}
        3+a \\
        -1+b \\
        2+c
        \end{array}\right]&=\left[\begin{array}{l}
        0 \\
        0 \\
        0
        \end{array}\right]\\
        3+a & =0 \\
        a  &= -3 \\
        -1+b & =0 \\
        b &= 1 \\
        2+c &= 0 \\
        c &= -2 \\
        A &= \left[\begin{array}{ccc}
        1 & 2 & -3 \\
        2 & -3 & 1 \\
        -3 & 5 & -2
        \end{array}\right]
        \end{aligned}
        $$
        
        \item [b.] \textbf{Q.} Row space contains $\left[\begin{array}{c}1 \\ 2 \\ -3\end{array}\right]$ and $\left[\begin{array}{c}2 \\ -3 \\ 5\end{array}\right]$, nullspace contains $\left[\begin{array}{l}1 \\ 1 \\ 1\end{array}\right]$ 
        
        \textbf{A.}
        The row space vectors must be orthogonal to the vectors in the null space.
        
        $$
        \begin{aligned}
        {\left[\begin{array}{lll}
        2 & -3 & 5
        \end{array}\right]\left[\begin{array}{l}
        1 \\
        1 \\
        1
        \end{array}\right] } &=2-3+5 \\
        &=4 \\
        & \neq 0
        \end{aligned}
        $$
        
        Therefore the matrix does not exist.
        
        \item [c.] \textbf{Q.} $A x=\left[\begin{array}{l}1 \\ 1 \\ 1\end{array}\right]$ has a solution and $A^{\mathrm{T}}\left[\begin{array}{l}1 \\ 0 \\ 0\end{array}\right]=\left[\begin{array}{l}0 \\ 0 \\ 0\end{array}\right]$ 
        
        \textbf{A.} $\left[\begin{array}{l}1 \\ 1 \\ 1\end{array}\right]$ is in the column space of $A$, $\left[\begin{array}{l}1 \\ 0 \\ 0\end{array}\right]$ is in the null space $A^T$. The vectors should be orthogonal, but they are not.
        
        $$
        \begin{aligned}
        {\left[\begin{array}{lll}
        1 & 0 & 0
        \end{array}\right]\left[\begin{array}{l}
        1 \\
        1 \\
        1
        \end{array}\right] } &=1 \\
        & \neq 0
        \end{aligned}
        $$
        
        The matrix does not exist.
        
        \item [d.] \textbf{Q.} Every row is orthogonal to every column ($A$ is not the zero matrix) \textbf{A.} The product of every row with every column is zero. 
        
        $$
        \begin{aligned}
        A &=\left[\begin{array}{cc}
        1 & 1 \\
        -1 & -1
        \end{array}\right]\\
        A^{2} &=\left[\begin{array}{cc}
        1 & 1 \\
        -1 & -1
        \end{array}\right]\left[\begin{array}{cc}
        1 & 1 \\
        -1 & -1
        \end{array}\right] \\
        &=\left[\begin{array}{ll}
        0 & 0 \\
        0 & 0
        \end{array}\right]
        \end{aligned}
        $$
        
        \item [e.] \textbf{Q.} Columns add up to a column of zeros, rows add to a row of 1's. 
        
        \textbf{A.} Null space contains $(1,1,1)$, and row space contains $(1,1,1)$.
        
        $$
        \begin{aligned}
        {\left[\begin{array}{lll}
        1 & 1 & 1
        \end{array}\right]\left[\begin{array}{l}
        1 \\
        1 \\
        1
        \end{array}\right] } &=1+1+1 \\
        &=3 \\
        & \neq 0
        \end{aligned}
        $$
        
        Not orthogonal, matrix does not exist.
        
    \end{enumerate}
    
    \item [8.] \textbf{Q.} In Figure $4.3$, how do we know that $A x_{r}$ is equal to $A x$ ? How do we know that this vector is in the column space? If $A=\left[\begin{array}{ll}1 & 1 \\ 1 & 1\end{array}\right]$ and $x=\left[\begin{array}{l}1 \\ 0\end{array}\right]$ what is $x_{r}$ ? 
    
    \textbf{A.} $x_{r}$ row space and $x_{n}$ null space.
    
    $$
    \begin{aligned}
    x &= x_{r}+x_{n}\\
    A x_{n} &= 0 \\
    A x &= A x_{r}+A x_{n} \\
    &= A x_{r}
    \end{aligned}
    $$
    
    $A x_{r}$ is in column space because $A x \in C(A)$
    
    $$
    \begin{aligned}
    A x_{n} &= 0\\
    \left[\begin{array}{ll}
    1 & 1 \\
    1 & 1
    \end{array}\right] \cdot\left[\begin{array}{l}
    x \\
    y
    \end{array}\right] &=0 \\
    x+y &=0 \\
    x &=-y \\
    x&=1 \\
    y&=-1 \\
    x_{n} &= \left[\begin{array}{c}
    1 \\
    -1
    \end{array}\right] \\
    x_{r} &=\left[\begin{array}{l}
    1 \\
    0
    \end{array}\right]-\left[\begin{array}{c}
    1 \\
    -1
    \end{array}\right] \\
    &=\left[\begin{array}{l}
    0 \\
    1
    \end{array}\right]
    \end{aligned}
    $$

    \item [12.] \textbf{Q.} Find the pieces $x_{r}$ and $x_{n}$ and draw Figure $4.3$ properly if
    $$
    A=\left[\begin{array}{rr}
    1 & -1 \\
    0 & 0 \\
    0 & 0
    \end{array}\right] \text { and } x=\left[\begin{array}{l}
    2 \\
    0
    \end{array}\right]
    $$
    
    \textbf{A.}
    $\bm{x} &= \bm{x}_{r}+\bm{x}_{n}$ , matrix $A$ is in reduced echelon form and the leading 1 is in the first row.  The basis or row space of $A$ is $(1,-1)$.
    
    $$
    \begin{aligned}
    \bm{x}_{r} &=\left[\begin{array}{c}
    1 \\
    -1
    \end{array}\right]\\
    \end{aligned}
    $$
    
    $x_{2}$ is a free variable, $x_2 = t$, with t as a parameter.
    
    $$
    \begin{aligned}
    x_{1}-x_{2} &= 0 \\
    \left[\begin{array}{l}
    x_{1} \\
    x_{2}
    \end{array}\right] &= \left[\begin{array}{l}
    t \\
    t
    \end{array}\right] \\
    &=t\left[\begin{array}{l}
    1 \\
    1
    \end{array}\right]\\
    \bm{x}_{n}&=\left[\begin{array}{l}
    1 \\
    1
    \end{array}\right]
    \end{aligned}
    $$
    
    Basis of null space of $A$ is $\left[\begin{array}{l}1 \\ 1\end{array}\right]$.
    
    $$
    \begin{aligned}
    \boldsymbol{x}_{r}+\boldsymbol{x}_{n} &=\left[\begin{array}{c}
    1 \\
    -1
    \end{array}\right]+\left[\begin{array}{l}
    1 \\
    1
    \end{array}\right] \\
    &=\left[\begin{array}{c}
    1+1 \\
    -1+1
    \end{array}\right] \\
    &=\left[\begin{array}{l}
    2 \\
    0
    \end{array}\right] \\
    &=\boldsymbol{x}
    \end{aligned}
    $$
    
    \item [16.] \textbf{Q.} Prove that every $y$ in $N\left(A^{\mathrm{T}}\right)$ is perpendicular to every $A x$ in the column space, using the matrix shorthand of equation $(2)$. Start from $A^{\mathrm{T}} \boldsymbol{y}=\mathbf{0}$. 
    
    \textbf{A.}
    
    $$
    \begin{aligned}
    \bm{y} &\in \bm{N}\left(\bm{A}^{T}\right)\\
    A^{T} \bm{y} &= \bm{0}\\
    (A \bm{x})^{T} \bm{y} &= \bm{x}^{T}\left(A^{T} \bm{y}\right) \\
    &= \bm{x}^{T}(\bm{0}) \\
    &= \bm{0}
    \end{aligned}
    $$
    
    $y$ is perpendicular to any $Ax$ in the column space.
    
    \item [28.] Why is each of these statements false?
    \begin{enumerate}
        \item [a.] \textbf{Q.} $(1,1,1)$ is perpendicular to $(1,1,-2)$ so the planes $x+y+z=0$ and $x+y-$ $2 z=0$ are orthogonal subspaces.
        
        \textbf{A.} Orthogonality is impossible when $\operatorname{dim}(x+y+z) + \operatorname{dim}(x+y+-2z) > \text{ dimension of whole space}$. $6>5$.\\
        
        \item [b.] \textbf{Q.} The subspace spanned by $(1,1,0,0,0)$ and $(0,0,0,1,1)$ is the orthogonal complement of the subspace spanned by $(1,-1,0,0,0)$ and $(2,-2,3,4,-4)$. 
        
        \textbf{A.} The vectors $(1,1,0,0,0)$ and $(0,0,0,1,1)$ are linearly independent, thus their subspace is 2 dimensional. The vectors $(1,-1,0,0,0)$ and $(2,-2,3,4,-4)$ are linearly independent, thus their subspace is 2 dimensional. Both subspaces are in $\mathbf{R}^{5}$, but the dimensions of the subspace and its orthogonal complement do not sum to $5$. Therefore, the statement is false.  
        
        \item [c.] \textbf{Q.} Two subspaces that meet only in the zero vector are orthogonal. 
        
        \textbf{A.} Two two dimensional lines meet at point $(0,0)$. Therefore the subspaces meet only in the zero vector. However they are not orthogonal.
        
        $$
        \begin{aligned}
        2x+y=0 \\
        x+y=0 \\
        (2,1) \cdot(1,1) & =2+1 \\
        &=3 \\
        & \neq 0
        \end{aligned}
        $$
        
        Therefore the statement is false.
        
    \end{enumerate}
\end{enumerate}

\subsection{Section 4.2}
\begin{enumerate}
    \item [1.] Project the vector $\boldsymbol{b}$ onto the line through $\boldsymbol{a}$. Check that $\boldsymbol{e}$ is perpendicular to $\boldsymbol{a}$:
    \begin{enumerate}
        \item [a.] \textbf{Q.} $b=\left[\begin{array}{l}1 \\ 2 \\ 2\end{array}\right]$ and $a=\left[\begin{array}{l}1 \\ 1 \\ 1\end{array}\right]$ 
        
        \textbf{A.}
        
        $$
        \begin{aligned}
        \bm{p}&=\frac{\bm{a}^{T} \bm{b}}{\bm{a}^{T} \bm{a}} \bm{a}\\
        &=\frac{\left[\begin{array}{lll}
        1 & 1 & 1
        \end{array}\right]\left[\begin{array}{l}
        1 \\
        2 \\
        2
        \end{array}\right]}{\left[\begin{array}{lll}
        1 & 1 & 1
        \end{array}\right]\left[\begin{array}{l}
        1 \\
        1 \\
        1
        \end{array}\right]}\bm{a}\\
        & = \frac{5}{3}\bm{a}\\
        \bm{e} &=\bm{b}-\bm{p} \\
        & = \left[\begin{array}{l}
        1 \\
        2 \\
        2
        \end{array}\right]-\frac{5}{3}\left[\begin{array}{l}
        1 \\
        1 \\
        1
        \end{array}\right] \\
        & = \left[\begin{array}{c}
        \frac{-2}{3} \\
        \frac{1}{3} \\
        \frac{1}{3}
        \end{array}\right] \\
        & = \frac{1}{3}\left[\begin{array}{c}
        -2 \\
        1 \\
        1
        \end{array}\right]\\
        \bm{e}\cdot \bm{a} &= \frac{1}{3}\left[\begin{array}{c}
        -2 \\
        1 \\
        1
        \end{array}\right] \cdot\left[\begin{array}{l}
        1 \\
        1 \\
        1
        \end{array}\right] \\
        &=\frac{1}{3}[-2+1+1] \\
        &=0
        \end{aligned}
        $$
        
        \item [b.] \textbf{Q.} $\boldsymbol{b}=\left[\begin{array}{l}1 \\ 3 \\ 1\end{array}\right]$ and $a=\left[\begin{array}{l}-1 \\ -3 \\ -1\end{array}\right]$ 
        
        \textbf{A.}
        $$
        \begin{aligned}
        \bm{p}&=\frac{\bm{a}^{T} \bm{b}}{\bm{a}^{T} \bm{a}} \bm{a}\\
        &=\frac{\left[\begin{array}{lll}
        -1 & -3 & -1
        \end{array}\right]\left[\begin{array}{l}
        1 \\
        3 \\
        1
        \end{array}\right]}{\left[\begin{array}{lll}
        -1 & -3 & -1
        \end{array}\right]\left[\begin{array}{r}
        -1 \\
        -3 \\
        -1
        \end{array}\right]}\bm{a}\\
        &=\frac{-11}{11} \bm{a} \\
        &=-\bm{a}\\
        \bm{e} &= \bm{b} - \bm{p} \\
        &=\left[\begin{array}{l}
        1 \\
        3 \\
        1
        \end{array}\right]-\left[\begin{array}{l}
        1 \\
        3 \\
        1
        \end{array}\right] \\
        &=\bm{0}\\
        \bm{e} \cdot \bm{a} &=\bm{0} \cdot \bm{a} \\
        &=0
        \end{aligned}
        $$
    
    \end{enumerate}


    \item [8.] \textbf{Q.} Project the vector $\boldsymbol{b}=(1,1)$ onto the lines through $\boldsymbol{a}_{1}=(1,0)$ and $\boldsymbol{a}_{2}=(1,2)$. Draw the projections $p_{1}$ and $p_{2}$ and add $p_{1}+p_{2}$. The projections do not add to $b$ because the $a$'s are not orthogonal. 
    
    \textbf{A.}
    $$
    \begin{aligned}
    \bm{p}_{1} &= \frac{\bm{a}_{1}^{T} \bm{b}}{\bm{a}_{1}^{T} \bm{a}_{1}} \bm{a}_{1}\\
    &=\frac{\left[\begin{array}{ll}
    1 & 0
    \end{array}\right]\left[\begin{array}{l}
    1 \\
    1
    \end{array}\right]}{\left[\begin{array}{ll}
    1 & 0
    \end{array}\right]\left[\begin{array}{l}
    1 \\
    0
    \end{array}\right]}\bm{a}_{1} \\
    &=\frac{1}{1}(1,0) \\
    &=(1,0) \\
    \bm{p}_{2} &= \frac{\bm{a}_{2}^{T} \bm{b}}{\bm{a}_{2}^{T} \bm{a}_{2}} \bm{a}_{2}\\
    & = \frac{\left[\begin{array}{ll}
    1 & 2
    \end{array}\right]\left[\begin{array}{l}
    1 \\
    1
    \end{array}\right]}{\left[\begin{array}{ll}
    1 & 2
    \end{array}\right]\left[\begin{array}{l}
    1 \\
    2
    \end{array}\right]}\bm{a}_{2}\\
    & = \frac{3}{5}(1,2) \\
    & = \left(\frac{3}{5}, \frac{6}{5}\right)\\
    \bm{p}_{1}+\bm{p}_{2} &=(1,0)+\left(\frac{3}{5}, \frac{6}{5}\right) \\
    &=\left(\frac{8}{5}, \frac{6}{5}\right) \\
    & \neq b
    \end{aligned}
    $$
    
    \item [11.] (Only a) Project $\boldsymbol{b}$ onto the column space of $A$ by solving $A^{\mathrm{T}} A \widehat{\boldsymbol{x}}=A^{\mathrm{T}} \boldsymbol{b}$ and $\boldsymbol{p}=A \widehat{\boldsymbol{x}}$ :
    \begin{enumerate}
        \item [a.] \textbf{Q.} $A=\left[\begin{array}{ll}1 & 1 \\ 0 & 1 \\ 0 & 0\end{array}\right]$ and $\bm{b}=\left[\begin{array}{l}2 \\ 3 \\ 4\end{array}\right]$. Find $\bm{e}=\bm{b}-\bm{p}$. It should be perpendicular to the columns of $A$. 
        
        \textbf{A.}
        $$
        \begin{aligned}
        \bm{A}^{T} \bm{A} &=\left[\begin{array}{lll}
        1 & 0 & 0 \\
        1 & 1 & 0
        \end{array}\right]\left[\begin{array}{ll}
        1 & 1 \\
        0 & 1 \\
        0 & 0
        \end{array}\right] \\
        &=\left[\begin{array}{ll}
        1 & 1 \\
        1 & 2
        \end{array}\right]\\
        \bm{A}^{T} \bm{b} &=\left[\begin{array}{lll}
        1 & 0 & 0 \\
        1 & 1 & 0
        \end{array}\right]\left[\begin{array}{l}
        2 \\
        3 \\
        4
        \end{array}\right] \\
        &=\left[\begin{array}{l}
        2 \\
        5
        \end{array}\right]\\
        \bm{A}^{T} \bm{A} \widehat{\bm{x}} &= \bm {A}^{T} \bm{b}\\
        \left[\begin{array}{ll}
        1 & 1 \\
        1 & 2
        \end{array}\right]
        \left[\begin{array}{l}
        \hat{x}_{1} \\
        \hat{x}_{2}
        \end{array}\right]&=\left[\begin{array}{l}
        2 \\
        5
        \end{array}\right]\\
        R_2 - R_1 & \rightarrow R_2\\
        \left[\begin{array}{ll}
        1 & 1 \\
        0 & 1
        \end{array}\right]\left[\begin{array}{l}
        \hat{x}_{1} \\
        \hat{x}_{2}
        \end{array}\right]&=\left[\begin{array}{l}
        2 \\
        3
        \end{array}\right]\\
        \hat{x}_{2} &=3 \\
        \hat{x}_{1}+\hat{x}_{2} &=2 \\
        \hat{x}_{1}+3 &=2 \\
        \hat{x}_{1} &=-1 \\
        \hat{x} &=\left[\begin{array}{l}
        \hat{x}_{1} \\
        \hat{x}_{2}
        \end{array}\right] \\
        &=\left[\begin{array}{c}
        -1 \\
        3
        \end{array}\right]\\
        \bm{p} &= \bm{A} \hat{\bm{x}} \\
        \bm{p} &=\left[\begin{array}{ll}
        1 & 1 \\
        0 & 1 \\
        0 & 0
        \end{array}\right]\left[\begin{array}{c}
        -1 \\
        3
        \end{array}\right] \\
        &=\left[\begin{array}{l}
        2 \\
        3 \\
        0
        \end{array}\right]\\
        \bm{e} &= \bm{p}- \bm{b} \\
        \bm{e} &=\left[\begin{array}{l}
        2 \\
        3 \\
        4
        \end{array}\right]-\left[\begin{array}{l}
        2 \\
        3 \\
        0
        \end{array}\right] \\
        &=\left[\begin{array}{l}
        0 \\
        0 \\
        4
        \end{array}\right]\\
        \bm{e}^{T} \cdot \bm{A} &=\left[\begin{array}{lll}
        0 & 0 & 4
        \end{array}\right]\left[\begin{array}{ll}
        1 & 1 \\
        0 & 1 \\
        0 & 0
        \end{array}\right] \\
        &=\left[\begin{array}{ll}
        0 & 0
        \end{array}\right] \\
        &=\bm{0}
        \end{aligned}        
        $$
        
    \end{enumerate}
    
\end{enumerate}

\subsection{Section 4.4}
\begin{enumerate}
    \item [2.] \textbf{Q.} The vectors $(2,2,-1)$ and $(-1,2,2)$ are orthogonal. Divide them by their lengths to find orthonormal vectors $\bm{q}_{1}$ and $\bm{q}_{2}$. Put those into the columns of $Q$ and multiply $Q^{\mathrm{T}} Q$ and $Q Q^{\mathrm{T}}$. 
    
    \textbf{A.}  
    
    $$
    \begin{aligned}
    \sqrt{\|(2,2,-1)\|} &=\sqrt{2^{2}+2^{2}+(-1)^{2}} \\
    &=\sqrt{9} \\
    &=3 \\
    \sqrt{\|(-1,2,2)\|} &=\sqrt{(-1)^{2}+2^{2}+(2)^{2}} \\
    &=\sqrt{9} \\
    &=3 \\
    \bm{q}_{1}&=\frac{1}{3}\left[\begin{array}{c}
    2 \\
    2 \\
    -1
    \end{array}\right] \\
    \bm{q}_{2}&=\frac{1}{3}\left[\begin{array}{c}
    -1 \\
    2 \\
    2
    \end{array}\right] \\
    Q&=\frac{1}{3}\left[\begin{array}{cc}
    2 & -1 \\
    2 & 2 \\
    -1 & 2
    \end{array}\right]\\
    Q^{T} Q &=\frac{1}{3}\left[\begin{array}{ccc}
    2 & 2 & -1 \\
    -1 & 2 & 2
    \end{array}\right] \cdot \frac{1}{3}\left[\begin{array}{cc}
    2 & -1 \\
    2 & 2 \\
    -1 & 2
    \end{array}\right] \\
    &=\frac{1}{9}\left[\begin{array}{ll}
    9 & 0 \\
    0 & 9
    \end{array}\right] \\
    &=\left[\begin{array}{ll}
    1 & 0 \\
    0 & 1
    \end{array}\right]\\
    Q Q^{T} &=\frac{1}{3}\left[\begin{array}{cc}
    2 & -1 \\
    2 & 2 \\
    -1 & 2
    \end{array}\right] \cdot \frac{1}{3}\left[\begin{array}{ccc}
    2 & 2 & -1 \\
    -1 & 2 & 2
    \end{array}\right] \\
    &=\frac{1}{9}\left[\begin{array}{ccc}
    5 & 2 & -4 \\
    2 & 8 & 2 \\
    -4 & 2 & 5
    \end{array}\right]
    \end{aligned}
    $$
    
    \item [13.] \textbf{Q.} What multiple of $\bm{a}=\left[\begin{array}{l}1 \\ 1\end{array}\right]$ should be subtracted from $\boldsymbol{b}=\left[\begin{array}{l}4 \\ 0\end{array}\right]$ to make the result $\boldsymbol{B}$ orthogonal to $\boldsymbol{a}$ ? Sketch a figure to show $\boldsymbol{a}, \boldsymbol{b}$, and $\boldsymbol{B}$. 
    
    \textbf{A.}
    
    $$
    \begin{aligned}
    \bm{a} &=\left[\begin{array}{l}
    1 \\
    1
    \end{array}\right] \\
    \bm{b} &=\left[\begin{array}{l}
    4 \\
    0
    \end{array}\right] \\
    \bm{a}^{T} \bm{b} &=\left[\begin{array}{ll}
    1 & 1
    \end{array}\right]\left[\begin{array}{l}
    4 \\
    0
    \end{array}\right] \\
    &=4 \\
    \bm{a}^{T} \bm{a} &=\left[\begin{array}{ll}
    1 & 1
    \end{array}\right]\left[\begin{array}{l}
    1 \\
    1
    \end{array}\right] \\
    &=2\\
    \bm{B} &=\bm{b}-\frac{\bm{a}^{T} \bm{b}}{\bm{a}^{T} \bm{a}} \\
    &=\left[\begin{array}{l}
    4 \\
    0
    \end{array}\right]-\frac{4}{2}\left[\begin{array}{l}
    1 \\
    1
    \end{array}\right] \\
    &=\left[\begin{array}{l}
    4 \\
    0
    \end{array}\right]-\left[\begin{array}{l}
    2 \\
    2
    \end{array}\right] \\
    &=\left[\begin{array}{c}
    2 \\
    -2
    \end{array}\right]\\
    \bm{a}^{T} \cdot \bm{B} &=\left[\begin{array}{ll}
    1 & 1
    \end{array}\right]\left[\begin{array}{c}
    2 \\
    -2
    \end{array}\right] \\
    &=0
    \end{aligned}
    $$
    
    \item [18.] \textbf{Q.} (Recommended) Find orthogonal vectors $\boldsymbol{A}, \boldsymbol{B}, \boldsymbol{C}$ by Gram-Schmidt from $\boldsymbol{a}, \boldsymbol{b}, \boldsymbol{c}$ :
    $$
    \boldsymbol{a}=(1,-1,0,0) \quad \boldsymbol{b}=(0,1,-1,0) \quad \boldsymbol{c}=(0,0,1,-1)
    $$
    $\boldsymbol{A}, \boldsymbol{B}, \boldsymbol{C}$ and $\boldsymbol{a}, \boldsymbol{b}, \boldsymbol{c}$ are bases for the vectors perpendicular to $\boldsymbol{d}=(1,1,1,1)$ 
    
    \textbf{A.} 
    
    $$
    \begin{aligned}
    \bm{A}&=\bm{a}\\
    \bm{B}&=\bm{b}-\frac{\bm{A}^{T} \bm{b}}{\bm{A}^{T} \bm{A}} \bm{A}\\
    \bm{C}&=\bm{c}-\frac{\bm{A}^{T} \bm{c}}{\bm{A}^{T} \bm{A}} \bm{A}-\frac{\bm{B}^{T} \bm{c}}{\bm{B}^{T} \bm{B}} \bm{B}\\
    \bm{A} &= (1,-1,0,0)\\
    \bm{A}^{T} \bm{b} &=\left[\begin{array}{llll}
    1 & -1 & 0 & 0
    \end{array}\right]\left[\begin{array}{c}
    0 \\
    1 \\
    -1 \\
    0
    \end{array}\right] \\
    &=-1 \\
    \bm{A}^{T} \bm{A} &=\left[\begin{array}{llll}
    1 & -1 & 0 & 0
    \end{array}\right]\left[\begin{array}{c}
    1 \\
    -1 \\
    0 \\
    0
    \end{array}\right] \\
    &=2 \\
    \bm{B} &=\bm{b}-\frac{\bm{A}^{T} \bm{b}}{\bm{A}^{T} \bm{A}} \bm{A} \\
    \bm{B} &=(0,1,-1,0)-\frac{(-1)}{2}(1,-1,0,0) \\
    &=(0,1,-1,0)+\left(\frac{1}{2},-\frac{1}{2}, 0,0\right) \\
    &=\left(\frac{1}{2}, \frac{1}{2},-1,0\right)\\
    \bm{B} \bm{A}^{T} &=\left(\frac{1}{2}, \frac{1}{2},-1,0\right)\left[\begin{array}{c}
    1 \\
    -1 \\
    0 \\
    0
    \end{array}\right] \\
    &=\frac{1}{2}-\frac{1}{2}+0+0 \\
    &=0
    \end{aligned}
    $$
    
    $$
    \begin{aligned}
    \bm{B}^{T} \bm{B} &=\left[\begin{array}{llll}
    \frac{1}{2} & \frac{1}{2} & -1 & 0
    \end{array}\right]\left[\begin{array}{c}
    \frac{1}{2} \\
    \frac{1}{2} \\
    -1 \\
    0
    \end{array}\right] \\
    &=\frac{1}{2} \cdot \frac{1}{2}+\frac{1}{2} \cdot \frac{1}{2}+(-1)(-1)+0 \cdot 0 \\
    &=\frac{3}{2}\\
    \bm{B}^{T} \bm{c} &=(0,0,1,-1)\left[\begin{array}{c}
    \frac{1}{2} \\
    \frac{1}{2} \\
    -1 \\
    0
    \end{array}\right] \\
    &=0+0-1+0 \\
    &=-1 \\
    \bm{A}^{T} \bm{c} &=(0,0,1,-1)\left[\begin{array}{c}
    1 \\
    -1 \\
    0 \\
    0
    \end{array}\right] \\
    &=0\\
    \bm{C} &=\bm{c}-\frac{\bm{A}^{T} c}{\bm{A}^{T} \bm{A}} \bm{A}-\frac{\bm{B}^{T} c}{\bm{B}^{T} \bm{B}} \bm{B} \\
    \bm{C} &=(0,0,1,-1)-0(1,-1,0,0)-\frac{(-1)}{3 / 2}\left(\frac{1}{2}, \frac{1}{2},-1,0\right) \\
    &=(0,0,1,-1)+\frac{2}{3}\left(\frac{1}{2}, \frac{1}{2},-1,0\right) \\
    &=\left(\frac{1}{3}, \frac{1}{3}, \frac{1}{3},-1\right)\\
    \bm{A} \cdot \bm{C} &=(1,-1,0,0) \cdot\left(\frac{1}{3}, \frac{1}{3}, \frac{1}{3},-1\right) \\
    &=\frac{1}{3}-\frac{1}{3} \\
    &=0\\
    \bm{B} \cdot \bm{C} &=\left(\frac{1}{2}, \frac{1}{2},-1,0\right) \cdot\left(\frac{1}{3}, \frac{1}{3}, \frac{1}{3},-1\right) \\
    &=\frac{1}{6}+\frac{1}{6}-\frac{1}{3} \\
    &=0
    \end{aligned}
    $$
    
    $\bm{C}$ is perpendicular to $\bm{A}$ and $\bm{B}$. $\bm{a}$, $\bm{b}$, $\bm{c}$ are linearly independent. $\bm{a} \cdot (1,1,1,1) = \bm{b} \cdot (1,1,1,1)= \bm{c} \cdot (1,1,1,1)=0$, therefore $\bm{a}$, $\bm{b}$, $\bm{c}$, are the null space of $(1,1,1,1)$. Therefore, $\bm{a}, \bm{b}, \bm{c}$ are bases for the space of vectors perpendicular to $(1,1,1,1)$. Since, $\bm{A}, \bm{B}, \bm{C}$ are orthonormal, they are linearly independent. Therefore, they must span the space of vectors perpendicular to $(1,1,1,1)$. Therefore, $\bm{A}, \bm{B}, \bm{C}$ are bases for the space of vectors perpendicular to $(1,1,1,1)$.
    
\end{enumerate}

\subsection{Section 5.1}
\begin{enumerate}
    \item [1.] \textbf{Q.} If a 4 by 4 matrix has $\operatorname{det} A=\frac{1}{2}$, find $\operatorname{det}(2 A)$ and $\operatorname{det}(-A)$ and $\operatorname{det}\left(A^{2}\right)$ and $\operatorname{det}\left(A^{-1}\right)$ 
    
    \textbf{A.}
    
    $$
    \begin{aligned}
    |2 A| &=2^{4}|A| \\
    &=2^{4} \times \frac{1}{2} \\
    &=8 \\
    |-A| &= (-1)^{4}|A| \\
    & = (-1)^{4} \times \frac{1}{2} \\
    & = \frac{1}{2} \\
    \left|A^{2}\right| &= |A||A| \\
    &= \frac{1}{2} \frac{1}{2} \\
    &= \frac{1}{4}\\
    \left|A^{-1}\right| &=\frac{1}{|A|} \\
    &=2
    \end{aligned}
    $$
    
    \item [10.] \textbf{Q.} If the entries in every row of $A$ add to zero, solve $A x=0$ to prove $\operatorname{det} A=0$. If those entries add to one, show that $\operatorname{det}(A-I)=0$. Does this mean $\operatorname{det} A=1$ ? 
    
    \textbf{A.}
    
    $$
    \begin{aligned}
    A&=\left[\begin{array}{cccc}
    a_{11} & a_{12} & \ldots & a_{1 n} \\
    a_{21} & a_{22} & \ldots & a_{2 n} \\
    \vdots & \vdots & \ldots & \vdots \\
    a_{n 1} & a_{n 2} & \ldots & a_{n n}
    \end{array}\right]\\
    a_{11}+a_{12}+\cdots+a_{1 n}&=0 \\
    a_{21}+a_{22}+\cdots+a_{2 n}&=0 \\
    \vdots & \\
    a_{n 1}+a_{n 2}+\cdots+a_{n n}&=0\\
    \left[\begin{array}{cccc}
    a_{11} & a_{12} & \ldots & a_{1 n} \\
    a_{21} & a_{22} & \ldots & a_{2 n} \\
    \vdots & \vdots & \ldots & \vdots \\
    a_{n 1} & a_{n 2} & \ldots & a_{n n}
    \end{array}\right]\left[\begin{array}{c}
    1 \\
    1 \\
    \vdots \\
    1
    \end{array}\right]&=\left[\begin{array}{c}
    0 \\
    0 \\
    \vdots \\
    0
    \end{array}\right]
    \end{aligned}
    $$
    
    The null space of $A$ is the solution space of the system $A \bm{x}=\bm{0}$. $A \bm{x}=\bm{0}$ has the only solution $x=(1,1, \ldots, 1)$ and therefore the null space of $A$ is nonzero. The null space of $A$ is nonzero, so $A$ is not invertible. $A$ is not invertible, so $\operatorname{det} A=\bm{0}$. 
    
    If those entries add to one
    
    $$
    \begin{aligned}
    a_{11}+a_{12}+\cdots+a_{1 n} &=1 \\
    a_{21}+a_{22}+\cdots+a_{2 n} &=1 \\
    \vdots & \\
    a_{n 1}+a_{n 2}+\cdots+a_{n n} &=1\\
    \left(a_{11}-1\right) \cdot 1+a_{12} \cdot 1+\cdots+a_{1 n} \cdot 1&=0 \\
    a_{21} \cdot 1+\left(a_{22}-1\right) \cdot 1+\cdots+a_{2 n} \cdot 1&=0 \\
    \vdots &\\
    a_{n 1} \cdot 1+a_{n 2} \cdot 1+\cdots+\left(a_{n n}-1\right) \cdot 1&=0 \\
    \left[\begin{array}{cccc}
    a_{11}-1 & a_{12} & \ldots & a_{1 n} \\
    a_{21} & a_{22}-1 & \ldots & a_{2 n} \\
    \vdots & \vdots & \ldots & \vdots \\
    a_{n 1} & a_{n 2} & \ldots & a_{n n}-1
    \end{array}\right]\left[\begin{array}{c}
    1 \\
    1 \\
    \vdots \\
    1
    \end{array}\right]&=\left[\begin{array}{c}
    0 \\
    0 \\
    \vdots \\
    0
    \end{array}\right]
    \end{aligned}
    $$
    
    $(A-I) \bm{x}=\bm{0}$ has the only solution $x=(1,1, \ldots, 1)$, therefore, the null space of $(A-I)$ is nonzero. The null space of $(A-I)$ is nonzero, so $(A-I)$ is not invertible. $(A-I)$ is not invertible, so $\operatorname{det}(A-I)=\bm{0}$.
    
    \item [15.] \textbf{Q.} Use row operations to simplify and compute these determinants:
    $$
    \operatorname{det}\left[\begin{array}{lll}
    101 & 201 & 301 \\
    102 & 202 & 302 \\
    103 & 203 & 303
    \end{array}\right] \quad \text { and } \operatorname{det}\left[\begin{array}{ccc}
    1 & t & t^{2} \\
    t & 1 & t \\
    t^{2} & t & 1
    \end{array}\right]
    $$
    
    \textbf{A.}
    
    $$
    \begin{aligned}
    &\left[\begin{array}{lll}
    101 & 201 & 301 \\
    102 & 202 & 302 \\
    103 & 203 & 303
    \end{array}\right]\\
    R_2 - R_1 &\rightarrow R_2\\
    &\left[\begin{array}{ccc}
    101 & 201 & 301 \\
    1 & 1 & 1 \\
    103 & 203 & 303
    \end{array}\right]\\
    R_3 - R_1 &\rightarrow R_3\\
    &\left[\begin{array}{ccc}
    101 & 201 & 301 \\
    1 & 1 & 1 \\
    2 & 2 & 2
    \end{array}\right]\\
    \frac{1}{2}R_3 &\rightarrow R_3\\
    &\left[\begin{array}{ccc}
    101 & 201 & 301 \\
    1 & 1 & 1 \\
    1 & 1 & 1
    \end{array}\right]\\
    \end{aligned}
    $$
    
    $R_2$ and $R_3$ are identical, therefore $\operatorname{det}\left[\begin{array}{lll}
    101 & 201 & 301 \\
    102 & 202 & 302 \\
    103 & 203 & 303
    \end{array}\right] = 0$.
    
    $$
    \begin{aligned}
    \left|\begin{array}{lll}
    1 & t & t^{2} \\
    t & 1 & t \\
    t^{2} & t & 1
    \end{array}\right|&\\
    -R_1 + R_3 &\rightarrow R_3\\
    -R_1 + R_2 &\rightarrow R_2\\
    \left|\begin{array}{ccc}
    1 & t & t^{2} \\
    t-1 & -(t-1) & -t(t-1) \\
    t^{2}-1 & 0 & -\left(t^{2}-1\right)
    \end{array}\right|& \\
    \left(t^{2}-1\right)(t-1)\left|\begin{array}{ccc}
    1 & t & t^{2} \\
    1 & -1 & -t \\
    1 & 0 & -1
    \end{array}\right|&\\
    &=\left(t^{2}-1\right)(t-1)\left[1(1)-t(-1+t)+t^{2}(0+1)\right] \\
    &=\left(t^{2}-1\right)(t-1)\left(1+t-t^{2}+t^{2}\right) \\
    &=\left(t^{2}-1\right)(t-1)(t+1) \\
    &=\left(t^{2}-1\right)\left(t^{2}-1\right) \\
    &=t^{4}-2 t^{2}+1
    \end{aligned}
    $$
    
    \item [23.] \textbf{Q.} From $A=\left[\begin{array}{ll}4 & 1 \\ 2 & 3\end{array}\right]$ find $A^{2}$ and $A^{-1}$ and $A-\lambda I$ and their determinants. Which two numbers $\lambda$ lead to $\operatorname{det}(A-\lambda I)=0$ ? 
    
    \textbf{A.}
    
    $$
    \begin{aligned}
    |A| &=4 \times 3-2 \times 1 \\
    &=12-2 \\
    &=10 \\
    A^{2} &=A \cdot A \\
    &=\left[\begin{array}{ll}
    4 & 1 \\
    2 & 3
    \end{array}\right]\left[\begin{array}{ll}
    4 & 1 \\
    2 & 3
    \end{array}\right] \\
    &=\left[\begin{array}{cc}
    18 & 7 \\
    14 & 11
    \end{array}\right]\\
    \operatorname{det} A^{2} &=18 \times 11-14 \times 7 \\
    &=198-98 \\
    &=100 \\
    A^{-1} &= \frac{1}{\operatorname{det} A}\left[\begin{array}{cc}
    3 & -1 \\
    -2 & 4
    \end{array}\right] \\
    & = \frac{1}{10}\left[\begin{array}{cc}
    3 & -1 \\
    -2 & 4
    \end{array}\right]\\
    \operatorname{det} A^{-1} &=\frac{1}{\operatorname{det} A} \\
    &=\frac{1}{10} \\
    A-\lambda I &=\left[\begin{array}{ll}
    4 & 1 \\
    2 & 3
    \end{array}\right]-\lambda\left[\begin{array}{ll}
    1 & 0 \\
    0 & 1
    \end{array}\right] \\
    &=\left[\begin{array}{cc}
    4-\lambda & 1 \\
    2 & 3-\lambda
    \end{array}\right] \\
    \operatorname{det}(A-\lambda I) &=0 \\
    (4-\lambda)(3-\lambda)-2 &=0 \\
    \lambda^{2}-7 \lambda+10 &=0 \\
    (\lambda-2)(\lambda-5) &=0 \\
    \lambda&=2,5
    \end{aligned}
    $$
    
    \item [27.] \textbf{Q.} Compute the determinants of these matrices by row operations:
    $$
    A=\left[\begin{array}{lll}
    0 & a & 0 \\
    0 & 0 & b \\
    c & 0 & 0
    \end{array}\right] \text { and } B=\left[\begin{array}{cccc}
    0 & a & 0 & 0 \\
    0 & 0 & b & 0 \\
    0 & 0 & 0 & c \\
    d & 0 & 0 & 0
    \end{array}\right] \text { and } C=\left[\begin{array}{ccc}
    a & a & a \\
    a & b & b \\
    a & b & c
    \end{array}\right] \text {. }
    $$
    
    \textbf{A.} 
    
    $$
    \begin{aligned}
    |A|&=\left|\begin{array}{lll}
    0 & a & 0 \\
    0 & 0 & b \\
    c & 0 & 0
    \end{array}\right|&\\
    R_1 &\leftrightarrow R_2 \\
    R_1 &\leftrightarrow R_3 \\
    &=\left|\begin{array}{lll}
    c & a & 0 \\
    0 & a & 0 \\
    0 & 0 & b
    \end{array}\right|\\
    & = cab\\
    |B| & =\left|\begin{array}{llll}
    0 & a & 0 & 0 \\
    0 & 0 & b & 0 \\
    0 & 0 & 0 & c \\
    d & 0 & 0 & 0
    \end{array}\right|\\
    R_4 &\leftrightarrow R_3 \\
    R_3 &\leftrightarrow R_2 \\
    R_2 &\leftrightarrow R_1 \\
    & =\left|\begin{array}{llll}
    d & 0 & 0 & 0 \\
    0 & a & 0 & 0 \\
    0 & 0 & b & 0 \\
    0 & 0 & 0 & c
    \end{array}\right|\\
    & = dabc\\
    |C|&=\left|\begin{array}{lll}
    a & a & a \\
    a & b & b \\
    a & b & c
    \end{array}\right|\\
    R_2 - R-1 &\rightarrow R_2 \\
    R_2 - R_3 &\rightarrow R_3 \\
    &=\left|\begin{array}{ccc}
    a & a & a \\
    0 & b-a & b-a \\
    0 & b-a & c-a
    \end{array}\right|\\
    &=a(b-a)(c-a)-a(b-a)(b-a) \\
    &=a(b-a)((c-a)-(b-a)) \\
    &=a(b-a)(c-b)
    \end{aligned}
    $$
    
\end{enumerate}

\subsection{Section 5.2}
\begin{enumerate}
    \item [2.] \textbf{Q.} Compute the determinants of $A, B, C, D$. Are their columns independent?
    $$
    A=\left[\begin{array}{lll}
    1 & 1 & 0 \\
    1 & 0 & 1 \\
    0 & 1 & 1
    \end{array}\right] \quad B=\left[\begin{array}{lll}
    1 & 2 & 3 \\
    4 & 5 & 6 \\
    7 & 8 & 9
    \end{array}\right] \quad C=\left[\begin{array}{ll}
    A & 0 \\
    0 & A
    \end{array}\right] \quad D=\left[\begin{array}{cc}
    A & 0 \\
    0 & B
    \end{array}\right]
    $$
    
    \textbf{A.}
    
    Column permutation $P=(\alpha, \beta \ldots \omega)$.
    
    $$
    \begin{aligned}
    A&=\left[\begin{array}{lll}
    1 & 1 & 0 \\
    1 & 0 & 1 \\
    0 & 1 & 1
    \end{array}\right]\\
    a_{11}&=1, a_{12}=1, a_{13}=0 \\
    a_{21}&=1, a_{22}=0, a_{23}=1 \\
    a_{31}&=0, a_{32}=1, a_{33}=1 \\
    |A|&=\sum \operatorname{det}(P) a_{1 \alpha} a_{2 \beta} \ldots a_{n \omega}\\
    &=a_{11} a_{22} a_{33}+a_{12} a_{23} a_{31}+a_{13} a_{21} a_{32}-a_{11} a_{23} a_{32}-a_{12} a_{21} a_{33}-a_{13} a_{22} a_{31} \\
    &=1 \times 0 \times 1+1 \times 1 \times 0+0 \times 1 \times 0-1 \times 1 \times 1-1 \times 1 \times 1-0 \times 0 \times 0 \\
    &=-1-1 \\
    &=-2
    \end{aligned}
    $$
    
    Determinant is non-zero, therefore columns of $A$ are linearly independent.
    
    $$
    \begin{aligned}
    B&=\left[\begin{array}{lll}
    1 & 2 & 3 \\
    4 & 5 & 6 \\
    7 & 8 & 9
    \end{array}\right]\\
    a_{11}&=1, a_{12}=2, a_{13}=3 \\
    a_{21}&=4, a_{22}=5, a_{23}=6 \\
    a_{31}&=7, a_{32}=8, a_{33}=9 \\
    |B|&=\sum \operatorname{det}(P) a_{1 \alpha} a_{2 \beta} \ldots a_{n \omega}\\
    &=a_{11} a_{22} a_{33}+a_{12} a_{23} a_{31}+a_{13} a_{21} a_{32}-a_{11} a_{23} a_{32}-a_{12} a_{21} a_{33}-a_{13} a_{22} a_{31} \\
    &=1 \times 5 \times 9+2 \times 6 \times 7+3 \times 4 \times 8-1 \times 6 \times 8-2 \times 4 \times 9-3 \times 5 \times 7 \\
    &=45+84+96-48-72-105 \\
    &=129+48-177 \\
    &=0
    \end{aligned}
    $$
    
    Determinant of matrix is 0, therefore the columns are linearly dependent.
    
    $$
    \begin{aligned}
    C &=\left[\begin{array}{ll}
    A & 0 \\
    0 & A
    \end{array}\right] \\
    |C| &=A^{2} \\
    &=\left[\begin{array}{lll}
    1 & 1 & 0 \\
    1 & 0 & 1 \\
    0 & 1 & 1
    \end{array}\right]\left[\begin{array}{lll}
    1 & 1 & 0 \\
    1 & 0 & 1 \\
    0 & 1 & 1
    \end{array}\right] \\
    &=\left[\begin{array}{lll}
    2 & 1 & 1 \\
    1 & 2 & 1 \\
    1 & 1 & 2
    \end{array}\right]\\
    a_{11} &=2, a_{12}=1, a_{13}=1 \\
    a_{21} &=1, a_{22}=2, a_{23}=1 \\
    a_{31} &=1, a_{32}=1, a_{33}=2 \\
    |C|&=\sum \operatorname{det}(P) a_{1 \alpha} a_{2 \beta} \ldots a_{n \omega} \\
    &=a_{11} a_{22} a_{33}+a_{12} a_{23} a_{31}+a_{13} a_{21} a_{32}-a_{11} a_{23} a_{32}-a_{12} a_{21} a_{33}-a_{13} a_{22} a_{31} \\
    &=2 \times 2 \times 2+1 \times 1 \times 1+1 \times 1 \times 1-2 \times 1 \times 1-1 \times 1 \times 2-1 \times 2 \times 1 \\
    &=8+1+1-2-2-2 \\
    &=4
    \end{aligned}
    $$
    
    Determinant is non-zero, the columns are linearly independent.
    
    $$
    \begin{aligned}
    D &=\left[\begin{array}{ll}
    A & 0 \\
    0 & B
    \end{array}\right] \\
    |D| &=A B \\
    &=\left[\begin{array}{lll}
    1 & 1 & 0 \\
    1 & 0 & 1 \\
    0 & 1 & 1
    \end{array}\right]\left[\begin{array}{lll}
    1 & 2 & 3 \\
    4 & 5 & 6 \\
    7 & 8 & 9
    \end{array}\right] \\
    &=\left[\begin{array}{ccc}
    5 & 7 & 9 \\
    8 & 10 & 12 \\
    11 & 13 & 15
    \end{array}\right]\\
    a_{11}&=5, a_{12}=7, a_{13}=9 \\
    a_{21}&=8, a_{22}=10, a_{23}=12 \\
    a_{31}&=11, a_{32}=13, a_{33}=15 \\
    &=a_{11} a_{22} a_{33}+a_{12} a_{23} a_{31}+a_{13} a_{21} a_{32}-a_{11} a_{23} a_{32}-a_{12} a_{21} a_{33}-a_{13} a_{22} a_{31} \\
    &=5 \times 10 \times 15+7 \times 12 \times 11+9 \times 8 \times 13-5 \times 12 \times 13-7 \times 8 \times 15-9 \times 10 \times 11 \\
    &=750+924+936-780-840-990 \\
    & = 0
    \end{aligned}
    $$
    
    The determinant is 0, the columns are linearly dependent.
    
    \item [12.] \textbf{Q.} Find the cofactor matrix $C$ and multiply $A$ times $C^{\mathrm{T}}$. Compare $A C^{\mathrm{T}}$ with $A^{-1}$ :
    $$
    A=\left[\begin{array}{rrr}
    2 & -1 & 0 \\
    -1 & 2 & -1 \\
    0 & -1 & 2
    \end{array}\right] \quad A^{-1}=\frac{1}{4}\left[\begin{array}{lll}
    3 & 2 & 1 \\
    2 & 4 & 2 \\
    1 & 2 & 3
    \end{array}\right]
    $$
    
    \textbf{A.}
    
    $$
    \begin{aligned}
    C_{i j} &= (-1)^{i+j} \operatorname{det} M_{i j}\\
    C_{11} &=(-1)^{1+1}\left|\begin{array}{rr}
    2 & -1 \\
    -1 & 2
    \end{array}\right|, \\
    &=+1(4-1) \\
    &=3 \\
    C_{12} &=(-1)^{1+2}\left|\begin{array}{rr}
    -1 & -1 \\
    0 & 2
    \end{array}\right|, \\
    &=-(-2-0) \\
    &=2 \\
    C_{13} &=(-1)^{1+3}\left|\begin{array}{rr}
    -1 & 2 \\
    0 & -1
    \end{array}\right| \\
    &=1-0 \\
    &=1 \\
    C_{21} &=(-1)^{2+1}\left|\begin{array}{ll}
    -1 & 0 \\
    -1 & 2
    \end{array}\right|, \\
    &=-(-2-0) \\
    &=2 \\
    C_{22} &=(-1)^{2+2}\left|\begin{array}{ll}
    2 & 0 \\
    0 & 2
    \end{array}\right|, \\
    &=(4-0) \\
    &=4 \\
    C_{23} &=(-1)^{2+3}\left|\begin{array}{rr}
    2 & -1 \\
    0 & -1
    \end{array}\right| \\
    &=-(-2-0) \\
    &=2 \\
    C_{31} &=(-1)^{3+1}\left|\begin{array}{cc}
    -1 & 0 \\
    2 & -1
    \end{array}\right| \\
    &=(1-0) \\
    &=1 \\
    C_{32} &=(-1)^{3+2}\left|\begin{array}{cc}
    2 & 0 \\
    -1 & -1
    \end{array}\right| \\
    &=-(-2-0) \\
    &=2 \\
    C_{33} &=(-1)^{3+3}\left|\begin{array}{cc}
    2 & -1 \\
    -1 & 2
    \end{array}\right| \\
    &=(4-1) \\
    &=3 \\
    C&=\left[\begin{array}{lll}
    3 & 2 & 1 \\
    2 & 4 & 2 \\
    1 & 2 & 3
    \end{array}\right]
    \end{aligned}
    $$
    
    $$
    \begin{aligned}
    C^{T} &=\left[\begin{array}{lll}
    3 & 2 & 1 \\
    2 & 4 & 2 \\
    1 & 2 & 3
    \end{array}\right]^{T} \\
    &=\left[\begin{array}{lll}
    3 & 2 & 1 \\
    2 & 4 & 2 \\
    1 & 2 & 3
    \end{array}\right] \\
    \operatorname{det} A &=\left|\begin{array}{ccc}
    2 & -1 & 0 \\
    -1 & 2 & -1 \\
    0 & -1 & 2
    \end{array}\right| \\
    &=2(4-1)-(-1)(-2-0)+0 \\
    &=2(3)-2 \\
    &=6-2 \\
    &=4 \\
    A C^{T} &=\left[\begin{array}{ccc}
    2 & -1 & 0 \\
    -1 & 2 & -1 \\
    0 & -1 & 2
    \end{array}\right]\left[\begin{array}{ccc}
    3 & 2 & 1 \\
    2 & 4 & 2 \\
    1 & 2 & 3
    \end{array}\right] \\
    &=\left[\begin{array}{ccc}
    6-2+0 & 4-4+0 & 2-2+0 \\
    -3+4-1 & -2+8-2 & -1+4-3 \\
    0-2+2 & 0-4+4 & 0-2+6
    \end{array}\right] \\
    &=\left[\begin{array}{ccc}
    4 & 0 & 0 \\
    0 & 4 & 0 \\
    0 & 0 & 4
    \end{array}\right] \\
    &=4\left[\begin{array}{ccc}
    1 & 0 & 0 \\
    0 & 1 & 0 \\
    0 & 0 & 1
    \end{array}\right] \\
    &=4 I \\
    A^{-1} &=\frac{1}{4} C^{T} \\
    &=\left(\frac{1}{\operatorname{det} A}\right) C^{T}
    \end{aligned}
    $$
    
    \item [13.] The $n$ by $n$ determinant $C_{n}$ has 1 's above and below the main diagonal:
    $$
    C_{1}=|0| \quad C_{2}=\left|\begin{array}{ll}
    0 & 1 \\
    1 & 0
    \end{array}\right| \quad C_{3}=\left|\begin{array}{lll}
    0 & 1 & 0 \\
    1 & 0 & 1 \\
    0 & 1 & 0
    \end{array}\right| \quad C_{4}=\left|\begin{array}{llll}
    0 & 1 & 0 & 0 \\
    1 & 0 & 1 & 0 \\
    0 & 1 & 0 & 1 \\
    0 & 0 & 1 & 0
    \end{array}\right|
    $$
    \begin{enumerate}
        \item [a.] \textbf{Q.} What are these determinants $C_{1}, C_{2}, C_{3}, C_{4} ?$ 
        
        \textbf{A.}
        $C_{1}$
        $$
        \begin{aligned}
        C_{1} &=|0| \\
        &=0
        \end{aligned}
        $$
        
        $C_{2}$
        $$
        \begin{aligned}
        C_{11} &=(-1)^{1+1} \operatorname{det} M_{11} \\
        &=(-1)^{2} \operatorname{det}[0] \\
        &=1(0) \\
        &=0 \\
        C_{12} &=(-1)^{1+2} \operatorname{det} M_{12} \\
        &=(-1)^{3} \operatorname{det}[1] \\
        &=(-1)(1) \\
        &=-1 \\
        C_{2}&=a_{11} C_{11}+a_{12} C_{12} \\
        &=(0)(0)+(1)(-1) \\
        &=0-1 \\
        &=-1
        \end{aligned}
        $$
        
        $C_{3}$
        $$
        \begin{aligned}
        C_{11} &=(-1)^{1+1} \operatorname{det} M_{11} \\
        &=(-1)^{2} \operatorname{det}\left[\begin{array}{ll}
        0 & 1 \\
        1 & 0
        \end{array}\right] \\
        &=1(0-1) \\
        &=-1 \\
        C_{12} &=(-1)^{1+2} \operatorname{det} M_{12} \\
        &=(-1)^{3} \operatorname{det}\left[\begin{array}{ll}
        1 & 1 \\
        0 & 0
        \end{array}\right] \\
        &=(-1)(0-0) \\
        &=0 \\
        C_{13} &=(-1)^{1+3} \operatorname{det} M_{13} \\
        &=(-1)^{4} \operatorname{det}\left[\begin{array}{ll}
        1 & 0 \\
        0 & 1
        \end{array}\right] \\
        &=(1)(1-0) \\
        &=1 \\
        C_{3} &= a_{11} C_{11}+a_{12} C_{12}+a_{13} C_{13} \\
        &=(0)(-1)+(1)(0)+(0)(1) \\
        &=0 +0+0 \\
        & =0 \\
        \end{aligned}
        $$
        
        $C_{4}$
        
        $$
        \begin{aligned}
        C_{11} &= (-1)^{1+1} \operatorname{det} M_{11}\\
        &=(-1)^{2} \operatorname{det}\left[\begin{array}{lll}
        0 & 1 & 0 \\
        1 & 0 & 1 \\
        0 & 1 & 0
        \end{array}\right]\\
        &=1(0)\\
        &=0\\
        C_{12} &= (-1)^{1+2} \operatorname{det} M_{12}\\
        &=(-1)^{3} \operatorname{det}\left[\begin{array}{lll}
        1 & 1 & 0 \\
        0 & 0 & 1 \\
        0 & 1 & 0
        \end{array}\right]\\
        &=(-1)(-1)\\
        &=1\\
        C_{13}&=(-1)^{1+3} \operatorname{det} M_{13}\\
        &=(-1)^{4} \operatorname{det}\left[\begin{array}{lll}
        1 & 0 & 0 \\
        0 & 1 & 1 \\
        0 & 0 & 0
        \end{array}\right]\\
        &=(1)(0)\\
        &=0\\
        C_{14}&=(-1)^{1+4} \operatorname{det} M_{14}\\
        &=(-1)^{5} \operatorname{det}\left[\begin{array}{lll}
        1 & 0 & 1 \\
        0 & 1 & 0 \\
        0 & 0 & 1
        \end{array}\right]\\
        &=(-1)(1)\\
        &=-1\\
        C_{4}&=a_{11} C_{11}+a_{12} C_{12}+a_{13} C_{13}+a_{14} C_{14}\\
        &=(0)(0)+(1)(1)+(0)(0)+(0)(-1)\\
        &=0+1+0+0\\
        &=1
        \end{aligned}
        $$
        
        \item [b.] \textbf{Q.} By cofactors find the relation between $C_{n}$ and $C_{n-1}$ and $C_{n-2}$. Find $C_{10}$. 
        
        \textbf{A.}
        
        $$
        \begin{aligned}
        C_{1}&=0\\
        C_{2}&=-1\\
        C_{3} &=0 \\
        &=-0 \\
        &=-C_{1} \\
        &=-C_{3-2} \\
        C_{4} &=1 \\
        &=-(-1) \\
        &=-C_{2} \\
        &=-C_{4-2}\\
        C_{n}&=-C_{n-2}\\
        C_{10} &=-C_{10-2} \\
        &=-C_{8} \\
        &=-\left(-C_{8-2}\right) \\
        &=C_{6} \\
        &=-C_{6-2} \\
        &=-C_{4} \\
        &=-\left(-C_{4-2}\right) \\
        &=-\left(-C_{2}\right) \\
        &= C_{2} \\
        &=-1
        \end{aligned}
        $$
        
    \end{enumerate}
    
    \item [20.] \textbf{Q.} Find $G_{2}$ and $G_{3}$ and then by row operations $G_{4}$. Can you predict $G_{n}$ ?
    $$
    G_{2}=\left|\begin{array}{ll}
    0 & 1 \\
    1 & 0
    \end{array}\right| \quad G_{3}=\left|\begin{array}{lll}
    0 & 1 & 1 \\
    1 & 0 & 1 \\
    1 & 1 & 0
    \end{array}\right| \quad G_{4}=\left|\begin{array}{llll}
    0 & 1 & 1 & 1 \\
    1 & 0 & 1 & 1 \\
    1 & 1 & 0 & 1 \\
    1 & 1 & 1 & 0
    \end{array}\right|
    $$
    
    \textbf{A.}
    
    $G_{2}$
    $$
    \begin{aligned}
    G_{2}&=a_{i 1} C_{i 1}+a_{i 2} C_{i 2}+\cdots+a_{i n} C_{i n}\\
    C_{i j}&=(-1)^{i+j} \operatorname{det} M_{i j}\\
    C_{11}&=0 \\
    C_{12}&=-1 \\
    C_{21}&=-1 \\
    C_{22}&=0 \\
    G_{2}&=-1 \\
    \end{aligned}
    $$
    
    $G_{3}$
    $$
    \begin{aligned}
    G_{3}&=a_{i 1} C_{i 1}+a_{i 2} C_{i 2}+\cdots+a_{i n} C_{i n}\\
    C_{31} &=\left|\begin{array}{ll}
    1 & 1 \\
    0 & 1
    \end{array}\right| \\
    &=1 \\
    C_{32} &=\left|\begin{array}{ll}
    0 & 1 \\
    1 & 1
    \end{array}\right| \\
    &=-1 \\
    G_{3} &=1 C_{31}-1 C_{32} \\
    &=1-1 \times(-1) \\
    &=2
    \end{aligned}
    $$
    
    $G_{4}$
    $$
    \begin{aligned}
    G_{4}&=\left|\begin{array}{llll}
    0 & 1 & 1 & 1 \\
    1 & 0 & 1 & 1 \\
    1 & 1 & 0 & 1 \\
    1 & 1 & 1 & 0
    \end{array}\right|\\
    R_1 &\leftrightarrow R_3\\
    &=\left|\begin{array}{llll}
    1 & 0 & 1 & 1 \\
    0 & 1 & 1 & 1 \\
    1 & 1 & 0 & 1 \\
    1 & 1 & 1 & 0
    \end{array}\right|\\
    R_4 - R_3 &\rightarrow R_4\\
    &=\left|\begin{array}{cccc}
    1 & 0 & 1 & 1 \\
    0 & 1 & 1 & 1 \\
    1 & 1 & 0 & 1 \\
    0 & 0 & 1 & -1
    \end{array}\right|\\
    R_4 &\leftrightarrow R_3\\
    &=\left|\begin{array}{cccc}
    1 & 0 & 1 & 1 \\
    0 & 1 & 1 & 1 \\
    0 & 0 & 1 & -1 \\
    1 & 1 & 0 & 1
    \end{array}\right|\\
    R_4 - R_1 &\rightarrow R_4\\
    &=\left|\begin{array}{cccc}
    1 & 0 & 1 & 1 \\
    0 & 1 & 1 & 1 \\
    0 & 0 & 1 & -1 \\
    0 & 1 & -1 & 0
    \end{array}\right|\\
    R_4 - R_2 &\rightarrow R_4\\
    &=\left|\begin{array}{cccc}
    1 & 0 & 1 & 1 \\
    0 & 1 & 1 & 1 \\
    0 & 0 & 1 & -1 \\
    0 & 0 & -2 & -1
    \end{array}\right|\\
    R_4 + 2R_3 &\rightarrow R_4\\
    &=\left|\begin{array}{cccc}
    1 & 0 & 1 & 1 \\
    0 & 1 & 1 & 1 \\
    0 & 0 & 1 & -1 \\
    0 & 0 & 0 & -3
    \end{array}\right|\\
    G_{4} &=1 \times 1 \times 1 \times(-3) \\
    &=-3\\
    G_{n}&=(-1)^{n-1}(n-1)
    \end{aligned}
    $$
    
\end{enumerate}

\subsection{Section 5.3}
\begin{enumerate}
    \item [2.] \textbf{Q.} (only a) Use Cramer's Rule to solve for $y$ (only). Call the 3 by 3 determinant $D$ :
    \begin{enumerate}
        \item [a.] 
            $$
            \begin{aligned}
            a x+b y &= 1\\
            c x+d y &=0
            \end{aligned}
            $$
            
            \textbf{A.}
            
            $$
            \begin{aligned}
            A&=\left[\begin{array}{ll}
            a & b \\
            c & d
            \end{array}\right], \mathbf{x}=\left[\begin{array}{l}
            x \\
            y
            \end{array}\right], \mathbf{b}=\left[\begin{array}{l}
            1 \\
            0
            \end{array}\right]\\
            \operatorname{det} A &=\left|\begin{array}{ll}
            a & b \\
            c & d
            \end{array}\right| \\
            &=a d-b c\\
            B_{2}&=\left[\begin{array}{ll}
            a & 1 \\
            c & 0
            \end{array}\right]\\
            \operatorname{det} B_{2} &=\left|\begin{array}{ll}
            a & 1 \\
            c & 0
            \end{array}\right| \\
            &=(a)(0)-(c)(1) \\
            &=0-c \\
            &=-c \\
            y &=\frac{\operatorname{det} B_{2}}{\operatorname{det} A} \\
            &=\frac{-c}{a d-b c} \\
            &=-\frac{c}{a d-b c}
            \end{aligned}
            $$
            
    \end{enumerate}
    
    \item [4.] Quick proof of Cramer's rule. The determinant is a linear function of column 1 . It is zero if two columns are equal. When $b=A x=x_{1} a_{1}+x_{2} a_{2}+x_{3} a_{3}$ goes into the first column of $A$, the determinant of this matrix $B_{1}$ is $\left|\begin{array}{lll}b & a_{2} & a_{3}\end{array}\right|=\left|x_{1} a_{1}+x_{2} a_{2}+x_{3} a_{3} \quad a_{2} \quad a_{3}\right|=x_{1}\left|a_{1} \quad a_{2} \quad a_{3}\right|=x_{1} \operatorname{det} A$
    \begin{enumerate}
        \item [a.] \textbf{Q.} What formula for $x_{1}$ comes from left side = right side? 
        
        \textbf{A.} Cramer's Rule states that: If $\operatorname{det} A \neq 0$, equation $A x=b$ is solved by determinants,
        
        $$
        \begin{aligned}
        x_{1}=& \frac{\operatorname{det} B_{1}}{\operatorname{det} A} \\
        x_{2}=& \frac{\operatorname{det} B_{2}}{\operatorname{det} A} \\
        x_{3}=& \frac{\operatorname{det} B_{3}}{\operatorname{det} A} \\
        & \vdots \\
        x_{n}=& \frac{\operatorname{det} B_{n}}{\operatorname{det} A}
        \end{aligned}
        $$
        
        The matrix $B_{\mathrm{j}}$ has the $j^{\text {th }}$ column of $A$ replaced by the vector $\boldsymbol{b}$.
        
        $$
        x_{1}=\frac{\left[\begin{array}{lll}
        b & a_{2} & a_{3}
        \end{array}\right]}{\operatorname{det} A}
        $$
        
        \item [b.] \textbf{Q.} What steps lead to the middle equation? 
        
        \textbf{A.}
        
        $$
        \begin{aligned}
        x_{1}\left|a_{1} \quad a_{2} \quad a_{3}\right|+x_{2}\left|a_{2} \quad a_{2} \quad a_{3}\right|+x_{1}\left|a_{3} \quad a_{2} \quad a_{3}\right| &\\
        x_{1}\left|a_{1} \quad a_{2} \quad a_{3}\right| &= x_{1} \operatorname{det} A
        \end{aligned}
        $$
    
        The last two determinants are zero because of repeated columns
        
    \end{enumerate}
    
    \item [6.] Find $A^{-1}$ from the cofactor formula $C^{\mathrm{T}} / \operatorname{det} A$. Use symmetry in part (b).
    \begin{enumerate}
        \item [a.] \textbf{Q.} $A=\left[\begin{array}{lll}1 & 2 & 0 \\ 0 & 3 & 0 \\ 0 & 7 & 1\end{array}\right]$ 
        \textbf{A.}
        
        $$
        \begin{aligned}
        A^{-1}&=\frac{C^{T}}{\operatorname{det} A}\\
        \operatorname{det} A &= \left|\begin{array}{lll}
        1 & 2 & 0 \\
        0 & 3 & 0 \\
        0 & 7 & 1
        \end{array}\right| \\
        &=1(3)-2(0)+0 \\
        &=3\\
        C&=\left[\begin{array}{lll}
        C_{11} & C_{12} & C_{13} \\
        C_{21} & C_{22} & C_{23} \\
        C_{31} & C_{32} & C_{33}
        \end{array}\right]\\
        &=\left[\begin{array}{ccc}
        \left|\begin{array}{cc}
        3 & 0 \\
        7 & 1
        \end{array}\right| & -\left|\begin{array}{ccc}
        0 & 0 \\
        0 & 1
        \end{array}\right| & \left|\begin{array}{ll}
        0 & 3 \\
        0 & 7
        \end{array}\right| \\
        -\left|\begin{array}{cc}
        2 & 0 \\
        7 & 1
        \end{array}\right| & \left|\begin{array}{cc}
        1 & 0 \\
        0 & 1
        \end{array}\right| & -\left|\begin{array}{cc}
        1 & 2 \\
        0 & 7
        \end{array}\right| \\
        \mid \begin{array}{cc}
        2 & 0 \\
        3 & 0
        \end{array} & -\left|\begin{array}{cc}
        1 & 0 \\
        0 & 0
        \end{array}\right| & \left|\begin{array}{ll}
        1 & 2 \\
        0 & 3
        \end{array}\right|
        \end{array}\right]\\
        &=\left[\begin{array}{ccc}
        3 & 0 & 0 \\
        -2 & 1 & -7 \\
        0 & 0 & 3
        \end{array}\right]\\
        A^{-1}&=\frac{1}{3}\left[\begin{array}{ccc}
        3 & -2 & 0 \\
        0 & 1 & 0 \\
        0 & -7 & 3
        \end{array}\right]
        \end{aligned}
        $$
        
        \item [b.] \textbf{Q.} $A=\left[\begin{array}{rrr}2 & -1 & 0 \\ -1 & 2 & -1 \\ 0 & -1 & 2\end{array}\right]$ 
        
        \textbf{A.}
        
        $$
        \begin{aligned}
        \operatorname{det} A &=\left|\begin{array}{ccc}
        2 & -1 & 0 \\
        -1 & 2 & -1 \\
        0 & -1 & 2
        \end{array}\right| \\
        &=2(4-1)+1(-2)+0 \\
        &=4\\
        C &=\left[\begin{array}{lll}
        C_{11} & C_{12} & C_{13} \\
        C_{12} & C_{22} & C_{23} \\
        C_{13} & C_{23} & C_{33}
        \end{array}\right] \\
        &=\left[\begin{array}{ccc}
        \left|\begin{array}{cc}
        2 & -1 \\
        -1 & 2
        \end{array}\right| & -\left|\begin{array}{cc}
        -1 & -1 \\
        0 & 2
        \end{array}\right| & \left|\begin{array}{cc}
        -1 & 2 \\
        0 & -1
        \end{array}\right| \\
        -\left|\begin{array}{cc}
        -1 & -1 \\
        0 & 2
        \end{array}\right| & \left|\begin{array}{cc}
        2 & 0 \\
        0 & 2
        \end{array}\right| & -\left|\begin{array}{cc}
        2 & -1 \\
        0 & -1
        \end{array}\right| \\
        \left|\begin{array}{cc}
        -1 & 2 \\
        0 & -1
        \end{array}\right| & -\left|\begin{array}{ll}
        2 & -1 \\
        0 & -1
        \end{array}\right| & \left|\begin{array}{cc}
        2 & -1 \\
        -1 & 2
        \end{array}\right|
        \end{array}\right] \\
        &=\left[\begin{array}{lll}
        3 & 2 & 1 \\
        2 & 4 & 2 \\
        1 & 2 & 3
        \end{array}\right]\\
        A^{-1}&=\frac{1}{4}\left[\begin{array}{lll}
        3 & 2 & 1 \\
        2 & 4 & 2 \\
        1 & 2 & 3
        \end{array}\right]
        \end{aligned}
        $$
        
    \end{enumerate}
    
    \item [9.] \textbf{Q.} Suppose $\operatorname{det} A=1$ and you know all the cofactors in $C$. How can you find $A$ ? 
    
    \textbf{A.}
    
    $$
    \begin{aligned}
    A^{-1}&=\frac{C^{T}}{\operatorname{det} A}\\
    \operatorname{det} A &= 1\\
    A^{-1}&=C^{T}
    \end{aligned}
    $$
    
    $\boldsymbol{A}$ is the inverse of $C^{T}$. Matrix $\boldsymbol{A}$ can be found from the cofactor matrix $\boldsymbol{C}$.
    
    
    \item [19.] \textbf{Q.} The parallelogram with sides $(2,1)$ and $(2,3)$ has the same area as the parallelogram with sides $(2,2)$ and $(1,3)$. Find those areas from 2 by 2 determinants and say why they must be equal. (I can't see why from a picture. Please write to me if you do.) 

    \textbf{A.}
    
    $$
    \begin{aligned}
    \left|\begin{array}{ll}
    2 & 1 \\
    2 & 3
    \end{array}\right| &=(2)(3)-(2)(1)\\
    & = 4\\
    \left|\begin{array}{ll}
    2 & 2 \\
    1 & 3
    \end{array}\right| &=(2)(3)-(2)(1)\\
    &= 4
    \end{aligned}
    $$
    
    Both have the same determinants because the transpose has the same determinant.

\end{enumerate}

\end{document}