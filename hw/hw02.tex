\documentclass[main.tex]{subfiles}
\begin{document}

\subsection{Section 2.4}
\begin{enumerate}
    \item [2.] What rows or columns or matrices do you multiply to find
    \begin{enumerate}
        \item [a.] \textbf{Q.} the third column of $A B$ ? \textbf{A.} Multiply the matrix $A$ with the third column of $B$.
        
        \item [b.] \textbf{Q.} the first row of $A B$ ? \textbf{A.} Multiply the first row of $A$ with the matrix $B$.
        
        \item [c.] \textbf{Q.} the entry in row 3, column 4 of $A B$? \textbf{A.} Multiply the third row of $A$ with the fourth column of $B$.
        
        \item [d.] \textbf{Q.} the entry in row 1, column 1 of $C D E$? \textbf{A.} Multiply the first row of $C$ with matrix $D$ with the first column of $E$
        
    \end{enumerate}
    
    \item [5.] \textbf{Q.} Compute $A^{2}$ and $A^{3}$. Make a prediction for $A^{5}$ and $A^{n}$ :
    $$
    A=\left[\begin{array}{ll}
    1 & b \\
    0 & 1
    \end{array}\right] \text { and } A=\left[\begin{array}{ll}
    2 & 2 \\
    0 & 0
    \end{array}\right]
    $$
    
    \textbf{A.} 
    
    $$
    \begin{aligned}
    A^{2} &=\left[\begin{array}{ll}
    1 & b \\
    0 & 1
    \end{array}\right]\left[\begin{array}{ll}
    1 & b \\
    0 & 1
    \end{array}\right] \\
    A^{2} &=\left[\begin{array}{cc}
    1 & 2 b \\
    0 & 1
    \end{array}\right]\\
    A^{3} &=\left[\begin{array}{ll}
    1 & b \\
    0 & 1
    \end{array}\right]\left[\begin{array}{ll}
    1 & b \\
    0 & 1
    \end{array}\right]\left[\begin{array}{ll}
    1 & b \\
    0 & 1
    \end{array}\right] \\
    A^{3} &=\left[\begin{array}{cc}
    1 & 2 b \\
    0 & 1
    \end{array}\right]\left[\begin{array}{ll}
    1 & b \\
    0 & 1
    \end{array}\right] \\
    A^{3} &=\left[\begin{array}{cc}
    1 & 3 b \\
    0 & 1
    \end{array}\right]\\
    A^{5} & =\left[\begin{array}{cc}
    1 & 5 b \\
    0 & 1
    \end{array}\right]\\
    A^{n} & =\left[\begin{array}{cc}
    1 & n b \\
    0 & 1
    \end{array}\right]
    \end{aligned}
    $$
    
    $$
    \begin{aligned}
    A^{2} & = \left[\begin{array}{ll}
    2 & 2 \\
    0 & 0
    \end{array}\right]\left[\begin{array}{ll}
    2 & 2 \\
    0 & 0
    \end{array}\right] \\
    A^{2} & = \left[\begin{array}{ll}
    4 & 4 \\
    0 & 0
    \end{array}\right]\\
    A^{3} & = \left[\begin{array}{ll}
    2 & 2 \\
    0 & 0
    \end{array}\right]\left[\begin{array}{ll}
    2 & 2 \\
    0 & 0
    \end{array}\right]\left[\begin{array}{ll}
    2 & 2 \\
    0 & 0
    \end{array}\right] \\
    A^{3} & = \left[\begin{array}{ll}
    4 & 4 \\
    0 & 0
    \end{array}\right]\left[\begin{array}{ll}
    2 & 2 \\
    0 & 0
    \end{array}\right] \\
    A^{3} & = \left[\begin{array}{ll}
    8 & 8 \\
    0 & 0
    \end{array}\right]\\
    A^{5} & = \left[\begin{array}{cc}
    32 & 32 \\
    0 & 0
    \end{array}\right]\\
    A^{n} & = \left[\begin{array}{ll}
    2^{n} & 2^{n} \\
    0 & 0
    \end{array}\right]
    \end{aligned}
    $$
    
    \item [11.] (3 by 3 matrices) Choose the only $B$ so that for every matrix $A$
    \begin{enumerate}
        \item [a.] \textbf{Q.} $BA=4A$ \textbf{A.} 
        $$
        \begin{aligned}
        B A &=4 A \\
        (B A) \cdot A^{-1} &= (4 A) \cdot A^{-1} \\
        B\left(A A^{-1}\right) &= 4\left(A A^{-1}\right) \\
        B(I) &=4(I) \\
        B & =4 I \\
        B & =\left[\begin{array}{lll}
        4 & 0 & 0 \\
        0 & 4 & 0 \\
        0 & 0 & 4
        \end{array}\right]
        \end{aligned}
        $$
        
        \item [b.] \textbf{Q.} $B A=4 B$ \textbf{A.}
        $$
        \begin{gathered}
        (B A) \cdot B^{-1}=(4 B) \cdot B^{-1} \\
        \left(B A B^{-1}\right)=\left(4 B B^{-1}\right) \\
        \left(A B B^{-1}\right)=\left(4 B B^{-1}\right) \\
        A B B^{-1}=4 B B^{-1} \\
        A I=4 I
        \end{gathered}
        $$
        
        There is no $B$ in final equation $AI = 4I$, therefore $B A=4 B$ is true for every matrix $A$ if the matrix $B=\left[\begin{array}{lll}0 & 0 & 0 \\ 0 & 0 & 0 \\ 0 & 0 & 0\end{array}\right]$.
        
        \item [c.] \textbf{Q.} $B A$ has rows 1 and 3 of $A$ reversed and row 2 unchanged \textbf{A.}
        $$
        \begin{aligned}
        B A &=\left[\begin{array}{lll}
        0 & 0 & 1 \\
        0 & 1 & 0 \\
        1 & 0 & 0
        \end{array}\right]\left[\begin{array}{lll}
        1 & 2 & 3 \\
        3 & 4 & 5 \\
        5 & 6 & 7
        \end{array}\right] \\
        &=\left[\begin{array}{lll}
        0+0+5 & 0+0+6 & 0+0+7 \\
        0+3+0 & 0+4+0 & 0+5+0 \\
        1+0+0 & 2+0+0 & 3+0+0
        \end{array}\right] \\
        &=\left[\begin{array}{lll}
        5 & 6 & 7 \\
        3 & 4 & 5 \\
        1 & 2 & 3
        \end{array}\right]\\
        B &= \left[\begin{array}{lll}
        0 & 0 & 1 \\
        0 & 1 & 0 \\
        1 & 0 & 0
        \end{array}\right]
        \end{aligned}
        $$
        
        \item [d.] \textbf{Q.} All rows of $B A$ are the same as row 1 of $A$. \textbf{A.}
        $$
        \begin{aligned}
        B A &=\left[\begin{array}{lll}
        1 & 0 & 0 \\
        1 & 0 & 0 \\
        1 & 0 & 0
        \end{array}\right]\left[\begin{array}{lll}
        5 & 6 & 7 \\
        3 & 4 & 5 \\
        1 & 2 & 3
        \end{array}\right] \\
        &=\left[\begin{array}{lll}
        5+0+0 & 6+0+0 & 7+0+0 \\
        5+0+0 & 6+0+0 & 7+0+0 \\
        5+0+0 & 6+0+0 & 7+0+0
        \end{array}\right] \\
        &=\left[\begin{array}{lll}
        5 & 6 & 7 \\
        5 & 6 & 7 \\
        5 & 6 & 7
        \end{array}\right]\\
        B & =\left[\begin{array}{lll}
        1 & 0 & 0 \\
        1 & 0 & 0 \\
        1 & 0 & 0
        \end{array}\right]
        \end{aligned}
        $$
    \end{enumerate}

    \item [24.] \textbf{Q.} By experiment with $n=2$ and $n=3$ predict $A^{n}$ for these matrices: 
    $$
    A_{1}=\left[\begin{array}{ll}2 & 1 \\ 0 & 1\end{array}\right] \text{ and } A_{2}=\left[\begin{array}{ll}1 & 1 \\ 1 & 1\end{array}\right] \text{ and } A_{3}=\left[\begin{array}{ll}a & b \\ 0 & 0\end{array}\right]
    $$

    \textbf{A.}
    $$
    \begin{aligned}
    A_{1}^{2} &= \left[\begin{array}{ll}
    2 & 1 \\
    0 & 1
    \end{array}\right]\left[\begin{array}{ll}
    2 & 1 \\
    0 & 1
    \end{array}\right] \\
    A_{1}^{2} &= \left[\begin{array}{ll}
    4 & 3 \\
    0 & 1
    \end{array}\right]\\
    A_{1}^{3} & = \left[\begin{array}{ll}
    4 & 3 \\
    0 & 1
    \end{array}\right]\left[\begin{array}{ll}
    2 & 1 \\
    0 & 1
    \end{array}\right] \\
    A_{1}^{3} & = \left[\begin{array}{ll}
    8 & 7 \\
    0 & 1
    \end{array}\right]\\
    A_{1}^{n} & = \left[\begin{array}{cc}
    2^{n} & 2^{n}-1 \\
    0 & 1
    \end{array}\right]
    \end{aligned}
    $$
    
    $$
    \begin{aligned}
    A_{2}^{2} &= \left[\begin{array}{ll}
    1 & 1 \\
    1 & 1
    \end{array}\right]\left[\begin{array}{ll}
    1 & 1 \\
    1 & 1
    \end{array}\right] \\
    A_{2}^{2} &= \left[\begin{array}{ll}
    2 & 2 \\
    2 & 2
    \end{array}\right]\\
    A_{2}^{3} &= \left[\begin{array}{ll}
    2 & 2 \\
    2 & 2
    \end{array}\right]\left[\begin{array}{ll}
    1 & 1 \\
    1 & 1
    \end{array}\right] \\
    A_{2}^{3} &= \left[\begin{array}{ll}
    4 & 4 \\
    4 & 4
    \end{array}\right] \\
    A_{2}^{n} &= \left[\begin{array}{ll}
    2^{n-1} & 2^{n-1} \\
    2^{n-1} & 2^{n-1}
    \end{array}\right]
    \end{aligned}
    $$
    
    $$
    \begin{aligned}
    A_{3}^{2} &=\left[\begin{array}{ll}
    a & b \\
    0 & 0
    \end{array}\right]\left[\begin{array}{ll}
    a & b \\
    0 & 0
    \end{array}\right] \\
    A_{3}^{2} &=\left[\begin{array}{cc}
    a^{2} & a b \\
    0 & 0
    \end{array}\right]\\
    A_{3}^{3} &= \left[\begin{array}{cc}
    a^{2} & a b \\
    0 & 0
    \end{array}\right]\left[\begin{array}{ll}
    a & b \\
    0 & 0
    \end{array}\right] \\
    A_{3}^{3} &= \left[\begin{array}{cc}
    a^{3} & a^{2} b \\
    0 & 0
    \end{array}\right]\\
    A_{3}^{n} &= \left[\begin{array}{cc}
    a^{n} & a^{n-1} b \\
    0 & 0
    \end{array}\right]
    \end{aligned}
    $$
    
    \item [28.] Draw the cuts in $A(2$ by 3$)$ and $B(3$ by 4$)$ and $A B$ to show how each of the four multiplication rules is really a block multiplication:
    \begin{enumerate}
        \item [1.] \textbf{Q.} Matrix $A$ times columns of $B$. Columns of $A B$ \textbf{A.}
        $$
        A B=A \cdot\left[\begin{array}{llll}
        b_{1} & b_{2} & b_{3} & b_{4}
        \end{array}\right]=\left[\begin{array}{llll}
        A b_{1} & A b_{2} & A b_{3} & A b_{4}
        \end{array}\right]
        $$
        
        \item [2.] \textbf{Q.} Rows of $A$ times the matrix $B$. Rows of $A B$ \textbf{A.}
        $$
        A B=\left[\begin{array}{l}
        a_{1} \\
        a_{2}
        \end{array}\right] \cdot B=\left[\begin{array}{l}
        a_{1} B \\
        a_{2} B
        \end{array}\right]
        $$
        
        \item [3.] \textbf{Q.} Rows of $A$ times columns of $B$. Inner products (Numbers in $A B$) \textbf{A.}
        $$
        A B=\left[\begin{array}{l}
        a_{1} \\
        a_{2}
        \end{array}\right] \cdot\left[\begin{array}{llll}
        b_{1} & b_{2} & b_{3} & b_{4}
        \end{array}\right]=\left[\begin{array}{llll}
        a_{1} b_{1} & a_{1} b_{2} & a_{1} b_{3} & a_{1} b_{4} \\
        a_{2} b_{1} & a_{2} b_{2} & a_{2} b_{3} & a_{2} b_{4}
        \end{array}\right]
        $$
        
        \item [4.] \textbf{Q.} Columns of $A$ times rows of $B$. Outer products (matrices add to $A B$ ) \textbf{A.}
        $$
        A B = \left[\begin{array}{lll}a_{1} & a_{2} & a_{3} \end{array}\right] 
        \left[\begin{array}{l}
        b_{1} \\
        b_{2} \\
        b_{3}
        \end{array}\right]
        \left[a_{1} b_{1}+a_{2} b_{2}+a_{3} b_{3}\right]
        $$
    \end{enumerate}
    
    \item [33.] \textbf{Q.} If the three solutions in Question 32 are $x_{1}=(1,1,1)$ and $x_{2}=(0,1,1)$ and $x_{3}=(0,0,1)$, solve $A \bm{x}=\bm{b}$ when $\bm{b}=(3,5,8)$. Challenge problem: What is $A$? 
    \textbf{A.}
    
    $$
    \begin{aligned}
    b &= \left[\begin{array}{l}
    3 \\
    5 \\
    8
    \end{array}\right] \\
    & = 3 \cdot\left[\begin{array}{l}
    1 \\
    0 \\
    0
    \end{array}\right]+5 \cdot\left[\begin{array}{l}
    0 \\
    1 \\
    0
    \end{array}\right]+8 \cdot\left[\begin{array}{l}
    0 \\
    0 \\
    1
    \end{array}\right]\\
    & = 3 A x_{1}+5 A x_{2}+8 A x_{3} \\
    & = A \cdot 3 x_{1}+A \cdot 5 x_{2}+A \cdot 8 x_{3} \\
    & = A \cdot\left(3 x_{1}+5 x_{2}+8 x_{3}\right) \\
    x & = 3 x_{1}+5 x_{2}+8 x_{3} \\
    & = 3 \cdot\left[\begin{array}{l}
    1 \\
    1 \\
    1
    \end{array}\right]+5 \cdot\left[\begin{array}{l}
    0 \\
    1 \\
    1
    \end{array}\right]+8\left[\begin{array}{l}
    0 \\
    0 \\
    1
    \end{array}\right]\\
    & = \left[\begin{array}{c}
    3 \\
    8 \\
    16
    \end{array}\right] \\
    A & = \left[\begin{array}{lll}
    a_{1} & a_{2} & a_{3}
    \end{array}\right]\\
    A x_{3} &= A \cdot\left[\begin{array}{l}
    0 \\
    0 \\
    1
    \end{array}\right] 
    =\left[\begin{array}{l}
    0 \\
    0 \\
    1
    \end{array}\right]\\
    a_{3} & = \left[\begin{array}{l}
    0 \\
    0 \\
    1
    \end{array}\right]\\
    A x_{2} &= A \cdot\left[\begin{array}{l}
    0 \\
    1 \\
    1
    \end{array}\right]=\left[\begin{array}{l}
    0 \\
    1 \\
    0
    \end{array}\right]\\
    a_{2}+a_{3} & = \left[\begin{array}{l}
    0 \\
    1 \\
    0
    \end{array}\right]\\
    a_{2} & = \left[\begin{array}{c}
    0 \\
    1 \\
    -1
    \end{array}\right]\\
    A x_{1} &= A \cdot\left[\begin{array}{l}
    1 \\
    1 \\
    1
    \end{array}\right]=\left[\begin{array}{l}
    1 \\
    0 \\
    0
    \end{array}\right]\\
    a_{1}+a_{2}+a_{3} &= \left[\begin{array}{l}
    1 \\
    0 \\
    0
    \end{array}\right]\\
    a_{1} &= \left[\begin{array}{r}
    1 \\
    -1 \\
    1
    \end{array}\right]\\
    A &= \left[\begin{array}{ccc}
    1 & 0 & 0 \\
    -1 & 1 & 0 \\
    0 & -1 & 1
    \end{array}\right]
    \end{aligned}
    $$
    
\end{enumerate}

\subsection{Section 2.5}
\begin{enumerate}
    \item [1.] \textbf{Q.} Find the inverses (directly or from the 2 by 2 formula) of $A, B, C$ :
    $$
    A=\left[\begin{array}{ll}0 & 3 \\ 4 & 0\end{array}\right] \text { and } B=\left[\begin{array}{ll}2 & 0 \\ 4 & 2\end{array}\right] \text { and } C=\left[\begin{array}{ll}3 & 4 \\ 5 & 7\end{array}\right]
    $$
    
    \textbf{A.}
    $$
    \begin{aligned}
    P^{-1} &=\frac{1}{a d-b c}\left[\begin{array}{cc}
    d & -b \\
    -c & a
    \end{array}\right] \\
    A^{-1} &=\frac{1}{0 \cdot 0-3 \cdot 4}\left[\begin{array}{cc}
    0 & -3 \\
    -4 & 0
    \end{array}\right] \\
    &=\frac{1}{-12}\left[\begin{array}{cc}
    0 & -3 \\
    -4 & 0
    \end{array}\right] \\
    &=\left[\begin{array}{cc}
    \frac{0}{-12} & \frac{-3}{-12} \\
    \frac{-4}{-12} & \frac{0}{-12}
    \end{array}\right] \\
    &=\left[\begin{array}{cc}
    0 & \frac{1}{4} \\
    \frac{1}{3} & 0
    \end{array}\right]\\
    B^{-1} &=\frac{1}{4-0}\left[\begin{array}{cc}
    2 & 0 \\
    -4 & 2
    \end{array}\right] \\
    &=\frac{1}{4}\left[\begin{array}{cc}
    2 & 0 \\
    -4 & 2
    \end{array}\right] \\
    &=\left[\begin{array}{cc}
    \frac{2}{4} & \frac{0}{4} \\
    \frac{-4}{4} & \frac{2}{4}
    \end{array}\right] \\
    &=\left[\begin{array}{cc}
    \frac{1}{2} & 0 \\
    -1 & \frac{1}{2}
    \end{array}\right]\\
    C^{-1} &=\frac{1}{3 \cdot 7-4 \cdot 5}\left[\begin{array}{cc}
    7 & -4 \\
    -5 & 3
    \end{array}\right] \\
    &=\frac{1}{21-20}\left[\begin{array}{cc}
    7 & -4 \\
    -5 & 3
    \end{array}\right] \\
    &=\left[\begin{array}{cc}
    7 & -4 \\
    -5 & 3
    \end{array}\right]
    \end{aligned}
    $$
    
    \item [12.] \textbf{Q.} If the product $C=A B$ is invertible ($A$ and $B$ are square), then $A$ itself is invertible. Find a formula for $A^{-1}$ that involves $C^{-1}$ and $B$. \textbf{A.}
    $$
    \begin{aligned}
    C &= A B\\
    C^{-1} & = (A B)^{-1} \\
    & = B^{-1} A^{-1}\\
    B C^{-1} & =\left(B B^{-1}\right) A^{-1} \\
    B C^{-1} & =I \cdot A^{-1} \\
    A^{-1} & =B C^{-1}
    \end{aligned}
    $$
    
    \item [14.] \textbf{Q.} If you add row 1 of $A$ to row 2 to get $B$, how do you find $B^{-1}$ from $A^{-1}$ ? Notice the order. The inverse of $B=\left[\begin{array}{ll}1 & 0 \\ 1 & 1\end{array}\right][A]$ is ? 
    \textbf{A.}
    $$
    \begin{aligned}
    B &= \left[\begin{array}{ll}
    1 & 0 \\
    1 & 1
    \end{array}\right][A]\\
    B^{-1} &= \left(\left[\begin{array}{ll}
    1 & 0 \\
    1 & 1
    \end{array}\right][A]\right)^{-1} \\
    B^{-1} &= [A]^{-1}\left[\begin{array}{ll}
    1 & 0 \\
    1 & 1
    \end{array}\right]^{-1}\\
    \left[\begin{array}{ll}
    a & b \\
    c & d
    \end{array}\right]^{-1} &= \frac{1}{a c-b d}\left[\begin{array}{cc}
    d & -b \\
    -c & a
    \end{array}\right]^{T}\\
    \left[\begin{array}{ll}
    1 & 0 \\
    1 & 1
    \end{array}\right]^{-1} & = \frac{1}{1-0}\left[\begin{array}{cc}
    1 & -1 \\
    0 & 1
    \end{array}\right] \\
    &=\left[\begin{array}{cc}
    1 & 0 \\
    -1 & 1
    \end{array}\right]\\
    B^{-1} &= [A]^{-1}\left[\begin{array}{cc}
    1 & 0 \\
    -1 & 1
    \end{array}\right]
    \end{aligned}
    $$
    
    \item [18.] \textbf{Q.} If $B$ is the inverse of $A^{2}$, show that $A B$ is the inverse of $A$. 
    \textbf{A.}
    
    $$
    \begin{aligned}
    \text{Show that } A(A B) &= (A B) A = I\\
    A(A B) &= (A A) B \\
    &= \left(A^{2}\right) B \\
    B\left(A^{2}\right) &= \left(A^{2}\right) B=I\\
    \text{Show that } (A B) A &= I\\
    \left(A^{2}\right) B & =I \\
    A(A B) & = I\\
    A^{-1} A(A B) A &= A^{-1} I A \\
    &=A^{-1} A \\
    &=I\\
    (A B) A &= I
    \end{aligned}
    $$
    
    \item [22.] \textbf{Q.} Change $I$ into $A^{-1}$ as you reduce $A$ to $I$ (by row operations):
    $$
    \left[\begin{array}{ll}
    A & I
    \end{array}\right]=\left[\begin{array}{llll}
    1 & 3 & 1 & 0 \\
    2 & 7 & 0 & 1
    \end{array}\right] \text { and }[A \quad I]=\left[\begin{array}{llll}
    1 & 4 & 1 & 0 \\
    3 & 9 & 0 & 1
    \end{array}\right]
    $$
    \textbf{A.}
    $$
    \begin{aligned}
    \left[\begin{array}{ll}
    A & I
    \end{array}\right] & =\left[\begin{array}{llll}
    1 & 3 & 1 & 0 \\
    2 & 7 & 0 & 1
    \end{array}\right] \\
    R_{2}-2 R_{1} & \rightarrow R_{2}\\
    & = \left[\begin{array}{cccc}
    1 & 0 & 7 & -3 \\
    0 & 1 & -2 & 1
    \end{array}\right]\\
    R_{1}-3 R_{2} & \rightarrow R_{1}\\
    & =\left[\begin{array}{cccc}
    1 & 0 & 7 & -3 \\
    0 & 1 & -2 & 1
    \end{array}\right] \\
    &=\left[\begin{array}{ll}
    I & A^{-1}
    \end{array}\right]\\
    \left[\begin{array}{ll}
    A & I
    \end{array}\right] &= \left[\begin{array}{llll}
    1 & 4 & 1 & 0 \\
    3 & 9 & 0 & 1
    \end{array}\right]\\
    R_{2} -3 R_{1} & \rightarrow R_{2}\\
    & = \left[\begin{array}{cccc}
    1 & 4 & 1 & 0 \\
    0 & -3 & -3 & 1
    \end{array}\right] \\
    R_{1} + R_{2} & \rightarrow R_{1} \\
    \frac{1}{3} R_{2} & \rightarrow R_{2} \\
    & = \left[\begin{array}{cccc}
    1 & 1 & -2 & 1 \\
    0 & -1 & -1 & \frac{1}{3}
    \end{array}\right] \\
    R_{1}+R_{2} & \rightarrow R_{1} \\
    -R_{2} & \rightarrow R_{2}\\
    &=\left[\begin{array}{cccc}
    1 & 0 & -3 & \frac{4}{3} \\
    0 & 1 & 1 & -\frac{1}{3}
    \end{array}\right] \\
    &=\left[\begin{array}{ll}
    I & A^{-1}
    \end{array}\right]
    \end{aligned}
    $$
    
    \item [34.] \textbf{Q.} Find and check the inverses (assuming they exist) of these block matrices:
    $$
    \left[\begin{array}{ll}
    I & 0 \\
    C & I
    \end{array}\right]\left[\begin{array}{ll}
    A & 0 \\
    C & D
    \end{array}\right]\left[\begin{array}{ll}
    0 & I \\
    I & D
    \end{array}\right] .
    $$
    \textbf{A.}
    $$
    \begin{aligned}
    \left[\begin{array}{ll}
    I & 0 \\
    C & I
    \end{array}\right]^{-1} &= \left[\begin{array}{cc}
    I & 0 \\
    -C & I
    \end{array}\right]\\
    \left[\begin{array}{cc}
    I & 0 \\
    -C & I
    \end{array}\right]\left[\begin{array}{ll}
    I & 0 \\
    C & I
    \end{array}\right] &=\left[\begin{array}{cc}
    I I+0 C & I 0+0 I \\
    -C I+I C & -C 0+I I
    \end{array}\right] \\
    & = I\\
    \left[\begin{array}{cc}
    A^{-1} & 0 \\
    X & D^{-1}
    \end{array}\right]\left[\begin{array}{cc}
    A & 0 \\
    C & D
    \end{array}\right]&=\left[\begin{array}{cc}
    A^{-1} A+0 C & A^{-1} 0+0 D \\
    X A+D^{-1} C & X 0+D^{-1} D
    \end{array}\right]\\
    & = I \quad \text{ if } \quad X=-D^{-1} C A^{-1}\\
    \left[\begin{array}{cc}
    I & -D \\
    0 & I
    \end{array}\right]\left[\begin{array}{ll}
    I & D \\
    0 & I
    \end{array}\right]&=I\\
    \left[\begin{array}{cc}
    I & -D \\
    0 & I
    \end{array}\right]\left[\begin{array}{ll}
    0 & I \\
    I & 0
    \end{array}\right]&=\left[\begin{array}{cc}
    -D & I \\
    I & 0
    \end{array}\right]\\
    \left[\begin{array}{cc}
    -D & I \\
    I & 0
    \end{array}\right]
    \left[\begin{array}{ll}
    0 & I \\
    I & D
    \end{array}\right] 
    & = I
    \end{aligned}
    $$

\end{enumerate}

\subsection{Section 2.6}
\begin{enumerate}
    \item [3.] \textbf{Q.} (Move to 3 by 3) Forward elimination changes $A\bm{x}=\bm{b}$ to a triangular $U\bm{x}=\bm{c}$:
    
    $$
    \begin{array}{lrr}x+y+z=5 & x+y+z=5 & x+y+z=5 \\ x+2 y+3 z=7 & y+2 z=2 & y+2 z=2 \\ x+3 y+6 z=11 & 2 y+5 z=6 & z=2\end{array}
    $$
    
    The equation $z=2$ in $U \bm{x}=\bm{c}$ comes from the original $x+3 y+6 z=11$ in $A \bm{x}=\bm{b}$ by subtracting $l_{31}= \dots$ times equation $1$ and $\l_{32}=\ldots$ times the final equation 2. Reverse that to recover $\left[\begin{array}{llll}1 & 3 & 6 & 11\end{array}\right]$ in the last row of $A$ and $\bm{b}$ from the final $\left[\begin{array}{llll}1 & 1 & 1 & 5\end{array}\right]$ and $\left[\begin{array}{llll}0 & 1 & 2 & 2\end{array}\right]$ and $\left[\begin{array}{llll}0 & 0 & 1 & 2\end{array}\right]$ in $U$ and $c$:
    
    $$
    \text{Row 3 of } \left[\begin{array}{ll}A & \bm{b} \end{array}\right]=\left(\ell_{31}\right. \text{ Row } 1+\ell_{32} \text{ Row } 2+1 \text{ Row } 3) 
    \text{ of } \left[\begin{array}{ll}U & \bm{c}\end{array}\right]
    $$
    
    In matrix notation this is multiplication by $L$. So $A=L U$ and $\bm{b}=L \bm{c}$. 
    \textbf{A.}
    $$
    \begin{aligned}
    A &= \left[\begin{array}{lll}
    1 & 1 & 1 \\
    1 & 2 & 3 \\
    1 & 3 & 6
    \end{array}\right] \\
    U &= \left[\begin{array}{lll}
    1 & 1 & 1 \\
    0 & 1 & 2 \\
    0 & 0 & 1
    \end{array}\right]\\
    \left(E_{32} E_{31} E_{21}\right) A &= U\\
    \left[\begin{array}{ccc}
    1 & 0 & 0 \\
    0 & 1 & 0 \\
    0 & -2 & 1
    \end{array}\right]\left[\begin{array}{ccc}
    1 & 0 & 0 \\
    0 & 1 & 0 \\
    -1 & 0 & 1
    \end{array}\right]\left[\begin{array}{ccc}
    1 & 0 & 0 \\
    -1 & 1 & 0 \\
    0 & 0 & 1
    \end{array}\right] A &= U\\
    \left[\begin{array}{ccc}
    1 & 0 & 0 \\
    -1 & 1 & 0 \\
    -1 & -2 & 1
    \end{array}\right] A &= U\\
    I_{31} &= 1\\
    l_{32} &= 2\\
    [U c] &=\left[\begin{array}{llll}
    1 & 1 & 1 & 5 \\
    0 & 1 & 2 & 2 \\
    0 & 0 & 1 & 2
    \end{array}\right] \\
    [A b] &= \left[\begin{array}{llll}
    1 & 1 & 1 & 5 \\
    1 & 2 & 3 & 7 \\
    1 & 3 & 6 & 11
    \end{array}\right] \\
    L[U c] &= \left[\begin{array}{llll}
    1 & 0 & 0 \\
    1 & 1 & 0 \\
    1 & 2 & 1
    \end{array}\right]\left[\begin{array}{lllll}
    1 & 1 & 1 & 5 \\
    0 & 1 & 2 & 2 \\
    0 & 0 & 1 & 2
    \end{array}\right] \\
    &= \left[\begin{array}{llll}
    1 & 1 & 1 & 5 \\
    1 & 2 & 3 & 7 \\
    1 & 3 & 6 & 11
    \end{array}\right]\\
    x+y+z=5+2 \quad y+2 z=2+1 \quad z=2 \quad & \text { gives } \quad x+3 y+6 z=11
    \end{aligned}
    $$
    
    \item [4.] \textbf{Q.} What are the 3 by 3 triangular systems $L \bm{c}=\bm{b}$ and $U\bm{x}=\bm{c}$ from Problem 3? Check that $c=(5,2,2)$ solves the first one. Which $\bm{x}$ solves the second one? 
    
    \textbf{A.} 
    
    $$
    \begin{aligned}
    A &= \left[\begin{array}{lll}
    1 & 1 & 1 \\
    1 & 2 & 3 \\
    1 & 3 & 6
    \end{array}\right] \\
    b &= \left[\begin{array}{l}
    5 \\
    7
    \end{array}\right]\\
    \left(E_{32} E_{31} E_{21}\right) A &= U\\
    \left[\begin{array}{ccc}
    1 & 0 & 0 \\
    0 & 1 & 0 \\
    0 & -2 & 1
    \end{array}\right]\left[\begin{array}{ccc}
    1 & 0 & 0 \\
    0 & 1 & 0 \\
    -1 & 0 & 1
    \end{array}\right]\left[\begin{array}{ccc}
    1 & 0 & 0 \\
    -1 & 1 & 0 \\
    0 & 0 & 1
    \end{array}\right]\left[\begin{array}{lll}
    1 & 1 & 1 \\
    1 & 2 & 3 \\
    1 & 3 & 6
    \end{array}\right] &= U\\
    \left[\begin{array}{ccc}
    1 & 0 & 0 \\
    -1 & 1 & 0 \\
    1 & -2 & 1
    \end{array}\right]\left[\begin{array}{lll}
    1 & 1 & 1 \\
    1 & 2 & 3 \\
    1 & 3 & 6
    \end{array}\right] &= U\\
    U &= \left[\begin{array}{lll}
    1 & 1 & 1 \\
    0 & 1 & 2 \\
    0 & 0 & 1
    \end{array}\right]\\
    L &=\left[\begin{array}{lll}
    1 & 0 & 0 \\
    1 & 1 & 0 \\
    0 & 0 & 1
    \end{array}\right]\left[\begin{array}{lll}
    1 & 0 & 0 \\
    0 & 1 & 0 \\
    1 & 0 & 1
    \end{array}\right]\left[\begin{array}{lll}
    1 & 0 & 0 \\
    0 & 1 & 0 \\
    0 & 2 & 1
    \end{array}\right] \\
    L &= \left[\begin{array}{lll}
    1 & 0 & 0 \\
    1 & 1 & 0 \\
    1 & 2 & 1
    \end{array}\right]\\
    L c &= b \\
    \left[\begin{array}{lll}
    1 & 0 & 0 \\
    1 & 1 & 0 \\
    1 & 2 & 1
    \end{array}\right] c &= \left[\begin{array}{l}
    5 \\
    7 \\
    11
    \end{array}\right]\\
    c&=\left[\begin{array}{l}
    5 \\
    2 \\
    2
    \end{array}\right]\\
    U x &= c \\
    \left[\begin{array}{lll}
    1 & 1 & 1 \\
    0 & 1 & 2 \\
    0 & 0 & 1
    \end{array}\right] x &= \left[\begin{array}{l}
    5 \\
    2 \\
    2
    \end{array}\right]\\
    x &= \left[\begin{array}{l}
    5 \\
    -2 \\
    2
    \end{array}\right]
    \end{aligned}
    $$
    
    \item [12.] \textbf{Q.} $A$ and $B$ are symmetric across the diagonal (because $4=4$ ). Find their triple factorizations $L D U$ and say how $U$ is related to $L$ for these symmetric matrices: 
    
    $$
    \text{Symmetric: } \quad 
    A=\left[\begin{array}{rr}
    2 & 4 \\
    4 & 11
    \end{array}\right] \text { and } B=\left[\begin{array}{rrr}
    1 & 4 & 0 \\
    4 & 12 & 4 \\
    0 & 4 & 0
    \end{array}\right]
    $$
    
    \textbf{A.}
    
    $$
    \begin{aligned}
    A &= \left[\begin{array}{cc}
    2 & 4 \\
    4 & 11
    \end{array}\right]\\
    R_{2}-2 R_{1} & \rightarrow R_2\\
    E A &= \left[\begin{array}{ll}
    2 & 4 \\
    0 & 3
    \end{array}\right]\\
    E A &= U\\
    \left[\begin{array}{ll}
    2 & 4 \\
    0 & 3
    \end{array}\right] &= U\\
    I &= \left[\begin{array}{ll}
    1 & 0 \\
    0 & 1
    \end{array}\right]\\
    \left(R_{2}-2 R_{1}\right) & \rightarrow R_2\\
    E &= \left[\begin{array}{cc}
    1 & 0 \\
    -2 & 1
    \end{array}\right]\\
    L &= E^{-1} \\
    & = \frac{1}{(1 \cdot 1)-(-2)(0)} \left[\begin{array}{cc}
    1 & 0 \\
    -2 & 1
    \end{array}\right]^{-1} \\
    & =\left[\begin{array}{ll}
    1 & 0 \\
    2 & 1
    \end{array}\right]\\
    \left[\begin{array}{ll}
    2 & 4 \\
    0 & 3
    \end{array}\right] &= \left[\begin{array}{ll}
    2 & 0 \\
    0 & 3
    \end{array}\right]\left[\begin{array}{ll}
    1 & 2 \\
    0 & 1
    \end{array}\right] \\
    & =D U\\
    U &= \left[\begin{array}{ll}
    1 & 2 \\
    0 & 1
    \end{array}\right] \quad \text{ new upper}\\
    A &= L D U \\
    A &= \left[\begin{array}{ll}
    1 & 0 \\
    2 & 1
    \end{array}\right]\left[\begin{array}{ll}
    2 & 0 \\
    0 & 3
    \end{array}\right]\left[\begin{array}{ll}
    1 & 2 \\
    0 & 1
    \end{array}\right]\\
    L &= \left[\begin{array}{ll}
    1 & 0 \\
    2 & 1
    \end{array}\right]\\ 
    U &= \left[\begin{array}{ll}
    1 & 2 \\
    0 & 1
    \end{array}\right]\\
    U &=L^{T}
    \end{aligned}
    $$
    
    $$
    \begin{aligned}
    B &= \left[\begin{array}{ccc}
    1 & 4 & 0 \\
    4 & 12 & 4 \\
    0 & 4 & 0
    \end{array}\right]\\
    R_{2} - 4 R_{1} & \rightarrow R_{2}\\
    & = \left[\begin{array}{ccc}
    1 & 4 & 0 \\
    0 & -4 & 4 \\
    0 & 4 & 0
    \end{array}\right] \\
     R_{2} / 4 & \rightarrow R_{2} \\
     R_{3} / 4 & \rightarrow R_{3} \\
    & =\left[\begin{array}{ccc}
    1 & 4 & 0 \\
    0 & -1 & 1 \\
    0 & 1 & 0
    \end{array}\right] \\
    R_{1}+R_{2} & \rightarrow R_{3}\\
    U & = \left[\begin{array}{ccc}
    1 & 4 & 0 \\
    0 & -1 & 1 \\
    0 & 0 & 1
    \end{array}\right]\\
    U &=L^{T} \\
    U^{T} &=\left(L^{T}\right)^{T} \\
    U^{T} &=L\\
    L &= \left[\begin{array}{ccc}
    1 & 4 & 0 \\
    0 & -1 & 1 \\
    0 & 0 & 1
    \end{array}\right]^{T} \\
    &=\left[\begin{array}{ccc}
    1 & 0 & 0 \\
    4 & -1 & 0 \\
    0 & 1 & 1
    \end{array}\right]\\
    D &= \left[\begin{array}{ccc}
    1 & 0 & 0 \\
    0 & -4 & 0 \\
    0 & 0 & 4
    \end{array}\right]\\
    B &=L D U \\
    & =\left[\begin{array}{ccc}
    1 & 0 & 0 \\
    4 & 1 & 0 \\
    0 & -1 & 1
    \end{array}\right]\left[\begin{array}{ccc}
    1 & 0 & 0 \\
    0 & -4 & 0 \\
    0 & 0 & 4
    \end{array}\right]\left[\begin{array}{ccc}
    1 & 4 & 0 \\
    0 & -1 & 1 \\
    0 & 0 & 1
    \end{array}\right]
    \end{aligned}
    $$
    
\end{enumerate}

\subsection{Section 2.7}
\begin{enumerate}
    \item [2.] Write down the 2 by 2 triangular systems $L \bm{c}=\bm{b}$ and $U \bm{x}=\bm{c}$ from Problem $1$. Check that $\bm{c}=(5,2)$ solves the first one. Find $\bm{x}$ that solves the second one.
    
    \item [6.] What two elimination matrices $E_{21}$ and $E_{32}$ put $A$ into upper triangular form $E_{32} E_{21} A=U$ ? Multiply by $E_{32}^{-1}$ and $E_{21}^{-1}$ to factor $A$ into $L U=E_{21}^{-1} E_{32}^{-1} U$:
    
    $$
    A=\left[\begin{array}{lll}
    1 & 1 & 1 \\
    2 & 4 & 5 \\
    0 & 4 & 0
    \end{array}\right]
    $$
    
    \item [29.] Prove that no reordering of rows and reordering of columns can transpose a typical matrix. (Watch the diagonal entries.)
    
\end{enumerate}

\subsection{Section 3.1}

The first problems 1-8 are about vector spaces in general. The vectors in those spaces are not necessarily column vectors. In the definition of a vector space, vector addition $x+y$ and scalar multiplication $c x$ must obey the following eight rules:
\begin{enumerate}
    \item [1.] $\bm{x}+\bm{y}=\bm{y}+\bm{x}$
    \item [2.] $\bm{x}+(\bm{y}+\bm{z})=(\bm{x}+\bm{y})+\bm{z}$
    \item [3.] There is a unique "zero vector" such that $\bm{x}+\bm{0}=\bm{x}$ for all $\bm{x}$
    \item [4.] For each $\bm{x}$ there is a unique vector $-\bm{x}$ such that $\bm{x}+(-\bm{x})=\bm{0}$
    \item [5.] 1 times $\bm{x}$ equals $\bm{x}$
    \item [6.] $\left(c_{1} c_{2}\right) \bm{x}=c_{1}\left(c_{2} \bm{x}\right)$
    \item [7.] $c(\bm{x}+\bm{y})=c \bm{x}+c \bm{y}$
    \item [8.] $\left(c_{1}+c_{2}\right) \bm{x}=c_{1} \bm{x}+c_{2} \bm{x}$.
\end{enumerate}

Questions

\begin{enumerate}
    \item [1.] Suppose $\left(x_{1}, x_{2}\right)+\left(y_{1}, y_{2}\right)$ is defined to be $\left(x_{1}+y_{2}, x_{2}+y_{1}\right)$. With the usual multiplication $c \bm{x}=\left(c x_{1}, c x_{2}\right)$, which of the eight conditions are not satisfied?
    
    \item [5.] 
    \begin{enumerate}
        \item [a.] Describe a subspace of $\mathbf{M}$ that contains $A=\left[\begin{array}{ll}1 & 0 \\ 0 & 0\end{array}\right]$ but not $B=\left[\begin{array}{ll}0 & 0 \\ 0 & -1\end{array}\right]$.
        \item [b.] If a subspace of $\mathbf{M}$ contains $A$ and $B$, must it contain $I$?
        \item [c.] Describe a subspace of $\mathbf{M}$ that contains no nonzero diagonal matrices.
    \end{enumerate}
    
    \item [11.] Describe the smallest subspace of the matrix space $\mathbf{M}$ that contains
    
    $$
    \text{ (a) } \left[\begin{array}{ll}1 & 0 \\ 0 & 0\end{array}\right] \text{ and } \left[\begin{array}{ll}0 & 1 \\ 0 & 0\end{array}\right] \quad
    \text{ (b) } \left[\begin{array}{ll}1 & 1 \\ 0 & 0\end{array}\right] \quad
    \text{ (c) } \left[\begin{array}{ll}1 & 0 \\ 0 & 0\end{array}\right] \text{ and } \left[\begin{array}{ll}1 & 0 \\ 0 & 1\end{array}\right]
    $$
    
    \item [20.] For which right sides (find a condition on $b_{1}, b_{2}, b_{3}$ ) are these systems solvable?
    $$
    \text{ (a) } \left[\begin{array}{rrr}1 & 4 & 2 \\ 2 & 8 & 4 \\ -1 & -4 & -2\end{array}\right]\left[\begin{array}{l}x_{1} \\ x_{2} \\ x_{3}\end{array}\right]=\left[\begin{array}{l}b_{1} \\ b_{2} \\ b_{3}\end{array}\right]
    \text{ (b) } \left[\begin{array}{rr}1 & 4 \\ 2 & 9 \\ -1 & -4\end{array}\right]\left[\begin{array}{l}x_{1} \\ x_{2}\end{array}\right]=\left[\begin{array}{l}b_{1} \\ b_{2} \\ b_{3}\end{array}\right].
    $$
    
    \item [23.] (Recommended) If we add an extra column $\bm{b}$ to a matrix $A$, then the column space gets larger unless $\dots$. Give an example where the column space gets larger and an example where it doesn't. Why is $A \bm{x}=\bm{b}$ solvable exactly when the column space doesn't get larger - it is the same for $A$ and $\left[\begin{array}{ll}A & \bm{b}\end{array}\right]$ ?
    
    \item [24.] The columns of $A B$ are combinations of the columns of $A$. This means: The column space of $A B$ is contained in (possibly equal to) the column space of $A$. Give an example where the column spaces of $A$ and $A B$ are not equal.
    
\end{enumerate}

\end{document}