\documentclass[main.tex]{subfiles}
\begin{document}

\subsection{Section 3.3}
\begin{enumerate}
    \item [3.] \textbf{Q.} Find the reduced $R$ for each of these (block) matrices:
        $$
        A=\left[\begin{array}{lll}
        0 & 0 & 0 \\
        0 & 0 & 3 \\
        2 & 4 & 6
        \end{array}\right] \quad B=\left[\begin{array}{ll}
        A & A
        \end{array}\right] \quad C=\left[\begin{array}{ll}
        A & A \\
        A & 0
        \end{array}\right]
        $$
        \textbf{A.}
        $$
        \begin{aligned}
        A &= \left[\begin{array}{lll}
        0 & 0 & 0 \\
        0 & 0 & 3 \\
        2 & 4 & 6
        \end{array}\right]\\
        R_1 & \leftrightarrow R_3\\
        &=\left[\begin{array}{lll}
        2 & 4 & 6 \\
        0 & 0 & 3 \\
        0 & 0 & 0
        \end{array}\right]\\
        \frac{1}{2}R_1 & \rightarrow R_1\\
        \frac{1}{3}R_2 & \rightarrow R_2\\
        & = \left[\begin{array}{lll}
        1 & 2 & 3 \\
        0 & 0 & 1 \\
        0 & 0 & 0
        \end{array}\right]\\
        3R_2 - R_1 & \rightarrow R_1\\
        R_{A} &= \left[\begin{array}{lll}
        1 & 2 & 0 \\
        0 & 0 & 1 \\
        0 & 0 & 0
        \end{array}\right]
        \end{aligned}
        $$
        
        $$
        \begin{aligned}
        B &=\left[\begin{array}{ll}
        A & A
        \end{array}\right]\\
        R_{B}&=\left[\begin{array}{llllll}
        1 & 2 & 0 & 1 & 2 & 0 \\
        0 & 0 & 1 & 0 & 0 & 1 \\
        0 & 0 & 0 & 0 & 0 & 0
        \end{array}\right]
        \end{aligned}
        $$
        
        $$
        \begin{aligned}
        C &= \left[\begin{array}{ll}
        A & A \\
        A & 0
        \end{array}\right]\\
        &= \left[\begin{array}{llllll}
        1 & 2 & 0 & 1 & 2 & 0 \\
        0 & 0 & 1 & 0 & 0 & 1 \\
        0 & 0 & 0 & 0 & 0 & 0 \\
        1 & 2 & 0 & 0 & 0 & 0 \\
        0 & 0 & 1 & 0 & 0 & 0 \\
        0 & 0 & 0 & 0 & 0 & 0
        \end{array}\right]\\
        R_1 - R_4 & \rightarrow R_4\\
        R_2 - R_5 & \rightarrow R_5\\
        & = \left[\begin{array}{cccccc}
        1 & 2 & 0 & 1 & 2 & 0 \\
        0 & 0 & 1 & 0 & 0 & 1 \\
        0 & 0 & 0 & 0 & 0 & 0 \\
        0 & 0 & 0 & -1 & -2 & 0 \\
        0 & 0 & 0 & 0 & 0 & -1 \\
        0 & 0 & 0 & 0 & 0 & 0
        \end{array}\right]\\
        R_4 + R_1 & \rightarrow R_1\\
        R_5 + R_2 & \rightarrow R_2\\
        & = \left[\begin{array}{cccccc}
        1 & 2 & 0 & 0 & 0 & 0 \\
        0 & 0 & 1 & 0 & 0 & 0 \\
        0 & 0 & 0 & 0 & 0 & 0 \\
        0 & 0 & 0 & -1 & -2 & 0 \\
        0 & 0 & 0 & 0 & 0 & -1 \\
        0 & 0 & 0 & 0 & 0 & 0
        \end{array}\right]\\
        -1R_4 & \rightarrow R_4\\
        -1R_5 & \rightarrow R_5\\
        R_C & = \left[\begin{array}{llllll}
        1 & 2 & 0 & 0 & 0 & 0 \\
        0 & 0 & 1 & 0 & 0 & 0 \\
        0 & 0 & 0 & 0 & 0 & 0 \\
        0 & 0 & 0 & 1 & 2 & 0 \\
        0 & 0 & 0 & 0 & 0 & 1 \\
        0 & 0 & 0 & 0 & 0 & 0
        \end{array}\right]
        \end{aligned}
        $$
        
    \item [7.] \textbf{Q.} What are the special solutions to $R x=0$ and $y^{\mathrm{T}} R=0$ for these $R$ ?
    $$
    R=\left[\begin{array}{llll}
    1 & 0 & 2 & 3 \\
    0 & 1 & 4 & 5 \\
    0 & 0 & 0 & 0
    \end{array}\right] \quad R=\left[\begin{array}{lll}
    0 & 1 & 2 \\
    0 & 0 & 0 \\
    0 & 0 & 0
    \end{array}\right]
    $$
    \textbf{A.}
    $$
    \begin{aligned}
    R x&=0\\
    \left[\begin{array}{llll}
    1 & 0 & 2 & 3 \\
    0 & 1 & 4 & 5 \\
    0 & 0 & 0 & 0
    \end{array}\right]\left[\begin{array}{l}
    x_{1} \\
    x_{2} \\
    x_{3} \\
    x_{4}
    \end{array}\right]&=\left[\begin{array}{l}
    0 \\
    0 \\
    0
    \end{array}\right]\\
    x_{1}+2 x_{3}+3 x_{4} &= 0 \\
    x_{2}+4 x_{3}+5 x_{4} &= 0 \\
    x_{3} &= k_{1} \\
    x_{4} &= k_{2} \\
    x_{1} &= -\left(2 k_{1}+3 k_{2}\right) \\
    x_{2} &= -\left(4 k_{1}+5 k_{2}\right) \\
    \left[\begin{array}{l}
    x_{1} \\
    x_{2} \\
    x_{3} \\
    x_{4}
    \end{array}\right]&=\left[\begin{array}{c}
    -\left(2 k_{1}+3 k_{2}\right) \\
    -\left(4 k_{1}+5 k_{2}\right) \\
    k_{1} \\
    k_{2}
    \end{array}\right]\\
    \left[\begin{array}{l}
    x_{1} \\
    x_{2} \\
    x_{3} \\
    x_{4}
    \end{array}\right]&=k_{1}\left[\begin{array}{c}
    -2 \\
    -4 \\
    1 \\
    0
    \end{array}\right]+k_{2}\left[\begin{array}{c}
    -3 \\
    -5 \\
    0 \\
    1
    \end{array}\right]\\
    \text{special solutions} &= \left[\begin{array}{c}
    -2 \\
    -4 \\
    1 \\
    0
    \end{array}\right],\left[\begin{array}{c}
    -3 \\
    -5 \\
    0 \\
    1
    \end{array}\right]
    \end{aligned}
    $$
    
    $$
    \begin{aligned}
    \bm{y}^{T} R &= \mathbf{0}\\
    \left(y_{1}, y_{2}, y_{3}\right)\left[\begin{array}{llll}
    1 & 0 & 2 & 3 \\
    0 & 1 & 4 & 5 \\
    0 & 0 & 0 & 0
    \end{array}\right]&=\left[\begin{array}{l}
    0 \\
    0 \\
    0 \\
    0
    \end{array}\right]\\
    y_{1} &= 0 \\
    y_{2} &= 0 \\
    2 y_{1}+4 y_{2} &= 0 \\
    3 y_{1}+5 y_{2} &= 0 \\
    y_3 &= k \text{ free variable }\\
    \left(y_{1}, y_{2}, y_{3}\right)^{T} &= k(0,0,1)\\
    \text{special solution } &= (0,0,1)
    \end{aligned}
    $$
    
    $$
    \begin{aligned}
    R\bm{x} &= 0\\
    \left[\begin{array}{lll}
    0 & 1 & 2 \\
    0 & 0 & 0 \\
    0 & 0 & 0
    \end{array}\right]\left[\begin{array}{l}
    x_{1} \\
    x_{2} \\
    x_{3}
    \end{array}\right]&=\left[\begin{array}{l}
    0 \\
    0 \\
    0
    \end{array}\right]\\
    x_{2}+2 x_{3} &= 0\\
    x_1 &= k_1 \text{ free variable }\\
    x_3 &= k_2 \text{ free variable }\\
    x_{2} &= -2x_{3}\\
    x_{2} &= -2k_{2}\\
    \left[\begin{array}{l}
    x_{1} \\
    x_{2} \\
    x_{3}
    \end{array}\right] &= \left[\begin{array}{c}
    k_{1} \\
    -2 k_{2} \\
    k_{2}
    \end{array}\right]} \\
    \left[\begin{array}{l}
    x_{1} \\
    x_{2} \\
    x_{3}
    \end{array}\right] &= k_{1}\left[\begin{array}{l}
    1 \\
    0 \\
    0
    \end{array}\right]+k_{2}\left[\begin{array}{c}
    0 \\
    -2 \\
    1
    \end{array}\right]\\
    \text{ special solution } &= \left[\begin{array}{l}
    1 \\
    0 \\
    0
    \end{array}\right],\left[\begin{array}{c}
    0 \\
    -2 \\
    1
    \end{array}\right]
    \end{aligned}
    $$
    
    $$
    \begin{aligned}
    \bm{y}^{T} R &= \bm{0}\\
    \left(y_{1}, y_{2}, y_{3}\right)\left[\begin{array}{lll}
    0 & 1 & 2 \\
    0 & 0 & 0 \\
    0 & 0 & 0
    \end{array}\right] &= \left[\begin{array}{l}
    0 \\
    0 \\
    0
    \end{array}\right]\\
    y_{1} &= 0 \\
    2 y_{1} &= 0 \\
    y_{2} &= k_{1} \text{ free variable}\\
    y_{3} &= k_{2} \text{ free variable}\\
    \left(y_{1}, y_{2}, y_{3}\right)^{T} &= \left(0, k_{1}, k_{2}\right) \\
    \left(y_{1}, y_{2}, y_{3}\right)^{T} &= k_{1}(0,1,0)+k_{2}(0,0,1)\\
    \text{special solutions } & = (0,1,0),(0,0,1)
    \end{aligned}
    $$
    
    \item [13.] \textbf{Q.} Suppose $P$ contains only the $r$ pivot columns of an $m$ by $n$ matrix. Explain why this $m$ by $r$ submatrix $P$ has rank $r$. \textbf{A.} $P$ contains only the $r$ pivot columns of an $m$ by $n$ matrix. Therefore, the number of pivots in the matrix $P$ is $r$. Therefore, the rank of this $m$ by $r$ sub matrix is $r$.
    
    \item [20.] \textbf{Q.} If $A$ is 2 by 3 and $B$ is 3 by 2 and $A B=I$, show from its rank that $B A \neq I$. Give an example of $A$ and $B$ with $B A=I$. For $m<n$, a right inverse is not a left inverse. \textbf{A.} $A$ is a 2 by 3 matrix, so the matrix $A$ has at most rank $2$. $B$ is a 3 by 2 matrix, so the matrix $B$ has at most rank $2$. Therefore the product $A B$ has at most rank $2$. $B A$ is a 3 by 3 matrix, so the matrix $B A$ has at most rank 3. Thus, the rank of $B A$ is not equal to rank of $I$. An example is not possible to show.
    
\end{enumerate}

\subsection{Section 3.4}
\begin{enumerate}
    \item [2.] \textbf{Q.} Carry out the same six steps for this matrix $A$ with rank one. You will find two conditions on $b_{1}, b_{2}, b_{3}$ for $A \bm{x}=\bm{b}$ to be solvable. Together these two conditions put $\bm{b}$ into the \rule{1cm}{0.15mm} space (two planes give a line):
    $$
    A=\left[\begin{array}{l}
    1 \\
    3 \\
    2
    \end{array}\right]\left[\begin{array}{lll}
    2 & 1 & 3
    \end{array}\right]=\left[\begin{array}{lll}
    2 & 1 & 3 \\
    6 & 3 & 9 \\
    4 & 2 & 6
    \end{array}\right] \quad b=\left[\begin{array}{l}
    b_{1} \\
    b_{2} \\
    b_{3}
    \end{array}\right]=\left[\begin{array}{l}
    10 \\
    30 \\
    20
    \end{array}\right]
    $$
    \textbf{A.}
    \begin{enumerate}
        \item [Step 1.]
        $$
        \begin{aligned}
        A b &= \left[\begin{array}{lll}
        2 & 1 & 3 \\
        6 & 3 & 9 \\
        4 & 2 & 6
        \end{array}\right] \left[\begin{array}{l}
        b_{1} \\
        b_{2} \\
        b_{3}
        \end{array}\right]\\
        3R_1 - R_2 & \rightarrow R_2\\
        2R_1 - R_3 & \rightarrow R_3\\
        \text{echelon form} & \\
        \left[\begin{array}{ll}
        U & c
        \end{array}\right] &= \left[\begin{array}{cccc}
        2 & 1 & 3 & b_1 \\
        0 & 0 & 0 & b_2 - 3 b_1 \\
        0 & 0 & 0 & b_3 - 2 b_1
        \end{array}\right]
        \end{aligned}
        $$
        
        \item[Step 2.]
        $$
        \begin{aligned}
        b_{2}-3 b_{1} &= 0 \\
        b_{3}-2 b_{1} &= 0
        \end{aligned}
        $$
        
        \item[Step 3.]
        The pivot element is in column 1. The column space is the line containing the combinations of the pivot column $(2,6,4)$. The column space contains all the vectors with 
        $$
        \begin{aligned}
        b_{2}-3 b_{1} &=0 \\
        b_{3}-2 b_{1} &=0
        \end{aligned}
        $$
        
        \item[Step 4.]
        Free variables $x_{2}, x_{3}$. Set
        $$
        \begin{aligned}
        x_{2} &= 1 \\
        x_{3} &= 0 \\
        2 x_{1}+x_{2}+3 x_{3} &= 0 \\
        2 x_{1}+1+3(0) &= 0\\
        x_{1} &= -\frac{1}{2}\\
        \text{special solution }s_{1} &= \left(-\frac{1}{2}, 1,0\right)
        \end{aligned}
        $$
        
        Set
        $$
        \begin{aligned}
        x_{2} &= 0 \\
        x_{3} &= 1 \\
        2 x_{1}+x_{2}+3 x_{3} &= 0 \\
        2 x_{1}+0+3(1) &= 0\\
        x_{1} &= -\frac{3}{2}\\
        \text{special solution }s_{2} &= \left(-\frac{3}{2},0,1\right)
        \end{aligned}
        $$
        
        \item[Step 5.] Particular Solution
        $$
        \begin{aligned}
        A x &= (10,30,20)\\
        \text{substitute in }\left[\begin{array}{ll}
        U & c
        \end{array}\right]\\
        2 x_{1}+x_{2}+3 x_{3} &=10 \\
        2 x_{1}+0+3(0) &=10 \\
        x_{1} &=5\\
        \text{particular solution } \bm{x}_{p}=\left[\begin{array}{l}
        5 \\
        0 \\
        0
        \end{array}\right]
        \end{aligned}
        $$
        
        Complete Solution is $\bm{x}=\bm{x}_{n}+\bm{x}_{p}$
        
        \item[Step 6.] 
        $$
        \begin{aligned}
        \left[\begin{array}{ll}
        U & c
        \end{array}\right]&\\
        \frac{1}{2}R_1 \rightarrow R_1\\
        &=\left[\begin{array}{cccc}
        1 & 1 / 2 & 3 / 2 & \mathbf{b}_{1} / 2 \\
        0 & 0 & 0 & \mathbf{b}_{2}-3 \mathbf{b}_{1} \\
        0 & 0 & 0 & \mathbf{b}_{3}-\mathbf{2} \mathbf{b}_{1}
        \end{array}\right]\\
        &\bm{b}=(10,30,20) \\
        \left[\begin{array}{ll}
        U & \boldsymbol{c}
        \end{array}\right]&=\left[\begin{array}{cccc}
        1 & 1 / 2 & 3 / 2 & 5 \\
        0 & 0 & 0 & 0 \\
        0 & 0 & 0 & 0
        \end{array}\right]\\
        \left[\begin{array}{ll}
        R & d
        \end{array}\right]&=\left[\begin{array}{cccc}
        1 & 1 / 2 & 3 / 2 & 5 \\
        0 & 0 & 0 & 0 \\
        0 & 0 & 0 & 0
        \end{array}\right]\\
        \bm{x}_{p} &= \left[\begin{array}{l}
        5 \\
        0 \\
        0
        \end{array}\right]
        \end{aligned}\\
        $$
        
        
    \end{enumerate}
    
    \item [4.] \textbf{Q.} Find the complete solution (also called the general solution) to
    $$
    \left[\begin{array}{llll}
    1 & 3 & 1 & 2 \\
    2 & 6 & 4 & 8 \\
    0 & 0 & 2 & 4
    \end{array}\right]\left[\begin{array}{l}
    x \\
    y \\
    z \\
    t
    \end{array}\right]=\left[\begin{array}{l}
    1 \\
    3 \\
    1
    \end{array}\right]
    $$
    \textbf{A.}
    
    Special Solution
    $$
    \begin{aligned}
    \boldsymbol{A} &=\left[\begin{array}{llll}
    1 & 3 & 1 & 2 \\
    2 & 6 & 4 & 8 \\
    0 & 0 & 2 & 4
    \end{array}\right] \\
    \boldsymbol{b} &=\left[\begin{array}{l}
    b_{1} \\
    b_{2} \\
    b_{3}
    \end{array}\right]=\left[\begin{array}{l}
    1 \\
    3 \\
    1
    \end{array}\right]\\
    R_2 - 2R_1 & \rightarrow R_2\\  
    & = \left[\begin{array}{ccccc}
    1 & 3 & 1 & 2 & \bm{b}_{1} \\
    0 & 0 & 2 & 4 & \bm{b}_{2}-\bm{2b}_{1} \\
    0 & 0 & 2 & 4 & \bm{b}_{3}
    \end{array}\right]\\
    R_3 - R_2 & \rightarrow R_3\\ 
    \text{echelon form }\left[\begin{array}{ll}
    U & c
    \end{array}\right]&=\left[\begin{array}{ccccc}
    1 & 3 & 1 & 2 & \bm{b}_{1} \\
    0 & 0 & 2 & 4 & \bm{b}_{2}-2 \bm{b}_{1} \\
    0 & 0 & 0 & 0 & \bm{b}_{3}-\bm{b}_{2}+\bm{2} \bm{b}_{1}
    \end{array}\right]\\
    b_{3}-b_{2}+2 b_{1} &= 0 \\    
    \text{ set free variables } & \\
    y &=1 \\
    t &=0 \\
    2 z+4 t &=0 \\
    2 z+4(0) &=0 \\
    z &=0 \\
    x+3 y+z+2 t &=0 \\
    x+3(1)+0+2(0) &=0 \\
    x &=-3 \\
    s_{1}&=(-3,1,0,0)\\
    \text{ set free variables } & \\
    y &= 0 \\
    t &= 1 \\
    2z+4 t &=0 \\
    2z+4(1) &=0 \\
    2z &=-4 \\
    z &=-2 \\
    x+3 y+z+2 t &=0 \\
    x+3(0)+(-2)+2(1) &=0 \\
    x &=0 \\
    s_{2}&=(0,0,-2,1)\\
    \bm{x}_{n}&=y s_{1}+t s_{2}
    \end{aligned}\\
    $$
    
    Particular Solution
    
    $$
    \begin{aligned}
    A x &=(1,3,1)\\
    2 z+4 t &=3-2(1) \\
    2 z+4(0) &=1 \\
    z &=\frac{1}{2} \\
    x+3 y+z+2 t &=1 \\
    x+3(0)+\left(\frac{1}{2}\right)+2(0) &=1 \\
    x &=\frac{1}{2}\\
    x_{p} &= \left[\begin{array}{c}
    \frac{1}{2} \\
    0 \\
    \frac{1}{2} \\
    0
    \end{array}\right]\\
    \end{aligned}
    $$

    General Solution
    $$
    \begin{aligned}
    \bm{x}&=\bm{x}_{n}+\bm{x}_{p}\\
    \bm{x}&=y\left[\begin{array}{c}
    -3 \\
    1 \\
    0 \\
    0
    \end{array}\right]+z\left[\begin{array}{c}
    0 \\
    0 \\
    -2 \\
    1
    \end{array}\right]+\left[\begin{array}{c}
    \frac{1}{2} \\
    0 \\
    \frac{1}{2} \\
    0
    \end{array}\right]
    \end{aligned}
    $$
    
    \item [6.] \textbf{Q.} What conditions on $b_{1}, b_{2}, b_{3}, b_{4}$ make each system solvable? Find $\boldsymbol{x}$ in that case:
    $$
    \left[\begin{array}{ll}
    1 & 2 \\
    2 & 4 \\
    2 & 5 \\
    3 & 9
    \end{array}\right]\left[\begin{array}{l}
    x_{1} \\
    x_{2}
    \end{array}\right]=\left[\begin{array}{l}
    b_{1} \\
    b_{2} \\
    b_{3} \\
    b_{4}
    \end{array}\right] \quad\left[\begin{array}{rrr}
    1 & 2 & 3 \\
    2 & 4 & 6 \\
    2 & 5 & 7 \\
    3 & 9 & 12
    \end{array}\right]\left[\begin{array}{l}
    x_{1} \\
    x_{2} \\
    x_{3}
    \end{array}\right]=\left[\begin{array}{l}
    b_{1} \\
    b_{2} \\
    b_{3} \\
    b_{4}
    \end{array}\right]
    $$
    \textbf{A.}
    
    Echelon Form
    $$
    \begin{aligned}
    \boldsymbol{A} &= \left[\begin{array}{ll}
    1 & 2 \\
    2 & 4 \\
    2 & 5 \\
    3 & 9
    \end{array}\right] \\
    \boldsymbol{b} &= \left[\begin{array}{l}
    b_{1} \\
    b_{2} \\
    b_{3} \\
    b_{4}
    \end{array}\right]\\
    R_2 - 2R_1 &\rightarrow R_2\\
    R_3 - 2R_1 &\rightarrow R_3\\
    R_4 - 3R_1 &\rightarrow R_4\\
    &=\left[\begin{array}{ccc}
    1 & 2 & \mathbf{b}_{1} \\
    0 & 0 & \mathbf{b}_{2}-\mathbf{2} \mathbf{b}_{1} \\
    0 & 1 & \mathbf{b}_{3}-\mathbf{2} \mathbf{b}_{1} \\
    0 & 3 & \mathbf{b}_{4}-\mathbf{3} \mathbf{b}_{1}
    \end{array}\right]\\
    R_2 &\leftrightarrow R_3\\
    &=\left[\begin{array}{ccc}
    1 & 2 & \mathbf{b}_{1} \\
    0 & 1 & \mathbf{b}_{3}-\mathbf{2} \mathbf{b}_{1} \\
    0 & 0 & \mathbf{b}_{2}-\mathbf{2} \mathbf{b}_{1} \\
    0 & 3 & \mathbf{b}_{4}-3 \mathbf{b}_{1}
    \end{array}\right]\\
    R_4 - 3R_2 &\rightarrow R_4\\
    \left[\begin{array}{ll}
    U & c
    \end{array}\right] &= \left[\begin{array}{ccc}
    1 & 2 & \mathbf{b}_{1} \\
    0 & 1 & \mathbf{b}_{3}-\mathbf{2} \mathbf{b}_{1} \\
    0 & 0 & \mathbf{b}_{2}-\mathbf{2} \mathbf{b}_{1} \\
    0 & 0 & \mathbf{b}_{4}-\mathbf{3} \mathbf{b}_{3}+\mathbf{3} \mathbf{b}_{1}
    \end{array}\right]
    \end{aligned}
    $$
    
    Special Solution
    
    $$
    \begin{aligned}
    A x &= b\\
    b_{2}-2 b_{1} &= 0 \\
    b_{4}-3 b_{3}+3 b_{1} &= 0\\
    \text{no free variables} &\\
    \text{therefore no special solution} &
    \end{aligned}
    $$
    
    Particular Solution
    
    $$
    \begin{aligned}
    Ax &= \left(b_{1}, b_{2}, b_{3}, b_{4}\right)\\
    x_{2} &= b_{3}-2 b_{1} \\
    x_{1}+2 x_{2} &=b_{1} \\
    x_{1}+2\left(b_{3}-2 b_{1}\right) &=b_{1} \\
    x_{1} &=5 b_{1}-2 b_{3}
    \end{aligned}
    $$
    
    General Solution
    $$
    \begin{aligned}
    x_{p} &= \left[\begin{array}{c}
    5 b_{1}-2\mathbf{b}_{3} \\
    \mathbf{b}_{3}-2\mathbf{b}_{1}
    \end{array}\right]\\
    x &= x_{n}+x_{p} \\
    x &= \left[\begin{array}{c}
    5 \mathbf{b}_{1}-2\mathbf{b}_{3} \\
    \mathbf{b}_{3}-2\mathbf{b}_{1}
    \end{array}\right]
    \end{aligned}
    $$
    
    2nd Echelon Form
    $$
    \begin{aligned}
    \boldsymbol{A} &=\left[\begin{array}{llc}
    1 & 2 & 3 \\
    2 & 4 & 6 \\
    2 & 5 & 7 \\
    3 & 9 & 12
    \end{array}\right] \\
    \boldsymbol{b} &=\left[\begin{array}{l}
    b_{1} \\
    b_{2} \\
    b_{3} \\
    b_{4}
    \end{array}\right]\\
    R_2 - 2R_1 &\rightarrow R_2\\
    R_3 - 2R_1 &\rightarrow R_3\\
    R_4 - 3R_1 &\rightarrow R_4\\
    &=\left[\begin{array}{cccc}
    1 & 2 & 3 & b_{1} \\
    0 & 0 & 0 & b_{2}-2 b_{1} \\
    0 & 1 & 1 & b_{3}-2 b_{1} \\
    0 & 3 & 3 & b_{4}-3 b_{1}
    \end{array}\right]\\
    R_4 - 3R_3 &\rightarrow R_4\\
    &=\left[\begin{array}{cccc}
    1 & 2 & 3 & b_{1} \\
    0 & 0 & 0 & b_{2}-2 b_{1} \\
    0 & 1 & 1 & b_{3}-2 b_{1} \\
    0 & 0 & 0 & b_{4}-3 b_{3}+3 b_{1}
    \end{array}\right]\\
    R_2 &\leftrightarrow R_3\\
    \left[\begin{array}{ll}
    U & c
    \end{array}\right] &= \left[\begin{array}{cccc}
    1 & 2 & 3 & b_{1} \\
    0 & 1 & 1 & b_{3}- 2 b_{1} \\
    0 & 0 & 0 & b_{2}- 2 b_{1} \\
    0 & 0 & 0 & b_{4}- 3 b_{3}+ 3 b_{1}
    \end{array}\right]
    \end{aligned}
    $$

    Special Solution
    
    $$
    \begin{aligned}
    A x &= b\\
    b_{2}-2 b_{1} &= 0 \\
    b_{4}-3 b_{3}+3 b_{1} &= 0 \\
    \text{free variable } & = x_3 \\
    x_3 &= 1\\
    x_2+x_{3} &= 0 \\
    x_2+1 &= 0 \\
    x_2 &= -1 \\
    x_{1}+2 x_{2}+3 x_{3} &=0 \\
    x_{1}+2(-1)+3(1) &=0 \\
    x_{1} &=-1\\
    s_{1}&=(-1,-1,1)\\
    \bm{x}_{n} &= x_{3} s_{1}\\
    \end{aligned}
    $$
    
    Particular Solution
    
    $$
    \begin{aligned}
    A x &= \left(b_{1}, b_{2}, b_{3}, b_{4}\right)\\
    x_{2} &= b_{3}-2 b_{1}\\
    x_{1}+2 x_{2} &= b_{1} \\
    x_{1}+2\left(b_{3}-2 b_{1}\right) &= b_{1} \\
    x_{1} &= 5 b_{1}-2 b_{3}
    \end{aligned}
    $$
    
    General Solution
    
    $$
    \begin{aligned}
    \bm{x}_{p} &= \left[\begin{array}{c}
    5b_1-2b_3 \\
    b_3-2b_1 \\
    0
    \end{array}\right]\\
    x &= x_{n}+x_{p} \\
    \bm{x} &= x_{3}\left[\begin{array}{c}
    -1 \\
    -1 \\
    1
    \end{array}\right]+\left[\begin{array}{c}
    5b_1-2b_3 \\
    b_3-2b_1 \\
    0
    \end{array}\right]
    \end{aligned}
    $$
    
    \item [18.] \textbf{Q.} Find by elimination the rank of $A$ and also the rank of $A^{\mathrm{T}}$ :
    $A=\left[\begin{array}{rrr}1 & 4 & 0 \\ 2 & 11 & 5 \\ -1 & 2 & 10\end{array}\right]$ and $A=\left[\begin{array}{lll}1 & 0 & 1 \\ 1 & 1 & 2 \\ 1 & 1 & q\end{array}\right]$ (rank depends on $q$). \textbf{A.}
    
    $$
    \begin{aligned}
    A &=\left[\begin{array}{ccc}
    1 & 4 & 0 \\
    2 & 11 & 5 \\
    -1 & 2 & 10
    \end{array}\right]\\
    R_2 - 2R_1 & \rightarrow R_2\\
    R_1 + R_3 & \rightarrow R_3\\
    &= \left[\begin{array}{ccc}
    1 & 4 & 0 \\
    0 & 3 & 5 \\
    0 & 6 & 10
    \end{array}\right]\\
    R_1 - \frac{4}{3}R_2 & \rightarrow R_1\\
    R_3 + 2R_2 & \rightarrow R_3\\
    &=\left[\begin{array}{ccc}
    1 & 0 & -20 / 3 \\
    0 & 3 & 5 \\
    0 & 0 & 0
    \end{array}\right]\\
    \operatorname{rank}(\bm{A})=2\\
    \operatorname{rank}\left(\bm{A}^{T}\right)=2
    \end{aligned}

    Rank depends on $q$
    
    $$
    \begin{aligned}
    A&=\left[\begin{array}{lll}
    1 & 0 & 1 \\
    1 & 1 & 2 \\
    1 & 1 & q
    \end{array}\right]\\
    R_2 - R_1 & \rightarrow R_2\\
    R_3 - R_1 & \rightarrow R_3\\
    &=\left[\begin{array}{ccc}
    1 & 0 & 1 \\
    0 & 1 & 1 \\
    0 & 1 & q-1
    \end{array}\right]\\
    R_3 - R_2 & \rightarrow R_3\\
    &=\left[\begin{array}{ccc}
    1 & 0 & 1 \\
    0 & 1 & 1 \\
    0 & 0 & q-2
    \end{array}\right]\\
    \text{when } q &= 2\\
    \operatorname{rank}(A)&=2\\
    \operatorname{rank}\left(A^{T}\right)&=2\\
    \text{when } q & \neq 2\\
    \operatorname{rank}(A)&=3\\
    \operatorname{rank}\left(A^{T}\right)&=3\\
    \end{aligned}
    $$
    
\end{enumerate}
\end{document}