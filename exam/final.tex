\documentclass[main.tex]{subfiles}
\begin{document}

\begin{enumerate}
\item [1.] \textbf{Q.} Solve differential equation:

$$
\mathrm{dU}(t) / \mathrm{dt}=\left[\begin{array}{ll}
0 & 1 \\
1 & 0
\end{array}\right] \mathrm{U}(t) ; \quad \mathrm{U}(0)=\left[\begin{array}{l}
4 \\
2
\end{array}\right]
$$

solve for $\mathrm{U}(\mathrm{t})$; and evaluate $\mathrm{U}(1)$.

\textbf{A.}

\item[2.] 
    \begin{enumerate}
    \item [a.] \textbf{Q.} Find a basis for the space of all vectors $\left(x_{1}, x_{2}, x_{3}, x_{4}\right)$ that are orthogonal (perpendicular) to both of these vectors:
    $$
    \left[\begin{array}{l}
    1 \\
    2 \\
    3 \\
    1
    \end{array}\right] \quad\left[\begin{array}{l}
    0 \\
    0 \\
    1 \\
    2
    \end{array}\right]
    $$
    \textbf{A.}
    
    \item [b.] \textbf{Q.} If $u, v, w$ are three nonzero vectors in $R^{7}$, what are the possible dimensions of the subspace they span? \textbf{A.}

    \end{enumerate}

\item[3.] Let $T: \mathrm{R}^{2} \rightarrow \mathrm{R}^{2}$ be the linear transformation satisfying
    $$
    T\left(\left[\begin{array}{l}
    1 \\
    0
    \end{array}\right]\right)=\left[\begin{array}{r}
    -4 \\
    3
    \end{array}\right] \quad \text { and } \quad T\left(\left[\begin{array}{c}
    1 \\
    1
    \end{array}\right]\right)=\left[\begin{array}{c}
    -10 \\
    8
    \end{array}\right]
    $$
    \begin{enumerate}
        \item [a.] \textbf{Q.} Find $T\left(\left[\begin{array}{l}0 \\ 1\end{array}\right]\right)$ \textbf{A.}
        
        \item [b.] \textbf{Q.} What is the matrix $A$ expressing $T$ in terms of the standard basis vectors $\left[\begin{array}{l}1 \\ 0\end{array}\right]$ and $\left[\begin{array}{l}0 \\ 1\end{array}\right] \? $ (The same basis is used for the input and the output.) \textbf{A.}
        
        \item [c.] \textbf{Q.} What is the matrix $B$ expressing $T$ in terms of the basis consisting of eigenvectors of $A$ ? (The same basis is used for the input and output.) (There are two possible correct answers, depending on what order you pick the eigenvectors.) \textbf{A.}
    \end{enumerate}

\item[4.] Let $A=\left[\begin{array}{rr}4 & 1 \\ -1 & 4\end{array}\right]$.
    \begin{enumerate}
    \item [1a.] \textbf{Q.} Find the eigenvalues of $A$. \textbf{A.}
    \end{enumerate}

\item[5.] . Each independent question refers to the matrix $A=\left[\begin{array}{cc}4 & 1 \\ d & -4\end{array}\right]$. In each case, find the value of $d$ that makes the statement true (and show your work!).
\begin{enumerate}
    \item [4a.] \textbf{Q.} Give a value for $d$ such that $\left[\begin{array}{l}5 \\ 1\end{array}\right]$ is an eigenvector of $A$. \textbf{A.}
\end{enumerate}

\item[6.] We are given two vectors $a$ and $b$ in $\mathbb{R}^{4}$ :
$$
a=\left[\begin{array}{l}
2 \\
5 \\
2 \\
4
\end{array}\right] \quad b=\left[\begin{array}{l}
1 \\
2 \\
1 \\
0
\end{array}\right]
$$
\begin{enumerate}
    \item [a.] Find the projection $p$ of the vector $b$ onto the line through $a$. Check (!) that the error $e=b-p$ is perpendicular to (what?)
    \item [b.] The subspace $S$ of all vectors in $\mathbb{R}^{4}$ that are perpendicular to this a is 3-dimensional. Find the projection $q$ of $b$ onto this perpendicular subspace $S$. The numerical answer (it doesn't need a big computation!) is $q=\underline{\hspace{2cm}}$.
\end{enumerate} 

\item[7.] Consider the following operations on the space of quadratic polynomials $f(x)=a x^{2}+b x+c$. Which of them are linear transformations? If they are linear transformations, find their matrices in the basis $1, x, x^{2}$. If they are not linear transformations, explain it using the definition of linear transformation.
    \begin{enumerate}
    \item [a.] $T_{1}(f)=f(x)-f(1)$
    \item [b.] $T_{2}(f)=f(x)-1$
    \item [c.] $T_{3}(f)=x-f(1)$.
    \item [d.] $T_{4}(f)=x^{2} f(1 / x)$
    \item [e.] The linear transformation $R: \mathbb{R}^{2} \longrightarrow \mathbb{R}^{2}$ is the reflection with respect to the line $x+y=0$. Find the matrix of $R$ in the standard basis of $\mathbb{R}^{2}$.
    \end{enumerate}

\end{enumerate}

\end{document}